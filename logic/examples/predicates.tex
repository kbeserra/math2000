\guard

\guard

\guard

\begin{defn}
\label{defn:statement}
  A \emph{statement} (or \emph{proposition}) is a sentence that is either true or false, but not both.
\end{defn}


\begin{defn}
\label{defn:predicate}
  A \emph{predicate} is a sentence which contains a finite number of variables which becomes a statement when a value is assigned to each of its variables.
  The \emph{domain} of a predicate variable is the set of possible values which that variable may take.
\end{defn}


\begin{exmp}
\label{exmp:predicates}
  Let $P(x)$ denote
    \begin{center}
      ``$x$ is a student at the University of North Texas.''
    \end{center}
    and $Q(x,y)$ denote
    \begin{center}
      ``$x$ is a student at $y$.''.
    \end{center}
  $x$ is the predicate variable for $P(x)$ while both $x$ and $y$ are predicate variables for $Q(x,y)$.

  Note that when values are substituded for $x$ and $y$ in $P(x)$ and $Q(x,y)$, these sentences become statements.
  For instance if we substitute $x$ for ``Taylor'' and $y$ for``Boise State University'', $P(x)$ and $Q(x,y)$ become the statements
    \begin{center}
      ``Taylor is a student at the University of North Texas.''
    \end{center}
    and
    \begin{center}
      ``Taylor is a student at Boise State University.''
    \end{center}
    respectively.
  Whence, $P(x)$ and $Q(x,y)$ are predicates
\end{exmp}
