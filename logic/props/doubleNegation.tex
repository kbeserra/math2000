\guard

\guard

\guard

\guard

\begin{defn}
\label{defn:statement}
  A \emph{statement} (or \emph{proposition}) is a sentence which is either true or false, but not both.
\end{defn}


% \guard

\input{logic/defns/statement.tex}
\input{logic/defns/truthValue.tex}


\begin{defn}
\label{defn:negationOfStatement}
  Let $p$ be a statement form.
  The \emph{negation} of $p$, written $\neg p$, is the statement form with the opposite truth value of $p$.
\end{defn}

% \guard

\input{logic/defns/statement.tex}
\input{logic/defns/truthValue.tex}


\begin{defn}
\label{defn:disjunctionOfStatement}
  Let $p$ and $q$ be statements.
  The \emph{disjunction} of $p$ and $q$, written $p \vee q$, is the statement that is true when either $p$ or $q$ is true and false precisely when $p$ and $q$ are both false.
\end{defn}

% \guard

\input{logic/defns/statement.tex}
\input{logic/defns/truthValue.tex}


\begin{defn}
\label{defn:conjunctionOfStatement}
  Let $p$ and $q$ be statements forms
  The \emph{conjunction} of $p$ and $q$, written $p \wedge q$, is the statement form that is true precisely when both $p$ and $q$ are true and is otherwise false.
\end{defn}


\begin{defn}
\label{defn:statementForm}
  A \emph{statement form} (or \emph{proposition form}) is an expression made up of statement variables and logical connections (such as $\neg$, $\vee$, or $\wedge$) which when substituting statements for statement variables becomes a statement.
\end{defn}


\begin{defn}
\label{defn:logicallyEquivalentStatementForm}
\index{logically equivalent statement form}
  Two statement forms are said to be \emph{logically equivalent} if, and only if, they have identical truth values for each possible truth value assignment for their statement variables.
  Let $P$ and $Q$ be statement forms, we write $P \equiv Q$ to mean that $P$ is logically equivalent to $Q$.
\end{defn}


\begin{prop}
\label{prop:doubleNegation}
  {\bf (Double negation):}
  The statement form $p$ is logically equivalent to $\neg\neg p$, i.e. $p\equiv \neg\neg p$.
\end{prop}
\begin{proof}
  In the truth table below, we will write $\neg\neg p$ and $\neg(\neg p)$.
  This is to highlight that we are negating the statement form $\neg p$.
  \begin{center}
    \input{ |"python3 logic/scripts/generateTruthTable.py '!p' '!(!p)' "}
  \end{center}
  We note that the column for $p$ is identical to the column for $\neg\neg p$.
  Thus, we can conclude that after substituting a statement for the statement variable $p$ in the statement forms $p$ and $\neg\neg p$ the statement forms will have identical truth values.
  Whence, the statement forms are identical.
\end{proof}

The conclusion of the previous definition is certainly a mouthful, but worth repeating.
Similar to the way statement forms provide a layer of abstraction to statements, examining the truth tables of two statement forms to determine equivalence abstracts away substituting actual statements for statement variables.
