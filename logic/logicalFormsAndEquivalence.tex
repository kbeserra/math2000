\guard

\section{Logical Forms And Equivalence}
\label{sec:logicalFormsAndEquivalence}


\guard

\guard

\begin{defn}
\label{defn:statement}
  A \emph{statement} (or \emph{proposition}) is a sentence that is either true or false, but not both.
\end{defn}


% \guard

\guard

\begin{defn}
\label{defn:statement}
  A \emph{statement} (or \emph{proposition}) is a sentence that is either true or false, but not both.
\end{defn}

\guard

\input{logic/defns/statement.tex}

\begin{defn}
\label{defn:statementTruthValue}
\index{truth value}
  The \emph{truth value} of a given statement is true if that sentence is itself true otherwise, the truth value of that statement is false.
\end{defn}



\begin{defn}
\label{defn:negationOfStatement}
  Let $p$ be a statement.
  The \emph{negation} of $p$, written $\neg p$, is the statement with the opposite truth value.
\end{defn}

Note, the symbol $\neg$ is not the only symbol used to denote negation.
For instance, it is not uncommon to see the symbol $\sim$ used in other logic texts.
Further, many c-like programming-languages will use the symbol $!$ for the same meaning.
The Python programming-language deserves a special call-out on this note, it allows for the use of the symbol `not' for its not symbol (which fits nicely with its use of `and' and `or' for its and \& or logical connectives).

% \guard

\guard

\begin{defn}
\label{defn:statement}
  A \emph{statement} (or \emph{proposition}) is a sentence that is either true or false, but not both.
\end{defn}

\guard

\input{logic/defns/statement.tex}

\begin{defn}
\label{defn:statementTruthValue}
\index{truth value}
  The \emph{truth value} of a given statement is true if that sentence is itself true otherwise, the truth value of that statement is false.
\end{defn}



\begin{defn}
\label{defn:disjunctionOfStatement}
  Let $p$ and $q$ be statements forms.
  The \emph{disjunction} of $p$ and $q$, written $p \vee q$, is the statement form that is true when either $p$ or $q$ is true and false precisely when $p$ and $q$ are both false.
\end{defn}

% \guard

\guard

\begin{defn}
\label{defn:statement}
  A \emph{statement} (or \emph{proposition}) is a sentence that is either true or false, but not both.
\end{defn}

\guard

\input{logic/defns/statement.tex}

\begin{defn}
\label{defn:statementTruthValue}
\index{truth value}
  The \emph{truth value} of a given statement is true if that sentence is itself true otherwise, the truth value of that statement is false.
\end{defn}



\begin{defn}
\label{defn:conjunctionOfStatement}
  Let $p$ and $q$ be statements.
  The \emph{conjunction} of $p$ and $q$, written $p \wedge q$, is the statement that is true precisely when both $p$ and $q$ are true and is otherwise false.
\end{defn}


\begin{defn}
\label{defn:statementForm}
\index{statement form}
  A \emph{statement form} (or \emph{proposition form}) is an expression made up of statement variables and logical connections (such as $\neg$, $\vee$, or $\wedge$) which when substituting statements for statement variables becomes a statement.
\end{defn}

Note, statement forms are acting as {\it Platonic forms} of statements.

\guard

\guard

\begin{defn}
\label{defn:statement}
  A \emph{statement} (or \emph{proposition}) is a sentence that is either true or false, but not both.
\end{defn}

\guard

\guard

\begin{defn}
\label{defn:statement}
  A \emph{statement} (or \emph{proposition}) is a sentence that is either true or false, but not both.
\end{defn}


\begin{defn}
\label{defn:statementTruthValue}
\index{truth value}
  The \emph{truth value} of a given statement is true if that sentence is itself true otherwise, the truth value of that statement is false.
\end{defn}



\begin{defn}
\label{defn:negationOfStatement}
  Let $p$ be a statement.
  The \emph{negation} of $p$, written $\neg p$, is the statement with the opposite truth value.
\end{defn}

Note, the symbol $\neg$ is not the only symbol used to denote negation.
For instance, it is not uncommon to see the symbol $\sim$ used in other logic texts.
Further, many c-like programming-languages will use the symbol $!$ for the same meaning.
The Python programming-language deserves a special call-out on this note, it allows for the use of the symbol `not' for its not symbol (which fits nicely with its use of `and' and `or' for its and \& or logical connectives).


Note, the symbol $\neg$ is not the only symbol used to denote negation.
For instance, it is not uncommon to see the symbol $\sim$ used in other logic texts.
Further, many c-like programming-languages will use the symbol $!$ for the same meaning.
The Python programming-language deserves a special call-out on this note, it allows for the use of the symbol `not' for its not symbol (which fits nicely with its use of `and' and `or' for its and and or logical connectives).

\guard

\guard

\begin{defn}
\label{defn:statement}
  A \emph{statement} (or \emph{proposition}) is a sentence that is either true or false, but not both.
\end{defn}

\guard

\guard

\begin{defn}
\label{defn:statement}
  A \emph{statement} (or \emph{proposition}) is a sentence that is either true or false, but not both.
\end{defn}


\begin{defn}
\label{defn:statementTruthValue}
\index{truth value}
  The \emph{truth value} of a given statement is true if that sentence is itself true otherwise, the truth value of that statement is false.
\end{defn}



\begin{defn}
\label{defn:disjunctionOfStatement}
  Let $p$ and $q$ be statements forms.
  The \emph{disjunction} of $p$ and $q$, written $p \vee q$, is the statement form that is true when either $p$ or $q$ is true and false precisely when $p$ and $q$ are both false.
\end{defn}

\guard

\guard

\begin{defn}
\label{defn:statement}
  A \emph{statement} (or \emph{proposition}) is a sentence that is either true or false, but not both.
\end{defn}

\guard

\guard

\begin{defn}
\label{defn:statement}
  A \emph{statement} (or \emph{proposition}) is a sentence that is either true or false, but not both.
\end{defn}


\begin{defn}
\label{defn:statementTruthValue}
\index{truth value}
  The \emph{truth value} of a given statement is true if that sentence is itself true otherwise, the truth value of that statement is false.
\end{defn}



\begin{defn}
\label{defn:conjunctionOfStatement}
  Let $p$ and $q$ be statements.
  The \emph{conjunction} of $p$ and $q$, written $p \wedge q$, is the statement that is true precisely when both $p$ and $q$ are true and is otherwise false.
\end{defn}


Just as in arithmetic, when more than one logical connective is used, we
\begin{itemize}
  \item perform the operations from left to right,
  \item evaluate parenthetical terms first,
  \item treat $\neg$ similar to a minus sign.
\end{itemize}

\guard

\guard

\begin{defn}
\label{defn:statement}
  A \emph{statement} (or \emph{proposition}) is a sentence that is either true or false, but not both.
\end{defn}


% \guard

\guard

\begin{defn}
\label{defn:statement}
  A \emph{statement} (or \emph{proposition}) is a sentence that is either true or false, but not both.
\end{defn}

\guard

\input{logic/defns/statement.tex}

\begin{defn}
\label{defn:statementTruthValue}
\index{truth value}
  The \emph{truth value} of a given statement is true if that sentence is itself true otherwise, the truth value of that statement is false.
\end{defn}



\begin{defn}
\label{defn:negationOfStatement}
  Let $p$ be a statement.
  The \emph{negation} of $p$, written $\neg p$, is the statement with the opposite truth value.
\end{defn}

Note, the symbol $\neg$ is not the only symbol used to denote negation.
For instance, it is not uncommon to see the symbol $\sim$ used in other logic texts.
Further, many c-like programming-languages will use the symbol $!$ for the same meaning.
The Python programming-language deserves a special call-out on this note, it allows for the use of the symbol `not' for its not symbol (which fits nicely with its use of `and' and `or' for its and \& or logical connectives).

% \guard

\guard

\begin{defn}
\label{defn:statement}
  A \emph{statement} (or \emph{proposition}) is a sentence that is either true or false, but not both.
\end{defn}

\guard

\input{logic/defns/statement.tex}

\begin{defn}
\label{defn:statementTruthValue}
\index{truth value}
  The \emph{truth value} of a given statement is true if that sentence is itself true otherwise, the truth value of that statement is false.
\end{defn}



\begin{defn}
\label{defn:disjunctionOfStatement}
  Let $p$ and $q$ be statements forms.
  The \emph{disjunction} of $p$ and $q$, written $p \vee q$, is the statement form that is true when either $p$ or $q$ is true and false precisely when $p$ and $q$ are both false.
\end{defn}

% \guard

\guard

\begin{defn}
\label{defn:statement}
  A \emph{statement} (or \emph{proposition}) is a sentence that is either true or false, but not both.
\end{defn}

\guard

\input{logic/defns/statement.tex}

\begin{defn}
\label{defn:statementTruthValue}
\index{truth value}
  The \emph{truth value} of a given statement is true if that sentence is itself true otherwise, the truth value of that statement is false.
\end{defn}



\begin{defn}
\label{defn:conjunctionOfStatement}
  Let $p$ and $q$ be statements.
  The \emph{conjunction} of $p$ and $q$, written $p \wedge q$, is the statement that is true precisely when both $p$ and $q$ are true and is otherwise false.
\end{defn}


\begin{defn}
\label{defn:statementForm}
\index{statement form}
  A \emph{statement form} (or \emph{proposition form}) is an expression made up of statement variables and logical connections (such as $\neg$, $\vee$, or $\wedge$) which when substituting statements for statement variables becomes a statement.
\end{defn}


\guard

\guard

\guard

\begin{defn}
\label{defn:statement}
  A \emph{statement} (or \emph{proposition}) is a sentence that is either true or false, but not both.
\end{defn}


% \guard

\input{logic/defns/statement.tex}
\input{logic/defns/truthValue.tex}


\begin{defn}
\label{defn:negationOfStatement}
  Let $p$ be a statement.
  The \emph{negation} of $p$, written $\neg p$, is the statement with the opposite truth value.
\end{defn}

Note, the symbol $\neg$ is not the only symbol used to denote negation.
For instance, it is not uncommon to see the symbol $\sim$ used in other logic texts.
Further, many c-like programming-languages will use the symbol $!$ for the same meaning.
The Python programming-language deserves a special call-out on this note, it allows for the use of the symbol `not' for its not symbol (which fits nicely with its use of `and' and `or' for its and \& or logical connectives).

% \guard

\input{logic/defns/statement.tex}
\input{logic/defns/truthValue.tex}


\begin{defn}
\label{defn:disjunctionOfStatement}
  Let $p$ and $q$ be statements forms.
  The \emph{disjunction} of $p$ and $q$, written $p \vee q$, is the statement form that is true when either $p$ or $q$ is true and false precisely when $p$ and $q$ are both false.
\end{defn}

% \guard

\input{logic/defns/statement.tex}
\input{logic/defns/truthValue.tex}


\begin{defn}
\label{defn:conjunctionOfStatement}
  Let $p$ and $q$ be statements.
  The \emph{conjunction} of $p$ and $q$, written $p \wedge q$, is the statement that is true precisely when both $p$ and $q$ are true and is otherwise false.
\end{defn}


\begin{defn}
\label{defn:statementForm}
\index{statement form}
  A \emph{statement form} (or \emph{proposition form}) is an expression made up of statement variables and logical connections (such as $\neg$, $\vee$, or $\wedge$) which when substituting statements for statement variables becomes a statement.
\end{defn}


\begin{defn}
\label{defn:truthTable}
\index{truth table}
  A \emph{truth table} for a statement form displays the truth values corresponding to every possible combination of truth values for its component statement variables.
\end{defn}


\guard

\guard

\guard

\begin{defn}
\label{defn:statement}
  A \emph{statement} (or \emph{proposition}) is a sentence that is either true or false, but not both.
\end{defn}

\guard

\input{logic/defns/statement.tex}

\begin{defn}
\label{defn:statementTruthValue}
\index{truth value}
  The \emph{truth value} of a given statement is true if that sentence is itself true otherwise, the truth value of that statement is false.
\end{defn}



\begin{defn}
\label{defn:negationOfStatement}
  Let $p$ be a statement.
  The \emph{negation} of $p$, written $\neg p$, is the statement with the opposite truth value.
\end{defn}

Note, the symbol $\neg$ is not the only symbol used to denote negation.
For instance, it is not uncommon to see the symbol $\sim$ used in other logic texts.
Further, many c-like programming-languages will use the symbol $!$ for the same meaning.
The Python programming-language deserves a special call-out on this note, it allows for the use of the symbol `not' for its not symbol (which fits nicely with its use of `and' and `or' for its and \& or logical connectives).

\guard

\guard

\begin{defn}
\label{defn:statement}
  A \emph{statement} (or \emph{proposition}) is a sentence that is either true or false, but not both.
\end{defn}

\guard

\input{logic/defns/statement.tex}

\begin{defn}
\label{defn:statementTruthValue}
\index{truth value}
  The \emph{truth value} of a given statement is true if that sentence is itself true otherwise, the truth value of that statement is false.
\end{defn}



\begin{defn}
\label{defn:disjunctionOfStatement}
  Let $p$ and $q$ be statements forms.
  The \emph{disjunction} of $p$ and $q$, written $p \vee q$, is the statement form that is true when either $p$ or $q$ is true and false precisely when $p$ and $q$ are both false.
\end{defn}

\guard

\guard

\begin{defn}
\label{defn:statement}
  A \emph{statement} (or \emph{proposition}) is a sentence that is either true or false, but not both.
\end{defn}

\guard

\input{logic/defns/statement.tex}

\begin{defn}
\label{defn:statementTruthValue}
\index{truth value}
  The \emph{truth value} of a given statement is true if that sentence is itself true otherwise, the truth value of that statement is false.
\end{defn}



\begin{defn}
\label{defn:conjunctionOfStatement}
  Let $p$ and $q$ be statements.
  The \emph{conjunction} of $p$ and $q$, written $p \wedge q$, is the statement that is true precisely when both $p$ and $q$ are true and is otherwise false.
\end{defn}

\guard

\guard

\begin{defn}
\label{defn:statement}
  A \emph{statement} (or \emph{proposition}) is a sentence that is either true or false, but not both.
\end{defn}


% \guard

\input{logic/defns/statement.tex}
\input{logic/defns/truthValue.tex}


\begin{defn}
\label{defn:negationOfStatement}
  Let $p$ be a statement.
  The \emph{negation} of $p$, written $\neg p$, is the statement with the opposite truth value.
\end{defn}

Note, the symbol $\neg$ is not the only symbol used to denote negation.
For instance, it is not uncommon to see the symbol $\sim$ used in other logic texts.
Further, many c-like programming-languages will use the symbol $!$ for the same meaning.
The Python programming-language deserves a special call-out on this note, it allows for the use of the symbol `not' for its not symbol (which fits nicely with its use of `and' and `or' for its and \& or logical connectives).

% \guard

\input{logic/defns/statement.tex}
\input{logic/defns/truthValue.tex}


\begin{defn}
\label{defn:disjunctionOfStatement}
  Let $p$ and $q$ be statements forms.
  The \emph{disjunction} of $p$ and $q$, written $p \vee q$, is the statement form that is true when either $p$ or $q$ is true and false precisely when $p$ and $q$ are both false.
\end{defn}

% \guard

\input{logic/defns/statement.tex}
\input{logic/defns/truthValue.tex}


\begin{defn}
\label{defn:conjunctionOfStatement}
  Let $p$ and $q$ be statements.
  The \emph{conjunction} of $p$ and $q$, written $p \wedge q$, is the statement that is true precisely when both $p$ and $q$ are true and is otherwise false.
\end{defn}


\begin{defn}
\label{defn:statementForm}
\index{statement form}
  A \emph{statement form} (or \emph{proposition form}) is an expression made up of statement variables and logical connections (such as $\neg$, $\vee$, or $\wedge$) which when substituting statements for statement variables becomes a statement.
\end{defn}

\guard

\guard

\input{logic/defns/statement.tex}

% \input{logic/defns/negation.tex}
% \input{logic/defns/disjunction.tex}
% \input{logic/defns/conjunction.tex}

\begin{defn}
\label{defn:statementForm}
\index{statement form}
  A \emph{statement form} (or \emph{proposition form}) is an expression made up of statement variables and logical connections (such as $\neg$, $\vee$, or $\wedge$) which when substituting statements for statement variables becomes a statement.
\end{defn}


\begin{defn}
\label{defn:truthTable}
\index{truth table}
  A \emph{truth table} for a statement form displays the truth values corresponding to every possible combination of truth values for its component statement variables.
\end{defn}


\begin{exmp}
\label{exmp:notOrAndTruthTables}
Truth tables for logical connectivesL: $\neg$ ( not ), $\vee$ ( or ), and $\wedge$ ( and ).
  \begin{center}
    \hfill
    \hfill
    \input{ |"python3 logic/scripts/generateTruthTable.py '!p' "}
    \hfill
    \input{ |"python3 logic/scripts/generateTruthTable.py 'p | q' "}
    \hfill
    \input{|"python3 logic/scripts/generateTruthTable.py 'p & q' "}
    \hfill
    \hfill
  \end{center}
\end{exmp}

\guard

\guard

\guard

\begin{defn}
\label{defn:statement}
  A \emph{statement} (or \emph{proposition}) is a sentence that is either true or false, but not both.
\end{defn}

\guard

\input{logic/defns/statement.tex}

\begin{defn}
\label{defn:statementTruthValue}
\index{truth value}
  The \emph{truth value} of a given statement is true if that sentence is itself true otherwise, the truth value of that statement is false.
\end{defn}



\begin{defn}
\label{defn:negationOfStatement}
  Let $p$ be a statement.
  The \emph{negation} of $p$, written $\neg p$, is the statement with the opposite truth value.
\end{defn}

Note, the symbol $\neg$ is not the only symbol used to denote negation.
For instance, it is not uncommon to see the symbol $\sim$ used in other logic texts.
Further, many c-like programming-languages will use the symbol $!$ for the same meaning.
The Python programming-language deserves a special call-out on this note, it allows for the use of the symbol `not' for its not symbol (which fits nicely with its use of `and' and `or' for its and \& or logical connectives).

\guard

\guard

\begin{defn}
\label{defn:statement}
  A \emph{statement} (or \emph{proposition}) is a sentence that is either true or false, but not both.
\end{defn}

\guard

\input{logic/defns/statement.tex}

\begin{defn}
\label{defn:statementTruthValue}
\index{truth value}
  The \emph{truth value} of a given statement is true if that sentence is itself true otherwise, the truth value of that statement is false.
\end{defn}



\begin{defn}
\label{defn:disjunctionOfStatement}
  Let $p$ and $q$ be statements forms.
  The \emph{disjunction} of $p$ and $q$, written $p \vee q$, is the statement form that is true when either $p$ or $q$ is true and false precisely when $p$ and $q$ are both false.
\end{defn}

\guard

\guard

\begin{defn}
\label{defn:statement}
  A \emph{statement} (or \emph{proposition}) is a sentence that is either true or false, but not both.
\end{defn}

\guard

\input{logic/defns/statement.tex}

\begin{defn}
\label{defn:statementTruthValue}
\index{truth value}
  The \emph{truth value} of a given statement is true if that sentence is itself true otherwise, the truth value of that statement is false.
\end{defn}



\begin{defn}
\label{defn:conjunctionOfStatement}
  Let $p$ and $q$ be statements.
  The \emph{conjunction} of $p$ and $q$, written $p \wedge q$, is the statement that is true precisely when both $p$ and $q$ are true and is otherwise false.
\end{defn}

\guard

\guard

\begin{defn}
\label{defn:statement}
  A \emph{statement} (or \emph{proposition}) is a sentence that is either true or false, but not both.
\end{defn}


% \guard

\input{logic/defns/statement.tex}
\input{logic/defns/truthValue.tex}


\begin{defn}
\label{defn:negationOfStatement}
  Let $p$ be a statement.
  The \emph{negation} of $p$, written $\neg p$, is the statement with the opposite truth value.
\end{defn}

Note, the symbol $\neg$ is not the only symbol used to denote negation.
For instance, it is not uncommon to see the symbol $\sim$ used in other logic texts.
Further, many c-like programming-languages will use the symbol $!$ for the same meaning.
The Python programming-language deserves a special call-out on this note, it allows for the use of the symbol `not' for its not symbol (which fits nicely with its use of `and' and `or' for its and \& or logical connectives).

% \guard

\input{logic/defns/statement.tex}
\input{logic/defns/truthValue.tex}


\begin{defn}
\label{defn:disjunctionOfStatement}
  Let $p$ and $q$ be statements forms.
  The \emph{disjunction} of $p$ and $q$, written $p \vee q$, is the statement form that is true when either $p$ or $q$ is true and false precisely when $p$ and $q$ are both false.
\end{defn}

% \guard

\input{logic/defns/statement.tex}
\input{logic/defns/truthValue.tex}


\begin{defn}
\label{defn:conjunctionOfStatement}
  Let $p$ and $q$ be statements.
  The \emph{conjunction} of $p$ and $q$, written $p \wedge q$, is the statement that is true precisely when both $p$ and $q$ are true and is otherwise false.
\end{defn}


\begin{defn}
\label{defn:statementForm}
\index{statement form}
  A \emph{statement form} (or \emph{proposition form}) is an expression made up of statement variables and logical connections (such as $\neg$, $\vee$, or $\wedge$) which when substituting statements for statement variables becomes a statement.
\end{defn}

\guard

\guard

\input{logic/defns/statement.tex}

% \input{logic/defns/negation.tex}
% \input{logic/defns/disjunction.tex}
% \input{logic/defns/conjunction.tex}

\begin{defn}
\label{defn:statementForm}
\index{statement form}
  A \emph{statement form} (or \emph{proposition form}) is an expression made up of statement variables and logical connections (such as $\neg$, $\vee$, or $\wedge$) which when substituting statements for statement variables becomes a statement.
\end{defn}


\begin{defn}
\label{defn:truthTable}
\index{truth table}
  A \emph{truth table} for a statement form displays the truth values corresponding to every possible combination of truth values for its component statement variables.
\end{defn}


\begin{exmp}
\label{exmp:notOrAndTruthTablesCombinations}
  Truth table for the statement form $(p\vee q ) \wedge \neg( p \wedge q )$ : \\
  Note, in this example we also include the truth table for the sub-statements forming the larger statement.
  This is not necessary, though this does reduce the risk for error at the cost of space and ink.
    \begin{center}
      % \input{ |"python3 logic/scripts/generateTruthTable.py '(p|q) & !(p&q)'"}
      \input{ |"python3 logic/scripts/generateTruthTable.py 'p|q' 'p&q' '!(p&q)' '(p|q) & !(p&q)'"}
    \end{center}
\end{exmp}

\guard

\guard

\guard

\begin{defn}
\label{defn:statement}
  A \emph{statement} (or \emph{proposition}) is a sentence that is either true or false, but not both.
\end{defn}

\guard

\input{logic/defns/statement.tex}

\begin{defn}
\label{defn:statementTruthValue}
\index{truth value}
  The \emph{truth value} of a given statement is true if that sentence is itself true otherwise, the truth value of that statement is false.
\end{defn}



\begin{defn}
\label{defn:negationOfStatement}
  Let $p$ be a statement.
  The \emph{negation} of $p$, written $\neg p$, is the statement with the opposite truth value.
\end{defn}

Note, the symbol $\neg$ is not the only symbol used to denote negation.
For instance, it is not uncommon to see the symbol $\sim$ used in other logic texts.
Further, many c-like programming-languages will use the symbol $!$ for the same meaning.
The Python programming-language deserves a special call-out on this note, it allows for the use of the symbol `not' for its not symbol (which fits nicely with its use of `and' and `or' for its and \& or logical connectives).

\guard

\guard

\begin{defn}
\label{defn:statement}
  A \emph{statement} (or \emph{proposition}) is a sentence that is either true or false, but not both.
\end{defn}

\guard

\input{logic/defns/statement.tex}

\begin{defn}
\label{defn:statementTruthValue}
\index{truth value}
  The \emph{truth value} of a given statement is true if that sentence is itself true otherwise, the truth value of that statement is false.
\end{defn}



\begin{defn}
\label{defn:disjunctionOfStatement}
  Let $p$ and $q$ be statements forms.
  The \emph{disjunction} of $p$ and $q$, written $p \vee q$, is the statement form that is true when either $p$ or $q$ is true and false precisely when $p$ and $q$ are both false.
\end{defn}

\guard

\guard

\begin{defn}
\label{defn:statement}
  A \emph{statement} (or \emph{proposition}) is a sentence that is either true or false, but not both.
\end{defn}

\guard

\input{logic/defns/statement.tex}

\begin{defn}
\label{defn:statementTruthValue}
\index{truth value}
  The \emph{truth value} of a given statement is true if that sentence is itself true otherwise, the truth value of that statement is false.
\end{defn}



\begin{defn}
\label{defn:conjunctionOfStatement}
  Let $p$ and $q$ be statements.
  The \emph{conjunction} of $p$ and $q$, written $p \wedge q$, is the statement that is true precisely when both $p$ and $q$ are true and is otherwise false.
\end{defn}

\guard

\guard

\begin{defn}
\label{defn:statement}
  A \emph{statement} (or \emph{proposition}) is a sentence that is either true or false, but not both.
\end{defn}


% \guard

\input{logic/defns/statement.tex}
\input{logic/defns/truthValue.tex}


\begin{defn}
\label{defn:negationOfStatement}
  Let $p$ be a statement.
  The \emph{negation} of $p$, written $\neg p$, is the statement with the opposite truth value.
\end{defn}

Note, the symbol $\neg$ is not the only symbol used to denote negation.
For instance, it is not uncommon to see the symbol $\sim$ used in other logic texts.
Further, many c-like programming-languages will use the symbol $!$ for the same meaning.
The Python programming-language deserves a special call-out on this note, it allows for the use of the symbol `not' for its not symbol (which fits nicely with its use of `and' and `or' for its and \& or logical connectives).

% \guard

\input{logic/defns/statement.tex}
\input{logic/defns/truthValue.tex}


\begin{defn}
\label{defn:disjunctionOfStatement}
  Let $p$ and $q$ be statements forms.
  The \emph{disjunction} of $p$ and $q$, written $p \vee q$, is the statement form that is true when either $p$ or $q$ is true and false precisely when $p$ and $q$ are both false.
\end{defn}

% \guard

\input{logic/defns/statement.tex}
\input{logic/defns/truthValue.tex}


\begin{defn}
\label{defn:conjunctionOfStatement}
  Let $p$ and $q$ be statements.
  The \emph{conjunction} of $p$ and $q$, written $p \wedge q$, is the statement that is true precisely when both $p$ and $q$ are true and is otherwise false.
\end{defn}


\begin{defn}
\label{defn:statementForm}
\index{statement form}
  A \emph{statement form} (or \emph{proposition form}) is an expression made up of statement variables and logical connections (such as $\neg$, $\vee$, or $\wedge$) which when substituting statements for statement variables becomes a statement.
\end{defn}

\guard

\guard

\input{logic/defns/statement.tex}

% \input{logic/defns/negation.tex}
% \input{logic/defns/disjunction.tex}
% \input{logic/defns/conjunction.tex}

\begin{defn}
\label{defn:statementForm}
\index{statement form}
  A \emph{statement form} (or \emph{proposition form}) is an expression made up of statement variables and logical connections (such as $\neg$, $\vee$, or $\wedge$) which when substituting statements for statement variables becomes a statement.
\end{defn}


\begin{defn}
\label{defn:truthTable}
\index{truth table}
  A \emph{truth table} for a statement form displays the truth values corresponding to every possible combination of truth values for its component statement variables.
\end{defn}


\begin{exmp}
\label{exmp:notOrAndTruthTable3}
  Write the truth table for the statement form $(p\wedge q) \vee \neg r$ :\\
  Note, in this example, we also include the truth table for the substatements forming the larger statement.
  This is not necessary, though this does reduce the risk for error at the cost of space and ink.
  \begin{center}
    \input{ |"python3 logic/scripts/generateTruthTable.py 'p&q' '!r' '(p&q)|!r'"}
  \end{center}
\end{exmp}


\guard

\guard

\guard

\begin{defn}
\label{defn:statement}
  A \emph{statement} (or \emph{proposition}) is a sentence that is either true or false, but not both.
\end{defn}


% \guard

\input{logic/defns/statement.tex}
\input{logic/defns/truthValue.tex}


\begin{defn}
\label{defn:negationOfStatement}
  Let $p$ be a statement.
  The \emph{negation} of $p$, written $\neg p$, is the statement with the opposite truth value.
\end{defn}

Note, the symbol $\neg$ is not the only symbol used to denote negation.
For instance, it is not uncommon to see the symbol $\sim$ used in other logic texts.
Further, many c-like programming-languages will use the symbol $!$ for the same meaning.
The Python programming-language deserves a special call-out on this note, it allows for the use of the symbol `not' for its not symbol (which fits nicely with its use of `and' and `or' for its and \& or logical connectives).

% \guard

\input{logic/defns/statement.tex}
\input{logic/defns/truthValue.tex}


\begin{defn}
\label{defn:disjunctionOfStatement}
  Let $p$ and $q$ be statements forms.
  The \emph{disjunction} of $p$ and $q$, written $p \vee q$, is the statement form that is true when either $p$ or $q$ is true and false precisely when $p$ and $q$ are both false.
\end{defn}

% \guard

\input{logic/defns/statement.tex}
\input{logic/defns/truthValue.tex}


\begin{defn}
\label{defn:conjunctionOfStatement}
  Let $p$ and $q$ be statements.
  The \emph{conjunction} of $p$ and $q$, written $p \wedge q$, is the statement that is true precisely when both $p$ and $q$ are true and is otherwise false.
\end{defn}


\begin{defn}
\label{defn:statementForm}
\index{statement form}
  A \emph{statement form} (or \emph{proposition form}) is an expression made up of statement variables and logical connections (such as $\neg$, $\vee$, or $\wedge$) which when substituting statements for statement variables becomes a statement.
\end{defn}


\begin{defn}
\label{defn:logicallyEquivalentStatementForm}
  Two statement forms are said to be \emph{logically equivalent} if, and only if, they have identical truth values for each possible truth value assignment for their statement variables.
  Let $P$ and $Q$ be statement forms, we write $P \equiv Q$ to mean that $P$ is logically equivalent to $Q$.
\end{defn}

\guard

\guard

\guard

\input{logic/defns/statement.tex}

% \input{logic/defns/negation.tex}
% \input{logic/defns/disjunction.tex}
% \input{logic/defns/conjunction.tex}

\begin{defn}
\label{defn:statementForm}
\index{statement form}
  A \emph{statement form} (or \emph{proposition form}) is an expression made up of statement variables and logical connections (such as $\neg$, $\vee$, or $\wedge$) which when substituting statements for statement variables becomes a statement.
\end{defn}


\begin{defn}
\label{defn:logicallyEquivalentStatementForm}
  Two statement forms are said to be \emph{logically equivalent} if, and only if, they have identical truth values for each possible truth value assignment for their statement variables.
  Let $P$ and $Q$ be statement forms, we write $P \equiv Q$ to mean that $P$ is logically equivalent to $Q$.
\end{defn}


\begin{defn}
\label{defn:logicallyEquivalentStatement}
  Two statements are said to be \emph{logically equivalent} if, and only if, they have logically equivalent forms after replacing identical component statements with statement variables.
\end{defn}

\guard

\guard

\guard

\input{logic/defns/statement.tex}

% \input{logic/defns/negation.tex}
% \input{logic/defns/disjunction.tex}
% \input{logic/defns/conjunction.tex}

\begin{defn}
\label{defn:statementForm}
\index{statement form}
  A \emph{statement form} (or \emph{proposition form}) is an expression made up of statement variables and logical connections (such as $\neg$, $\vee$, or $\wedge$) which when substituting statements for statement variables becomes a statement.
\end{defn}


\begin{defn}
\label{defn:logicallyEquivalentStatementForm}
  Two statement forms are said to be \emph{logically equivalent} if, and only if, they have identical truth values for each possible truth value assignment for their statement variables.
  Let $P$ and $Q$ be statement forms, we write $P \equiv Q$ to mean that $P$ is logically equivalent to $Q$.
\end{defn}


\begin{prop}
\label{prop:doubleNegation}
  {\bf (Double negation):}
  The statement form $p$ is logically equivalent to $\neg\neg p$, i.e. $p\equiv \neg\neg p$.
\end{prop}
\begin{proof}
  In the truth table below, we will write $\neg\neg p$ and $\neg(\neg p)$.
  This is to highlight that we are negating the statement form $\neg p$.
  \begin{center}
    \input{ |"python3 logic/scripts/generateTruthTable.py '!p' '!(!p)' "}
  \end{center}
  We note that the column for $p$ is identical to the column for $\neg\neg p$.
  Thus, we can conclude that after substituting a statement for the statement variable $p$ in the statement forms $p$ and $\neg\neg p$ the statement forms will have identical truth values.
  Whence, the statement forms are identical.
\end{proof}

The conclusion of the previous definition is certainly a mouthful, but worth repeating.
Similar to the way statement forms provide a layer of abstraction to statements, examining the truth tables of two statement forms to determine equivalence abstracts away substituting actual statements for statement variables.

\guard

\guard

\guard

\input{logic/defns/statement.tex}

% \input{logic/defns/negation.tex}
% \input{logic/defns/disjunction.tex}
% \input{logic/defns/conjunction.tex}

\begin{defn}
\label{defn:statementForm}
\index{statement form}
  A \emph{statement form} (or \emph{proposition form}) is an expression made up of statement variables and logical connections (such as $\neg$, $\vee$, or $\wedge$) which when substituting statements for statement variables becomes a statement.
\end{defn}


\begin{defn}
\label{defn:logicallyEquivalentStatementForm}
  Two statement forms are said to be \emph{logically equivalent} if, and only if, they have identical truth values for each possible truth value assignment for their statement variables.
  Let $P$ and $Q$ be statement forms, we write $P \equiv Q$ to mean that $P$ is logically equivalent to $Q$.
\end{defn}

\guard

\guard

\input{logic/defns/statement.tex}

% \input{logic/defns/negation.tex}
% \input{logic/defns/disjunction.tex}
% \input{logic/defns/conjunction.tex}

\begin{defn}
\label{defn:statementForm}
\index{statement form}
  A \emph{statement form} (or \emph{proposition form}) is an expression made up of statement variables and logical connections (such as $\neg$, $\vee$, or $\wedge$) which when substituting statements for statement variables becomes a statement.
\end{defn}


\begin{defn}
\label{defn:truthTable}
\index{truth table}
  A \emph{truth table} for a statement form displays the truth values corresponding to every possible combination of truth values for its component statement variables.
\end{defn}


\begin{exmp}
  The statement form $\neg(p\wedge q)$ is not equivalent to the statement form $\neg p \wedge \neg q$ ($\neg(p\wedge q) \not\equiv \neg p \wedge \neg q$).
  \begin{center}
    \input{ |"python3 logic/scripts/generateTruthTable.py '!(p&q)' '!p&!q' "}
  \end{center}
  Note the middle two rows of the $\neg(p\wedge q)$ and the $\neg p\wedge\neg q$ do not have the same values.
  These rows are counterexamples to the two statement forms being logically equivalent.
\end{exmp}


\guard

\guard

\guard

\begin{defn}
\label{defn:statement}
  A \emph{statement} (or \emph{proposition}) is a sentence that is either true or false, but not both.
\end{defn}

\guard

\input{logic/defns/statement.tex}

\begin{defn}
\label{defn:statementTruthValue}
\index{truth value}
  The \emph{truth value} of a given statement is true if that sentence is itself true otherwise, the truth value of that statement is false.
\end{defn}



\begin{defn}
\label{defn:negationOfStatement}
  Let $p$ be a statement.
  The \emph{negation} of $p$, written $\neg p$, is the statement with the opposite truth value.
\end{defn}

Note, the symbol $\neg$ is not the only symbol used to denote negation.
For instance, it is not uncommon to see the symbol $\sim$ used in other logic texts.
Further, many c-like programming-languages will use the symbol $!$ for the same meaning.
The Python programming-language deserves a special call-out on this note, it allows for the use of the symbol `not' for its not symbol (which fits nicely with its use of `and' and `or' for its and \& or logical connectives).

\guard

\guard

\begin{defn}
\label{defn:statement}
  A \emph{statement} (or \emph{proposition}) is a sentence that is either true or false, but not both.
\end{defn}

\guard

\input{logic/defns/statement.tex}

\begin{defn}
\label{defn:statementTruthValue}
\index{truth value}
  The \emph{truth value} of a given statement is true if that sentence is itself true otherwise, the truth value of that statement is false.
\end{defn}



\begin{defn}
\label{defn:disjunctionOfStatement}
  Let $p$ and $q$ be statements forms.
  The \emph{disjunction} of $p$ and $q$, written $p \vee q$, is the statement form that is true when either $p$ or $q$ is true and false precisely when $p$ and $q$ are both false.
\end{defn}

\guard

\guard

\begin{defn}
\label{defn:statement}
  A \emph{statement} (or \emph{proposition}) is a sentence that is either true or false, but not both.
\end{defn}

\guard

\input{logic/defns/statement.tex}

\begin{defn}
\label{defn:statementTruthValue}
\index{truth value}
  The \emph{truth value} of a given statement is true if that sentence is itself true otherwise, the truth value of that statement is false.
\end{defn}



\begin{defn}
\label{defn:conjunctionOfStatement}
  Let $p$ and $q$ be statements.
  The \emph{conjunction} of $p$ and $q$, written $p \wedge q$, is the statement that is true precisely when both $p$ and $q$ are true and is otherwise false.
\end{defn}


\guard

\guard

\input{logic/defns/statement.tex}

% \input{logic/defns/negation.tex}
% \input{logic/defns/disjunction.tex}
% \input{logic/defns/conjunction.tex}

\begin{defn}
\label{defn:statementForm}
\index{statement form}
  A \emph{statement form} (or \emph{proposition form}) is an expression made up of statement variables and logical connections (such as $\neg$, $\vee$, or $\wedge$) which when substituting statements for statement variables becomes a statement.
\end{defn}


\begin{defn}
\label{defn:logicallyEquivalentStatementForm}
  Two statement forms are said to be \emph{logically equivalent} if, and only if, they have identical truth values for each possible truth value assignment for their statement variables.
  Let $P$ and $Q$ be statement forms, we write $P \equiv Q$ to mean that $P$ is logically equivalent to $Q$.
\end{defn}

\guard

\guard

\input{logic/defns/statement.tex}

% \input{logic/defns/negation.tex}
% \input{logic/defns/disjunction.tex}
% \input{logic/defns/conjunction.tex}

\begin{defn}
\label{defn:statementForm}
\index{statement form}
  A \emph{statement form} (or \emph{proposition form}) is an expression made up of statement variables and logical connections (such as $\neg$, $\vee$, or $\wedge$) which when substituting statements for statement variables becomes a statement.
\end{defn}


\begin{defn}
\label{defn:truthTable}
\index{truth table}
  A \emph{truth table} for a statement form displays the truth values corresponding to every possible combination of truth values for its component statement variables.
\end{defn}



\begin{prop}
\label{prop:DeMorgans}
    {\bf (DeMorgan's Law):}
    \[ \neg(p\wedge q) \equiv \neg p \vee \neg q \]
    and
    \[ \neg(p\vee q) \equiv \neg p \wedge \neg q\,.\]
\end{prop}
\begin{proof}
  We'll show this by examining the truth tables for each statement form and noting that the relevant columns are identical.
  \begin{center}
    \input{ |"python3 logic/scripts/generateTruthTable.py '!(p&q)' '!p|!q' '!(p|q)' '!p&!q' "}
  \end{center}
\end{proof}

\guard

\guard

\guard

\input{logic/defns/statement.tex}
\input{logic/defns/truthValue.tex}


\begin{defn}
\label{defn:negationOfStatement}
  Let $p$ be a statement.
  The \emph{negation} of $p$, written $\neg p$, is the statement with the opposite truth value.
\end{defn}

Note, the symbol $\neg$ is not the only symbol used to denote negation.
For instance, it is not uncommon to see the symbol $\sim$ used in other logic texts.
Further, many c-like programming-languages will use the symbol $!$ for the same meaning.
The Python programming-language deserves a special call-out on this note, it allows for the use of the symbol `not' for its not symbol (which fits nicely with its use of `and' and `or' for its and \& or logical connectives).

\guard

\input{logic/defns/statement.tex}
\input{logic/defns/truthValue.tex}


\begin{defn}
\label{defn:disjunctionOfStatement}
  Let $p$ and $q$ be statements forms.
  The \emph{disjunction} of $p$ and $q$, written $p \vee q$, is the statement form that is true when either $p$ or $q$ is true and false precisely when $p$ and $q$ are both false.
\end{defn}

\guard

\input{logic/defns/statement.tex}
\input{logic/defns/truthValue.tex}


\begin{defn}
\label{defn:conjunctionOfStatement}
  Let $p$ and $q$ be statements.
  The \emph{conjunction} of $p$ and $q$, written $p \wedge q$, is the statement that is true precisely when both $p$ and $q$ are true and is otherwise false.
\end{defn}


\guard

\input{logic/defns/statementForm.tex}

\begin{defn}
\label{defn:logicallyEquivalentStatementForm}
  Two statement forms are said to be \emph{logically equivalent} if, and only if, they have identical truth values for each possible truth value assignment for their statement variables.
  Let $P$ and $Q$ be statement forms, we write $P \equiv Q$ to mean that $P$ is logically equivalent to $Q$.
\end{defn}

\guard

\input{logic/defns/statementForm.tex}

\begin{defn}
\label{defn:truthTable}
\index{truth table}
  A \emph{truth table} for a statement form displays the truth values corresponding to every possible combination of truth values for its component statement variables.
\end{defn}



\begin{prop}
\label{prop:DeMorgans}
    {\bf (DeMorgan's Law):}
    \[ \neg(p\wedge q) \equiv \neg p \vee \neg q \]
    and
    \[ \neg(p\vee q) \equiv \neg p \wedge \neg q\,.\]
\end{prop}
\begin{proof}
  We'll show this by examining the truth tables for each statement form and noting that the relevant columns are identical.
  \begin{center}
    \input{ |"python3 logic/scripts/generateTruthTable.py '!(p&q)' '!p|!q' '!(p|q)' '!p&!q' "}
  \end{center}
\end{proof}


\begin{exercise}
  Use DeMorgan's, Proposition \ref{prop:DeMorgans}, to write the negation of the following statements:
  \begin{enumerate}
    \item John is $6$ feet tall or he weighs less than $200$ pounds.
    \item The bus was late or Tom's watch was slow.
    \item $-1\leq x\leq 4$.
  \end{enumerate}
\end{exercise}


\guard

\guard

\guard

\begin{defn}
\label{defn:statement}
  A \emph{statement} (or \emph{proposition}) is a sentence that is either true or false, but not both.
\end{defn}


% \guard

\input{logic/defns/statement.tex}
\input{logic/defns/truthValue.tex}


\begin{defn}
\label{defn:negationOfStatement}
  Let $p$ be a statement.
  The \emph{negation} of $p$, written $\neg p$, is the statement with the opposite truth value.
\end{defn}

Note, the symbol $\neg$ is not the only symbol used to denote negation.
For instance, it is not uncommon to see the symbol $\sim$ used in other logic texts.
Further, many c-like programming-languages will use the symbol $!$ for the same meaning.
The Python programming-language deserves a special call-out on this note, it allows for the use of the symbol `not' for its not symbol (which fits nicely with its use of `and' and `or' for its and \& or logical connectives).

% \guard

\input{logic/defns/statement.tex}
\input{logic/defns/truthValue.tex}


\begin{defn}
\label{defn:disjunctionOfStatement}
  Let $p$ and $q$ be statements forms.
  The \emph{disjunction} of $p$ and $q$, written $p \vee q$, is the statement form that is true when either $p$ or $q$ is true and false precisely when $p$ and $q$ are both false.
\end{defn}

% \guard

\input{logic/defns/statement.tex}
\input{logic/defns/truthValue.tex}


\begin{defn}
\label{defn:conjunctionOfStatement}
  Let $p$ and $q$ be statements.
  The \emph{conjunction} of $p$ and $q$, written $p \wedge q$, is the statement that is true precisely when both $p$ and $q$ are true and is otherwise false.
\end{defn}


\begin{defn}
\label{defn:statementForm}
\index{statement form}
  A \emph{statement form} (or \emph{proposition form}) is an expression made up of statement variables and logical connections (such as $\neg$, $\vee$, or $\wedge$) which when substituting statements for statement variables becomes a statement.
\end{defn}


\begin{defn}
\label{defn:tautology}
  A \emph{tautology} is a statement form that is true independent of the truth value assignments of its truth value assignments.
  A statement whose statement form is a tautology is a \emph{tautological statement}.
\end{defn}

\guard

\guard

\guard

\begin{defn}
\label{defn:statement}
  A \emph{statement} (or \emph{proposition}) is a sentence that is either true or false, but not both.
\end{defn}


% \guard

\input{logic/defns/statement.tex}
\input{logic/defns/truthValue.tex}


\begin{defn}
\label{defn:negationOfStatement}
  Let $p$ be a statement.
  The \emph{negation} of $p$, written $\neg p$, is the statement with the opposite truth value.
\end{defn}

Note, the symbol $\neg$ is not the only symbol used to denote negation.
For instance, it is not uncommon to see the symbol $\sim$ used in other logic texts.
Further, many c-like programming-languages will use the symbol $!$ for the same meaning.
The Python programming-language deserves a special call-out on this note, it allows for the use of the symbol `not' for its not symbol (which fits nicely with its use of `and' and `or' for its and \& or logical connectives).

% \guard

\input{logic/defns/statement.tex}
\input{logic/defns/truthValue.tex}


\begin{defn}
\label{defn:disjunctionOfStatement}
  Let $p$ and $q$ be statements forms.
  The \emph{disjunction} of $p$ and $q$, written $p \vee q$, is the statement form that is true when either $p$ or $q$ is true and false precisely when $p$ and $q$ are both false.
\end{defn}

% \guard

\input{logic/defns/statement.tex}
\input{logic/defns/truthValue.tex}


\begin{defn}
\label{defn:conjunctionOfStatement}
  Let $p$ and $q$ be statements.
  The \emph{conjunction} of $p$ and $q$, written $p \wedge q$, is the statement that is true precisely when both $p$ and $q$ are true and is otherwise false.
\end{defn}


\begin{defn}
\label{defn:statementForm}
\index{statement form}
  A \emph{statement form} (or \emph{proposition form}) is an expression made up of statement variables and logical connections (such as $\neg$, $\vee$, or $\wedge$) which when substituting statements for statement variables becomes a statement.
\end{defn}


\begin{defn}
\label{defn:contradiction}
  A \emph{contradiction} is a statement form that is false independent of the truth value assignments of its truth value assignments.
  A statement whose statement form is a contradiction is a \emph{contradictory statement}.
\end{defn}


\guard


\guard

\guard

\begin{defn}
\label{defn:statement}
  A \emph{statement} (or \emph{proposition}) is a sentence that is either true or false, but not both.
\end{defn}


% \guard

\input{logic/defns/statement.tex}
\input{logic/defns/truthValue.tex}


\begin{defn}
\label{defn:negationOfStatement}
  Let $p$ be a statement.
  The \emph{negation} of $p$, written $\neg p$, is the statement with the opposite truth value.
\end{defn}

Note, the symbol $\neg$ is not the only symbol used to denote negation.
For instance, it is not uncommon to see the symbol $\sim$ used in other logic texts.
Further, many c-like programming-languages will use the symbol $!$ for the same meaning.
The Python programming-language deserves a special call-out on this note, it allows for the use of the symbol `not' for its not symbol (which fits nicely with its use of `and' and `or' for its and \& or logical connectives).

% \guard

\input{logic/defns/statement.tex}
\input{logic/defns/truthValue.tex}


\begin{defn}
\label{defn:disjunctionOfStatement}
  Let $p$ and $q$ be statements forms.
  The \emph{disjunction} of $p$ and $q$, written $p \vee q$, is the statement form that is true when either $p$ or $q$ is true and false precisely when $p$ and $q$ are both false.
\end{defn}

% \guard

\input{logic/defns/statement.tex}
\input{logic/defns/truthValue.tex}


\begin{defn}
\label{defn:conjunctionOfStatement}
  Let $p$ and $q$ be statements.
  The \emph{conjunction} of $p$ and $q$, written $p \wedge q$, is the statement that is true precisely when both $p$ and $q$ are true and is otherwise false.
\end{defn}


\begin{defn}
\label{defn:statementForm}
\index{statement form}
  A \emph{statement form} (or \emph{proposition form}) is an expression made up of statement variables and logical connections (such as $\neg$, $\vee$, or $\wedge$) which when substituting statements for statement variables becomes a statement.
\end{defn}

\guard

\guard

\input{logic/defns/statement.tex}

% \input{logic/defns/negation.tex}
% \input{logic/defns/disjunction.tex}
% \input{logic/defns/conjunction.tex}

\begin{defn}
\label{defn:statementForm}
\index{statement form}
  A \emph{statement form} (or \emph{proposition form}) is an expression made up of statement variables and logical connections (such as $\neg$, $\vee$, or $\wedge$) which when substituting statements for statement variables becomes a statement.
\end{defn}


\begin{defn}
\label{defn:tautology}
  A \emph{tautology} is a statement form that is true independent of the truth value assignments of its truth value assignments.
  A statement whose statement form is a tautology is a \emph{tautological statement}.
\end{defn}

\guard

\guard

\input{logic/defns/statement.tex}

% \input{logic/defns/negation.tex}
% \input{logic/defns/disjunction.tex}
% \input{logic/defns/conjunction.tex}

\begin{defn}
\label{defn:statementForm}
\index{statement form}
  A \emph{statement form} (or \emph{proposition form}) is an expression made up of statement variables and logical connections (such as $\neg$, $\vee$, or $\wedge$) which when substituting statements for statement variables becomes a statement.
\end{defn}


\begin{defn}
\label{defn:contradiction}
  A \emph{contradiction} is a statement form that is false independent of the truth value assignments of its truth value assignments.
  A statement whose statement form is a contradiction is a \emph{contradictory statement}.
\end{defn}


\guard

\guard

\input{logic/defns/statement.tex}

% \input{logic/defns/negation.tex}
% \input{logic/defns/disjunction.tex}
% \input{logic/defns/conjunction.tex}

\begin{defn}
\label{defn:statementForm}
\index{statement form}
  A \emph{statement form} (or \emph{proposition form}) is an expression made up of statement variables and logical connections (such as $\neg$, $\vee$, or $\wedge$) which when substituting statements for statement variables becomes a statement.
\end{defn}


\begin{defn}
\label{defn:logicallyEquivalentStatementForm}
  Two statement forms are said to be \emph{logically equivalent} if, and only if, they have identical truth values for each possible truth value assignment for their statement variables.
  Let $P$ and $Q$ be statement forms, we write $P \equiv Q$ to mean that $P$ is logically equivalent to $Q$.
\end{defn}


\guard

\guard

\input{logic/defns/statementForm.tex}

\begin{defn}
\label{defn:logicallyEquivalentStatementForm}
  Two statement forms are said to be \emph{logically equivalent} if, and only if, they have identical truth values for each possible truth value assignment for their statement variables.
  Let $P$ and $Q$ be statement forms, we write $P \equiv Q$ to mean that $P$ is logically equivalent to $Q$.
\end{defn}


\begin{prop}
\label{prop:doubleNegation}
  {\bf (Double negation):}
  The statement form $p$ is logically equivalent to $\neg\neg p$, i.e. $p\equiv \neg\neg p$.
\end{prop}
\begin{proof}
  In the truth table below, we will write $\neg\neg p$ and $\neg(\neg p)$.
  This is to highlight that we are negating the statement form $\neg p$.
  \begin{center}
    \input{ |"python3 logic/scripts/generateTruthTable.py '!p' '!(!p)' "}
  \end{center}
  We note that the column for $p$ is identical to the column for $\neg\neg p$.
  Thus, we can conclude that after substituting a statement for the statement variable $p$ in the statement forms $p$ and $\neg\neg p$ the statement forms will have identical truth values.
  Whence, the statement forms are identical.
\end{proof}

The conclusion of the previous definition is certainly a mouthful, but worth repeating.
Similar to the way statement forms provide a layer of abstraction to statements, examining the truth tables of two statement forms to determine equivalence abstracts away substituting actual statements for statement variables.

\guard

\guard

\input{logic/defns/statement.tex}
\input{logic/defns/truthValue.tex}


\begin{defn}
\label{defn:negationOfStatement}
  Let $p$ be a statement.
  The \emph{negation} of $p$, written $\neg p$, is the statement with the opposite truth value.
\end{defn}

Note, the symbol $\neg$ is not the only symbol used to denote negation.
For instance, it is not uncommon to see the symbol $\sim$ used in other logic texts.
Further, many c-like programming-languages will use the symbol $!$ for the same meaning.
The Python programming-language deserves a special call-out on this note, it allows for the use of the symbol `not' for its not symbol (which fits nicely with its use of `and' and `or' for its and \& or logical connectives).

\guard

\input{logic/defns/statement.tex}
\input{logic/defns/truthValue.tex}


\begin{defn}
\label{defn:disjunctionOfStatement}
  Let $p$ and $q$ be statements forms.
  The \emph{disjunction} of $p$ and $q$, written $p \vee q$, is the statement form that is true when either $p$ or $q$ is true and false precisely when $p$ and $q$ are both false.
\end{defn}

\guard

\input{logic/defns/statement.tex}
\input{logic/defns/truthValue.tex}


\begin{defn}
\label{defn:conjunctionOfStatement}
  Let $p$ and $q$ be statements.
  The \emph{conjunction} of $p$ and $q$, written $p \wedge q$, is the statement that is true precisely when both $p$ and $q$ are true and is otherwise false.
\end{defn}


\guard

\input{logic/defns/statementForm.tex}

\begin{defn}
\label{defn:logicallyEquivalentStatementForm}
  Two statement forms are said to be \emph{logically equivalent} if, and only if, they have identical truth values for each possible truth value assignment for their statement variables.
  Let $P$ and $Q$ be statement forms, we write $P \equiv Q$ to mean that $P$ is logically equivalent to $Q$.
\end{defn}

\guard

\input{logic/defns/statementForm.tex}

\begin{defn}
\label{defn:truthTable}
\index{truth table}
  A \emph{truth table} for a statement form displays the truth values corresponding to every possible combination of truth values for its component statement variables.
\end{defn}



\begin{prop}
\label{prop:DeMorgans}
    {\bf (DeMorgan's Law):}
    \[ \neg(p\wedge q) \equiv \neg p \vee \neg q \]
    and
    \[ \neg(p\vee q) \equiv \neg p \wedge \neg q\,.\]
\end{prop}
\begin{proof}
  We'll show this by examining the truth tables for each statement form and noting that the relevant columns are identical.
  \begin{center}
    \input{ |"python3 logic/scripts/generateTruthTable.py '!(p&q)' '!p|!q' '!(p|q)' '!p&!q' "}
  \end{center}
\end{proof}


\begin{thm}
\label{thm:logicalEquivalences}
  Let $p$, $q$, and $r$ be statement variables, $\tau$ a tautology, and $c$ a contradiction.
  Then the following hold:
  \begin{enumerate}
    \item Commutativity: $p\wedge q\equiv q\wedge p$ and $p\vee q\equiv q\vee p$.
    \item Associativity: $(p\wedge q)\wedge r \equiv p\wedge(q\wedge r)$ and $(p\vee q)\vee r \equiv p\vee(q\vee r)$.
    \item Distribution: $p\wedge(q\vee r)\equiv (p\wedge q)\vee(p\wedge r)$ and $p\vee(q\wedge r)\equiv (p\vee q)\wedge(p\vee r)$.
    \item Identity: $p\wedge \tau\equiv p$ and $p\vee c \equiv p$.
    \item Negation: $p\wedge\neg p \equiv c$ and $p\vee\neg p \equiv \tau$.
    \item Double negation: $\neg\neg p \equiv p$.
    \item Idempotent: $p\wedge p\equiv p$ and $p\vee p \equiv p$.
    \item Universal bound: $p\vee\tau\equiv\tau$ and $p\wedge c\equiv c $.
    \item DeMorgan's Law: $\neg(p\wedge q) \equiv \neg p \vee \neg q$ and $\neg(p\vee q) \equiv \neg p \wedge \neg q$.
    \item Absorption: $p\vee(p\wedge q)\equiv p$ and $p\wedge(p\vee q)\equiv q$.
    \item Negation of $\tau$ and $c$: $\neg\tau\equiv c$ and $\neg c \equiv \tau$.
  \end{enumerate}
\end{thm}
\begin{proof}
  Claims (6) and (9) have been proved in Proposition \ref{prop:doubleNegation} and Proposition \ref{prop:DeMorgans} respectively.
  The remaining claims are left as an exercise to the reader.
\end{proof}

\guard

\guard

\input{logic/defns/statement.tex}

% \input{logic/defns/negation.tex}
% \input{logic/defns/disjunction.tex}
% \input{logic/defns/conjunction.tex}

\begin{defn}
\label{defn:statementForm}
\index{statement form}
  A \emph{statement form} (or \emph{proposition form}) is an expression made up of statement variables and logical connections (such as $\neg$, $\vee$, or $\wedge$) which when substituting statements for statement variables becomes a statement.
\end{defn}

\guard

\input{logic/defns/statementForm.tex}

\begin{defn}
\label{defn:tautology}
  A \emph{tautology} is a statement form that is true independent of the truth value assignments of its truth value assignments.
  A statement whose statement form is a tautology is a \emph{tautological statement}.
\end{defn}

\guard

\input{logic/defns/statementForm.tex}

\begin{defn}
\label{defn:contradiction}
  A \emph{contradiction} is a statement form that is false independent of the truth value assignments of its truth value assignments.
  A statement whose statement form is a contradiction is a \emph{contradictory statement}.
\end{defn}


\guard

\input{logic/defns/statementForm.tex}

\begin{defn}
\label{defn:logicallyEquivalentStatementForm}
  Two statement forms are said to be \emph{logically equivalent} if, and only if, they have identical truth values for each possible truth value assignment for their statement variables.
  Let $P$ and $Q$ be statement forms, we write $P \equiv Q$ to mean that $P$ is logically equivalent to $Q$.
\end{defn}


\guard

\input{logic/defns/logicallyEquivalentStatementForm.tex}

\begin{prop}
\label{prop:doubleNegation}
  {\bf (Double negation):}
  The statement form $p$ is logically equivalent to $\neg\neg p$, i.e. $p\equiv \neg\neg p$.
\end{prop}
\begin{proof}
  In the truth table below, we will write $\neg\neg p$ and $\neg(\neg p)$.
  This is to highlight that we are negating the statement form $\neg p$.
  \begin{center}
    \input{ |"python3 logic/scripts/generateTruthTable.py '!p' '!(!p)' "}
  \end{center}
  We note that the column for $p$ is identical to the column for $\neg\neg p$.
  Thus, we can conclude that after substituting a statement for the statement variable $p$ in the statement forms $p$ and $\neg\neg p$ the statement forms will have identical truth values.
  Whence, the statement forms are identical.
\end{proof}

The conclusion of the previous definition is certainly a mouthful, but worth repeating.
Similar to the way statement forms provide a layer of abstraction to statements, examining the truth tables of two statement forms to determine equivalence abstracts away substituting actual statements for statement variables.

\guard

\input{logic/defns/negation.tex}
\input{logic/defns/disjunction.tex}
\input{logic/defns/conjunction.tex}

\input{logic/defns/logicallyEquivalentStatementForm.tex}
\input{logic/defns/truthTable.tex}


\begin{prop}
\label{prop:DeMorgans}
    {\bf (DeMorgan's Law):}
    \[ \neg(p\wedge q) \equiv \neg p \vee \neg q \]
    and
    \[ \neg(p\vee q) \equiv \neg p \wedge \neg q\,.\]
\end{prop}
\begin{proof}
  We'll show this by examining the truth tables for each statement form and noting that the relevant columns are identical.
  \begin{center}
    \input{ |"python3 logic/scripts/generateTruthTable.py '!(p&q)' '!p|!q' '!(p|q)' '!p&!q' "}
  \end{center}
\end{proof}


\guard


\input{logic/defns/statementForm.tex}
\input{logic/defns/tautology.tex}
\input{logic/defns/contradiction.tex}

\input{logic/defns/logicallyEquivalentStatementForm.tex}

\input{logic/props/doubleNegation.tex}
\input{logic/props/deMorgans.tex}

\begin{thm}
\label{thm:logicalEquivalences}
  Let $p$, $q$, and $r$ be statement variables, $\tau$ a tautology, and $c$ a contradiction.
  Then the following hold:
  \begin{enumerate}
    \item Commutativity: $p\wedge q\equiv q\wedge p$ and $p\vee q\equiv q\vee p$.
    \item Associativity: $(p\wedge q)\wedge r \equiv p\wedge(q\wedge r)$ and $(p\vee q)\vee r \equiv p\vee(q\vee r)$.
    \item Distribution: $p\wedge(q\vee r)\equiv (p\wedge q)\vee(p\wedge r)$ and $p\vee(q\wedge r)\equiv (p\vee q)\wedge(p\vee r)$.
    \item Identity: $p\wedge \tau\equiv p$ and $p\vee c \equiv p$.
    \item Negation: $p\wedge\neg p \equiv c$ and $p\vee\neg p \equiv \tau$.
    \item Double negation: $\neg\neg p \equiv p$.
    \item Idempotent: $p\wedge p\equiv p$ and $p\vee p \equiv p$.
    \item Universal bound: $p\vee\tau\equiv\tau$ and $p\wedge c\equiv c $.
    \item DeMorgan's Law: $\neg(p\wedge q) \equiv \neg p \vee \neg q$ and $\neg(p\vee q) \equiv \neg p \wedge \neg q$.
    \item Absorption: $p\vee(p\wedge q)\equiv p$ and $p\wedge(p\vee q)\equiv q$.
    \item Negation of $\tau$ and $c$: $\neg\tau\equiv c$ and $\neg c \equiv \tau$.
  \end{enumerate}
\end{thm}
\begin{proof}
  Claims (6) and (9) have been proved in Proposition \ref{prop:doubleNegation} and Proposition \ref{prop:DeMorgans} respectively.
  The remaining claims are left as an exercise to the reader.
\end{proof}

\input{logic/exercises/logicalEquivalences.tex}


\begin{exercise}
  Use truth tables to show that claims of Theorem \ref{thm:logicalEquivalences}.
\end{exercise}


