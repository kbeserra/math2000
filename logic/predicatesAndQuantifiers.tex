\guard

\section{Predicates and Quantifiers}
\label{sec:predicatesAndQuantifiers}

Note the sentence
  \begin{center}
    ``They are a college student.''
  \end{center}
  is not a statement as it may be true or false depending on the value of ``They''.
In this sentence ``They'' is a free variable.
Similarly, ``$x+y>0$'' is not a statement, it has more familiar variables $x$ and $y$.

The predicate refers to the part of the sentence with some or all of its nouns removed.
That is, in the sentence
  \begin{center}
    ``James is a student at the University of North Texas.'',
  \end{center}
  ``James'' is the subject and
  \begin{center}
    ``is a student at the University of North Texas.''
  \end{center}
  is the predicate.

In logic, predicates are formed in much the same way.
\guard

\guard

\guard

\begin{defn}
\label{defn:statement}
  A \emph{statement} (or \emph{proposition}) is a sentence that is either true or false, but not both.
\end{defn}


\begin{defn}
\label{defn:predicate}
  A \emph{predicate} is a sentence which contains a finite number of variables which becomes a statement when a value is assigned to each of its variables.
  The \emph{domain} of a predicate variable is the set of possible values which that variable may take.
\end{defn}


\begin{exmp}
\label{exmp:predicates}
  Let $P(x)$ denote
    \begin{center}
      ``$x$ is a student at the University of North Texas.''
    \end{center}
    and $Q(x,y)$ denote
    \begin{center}
      ``$x$ is a student at $y$.''.
    \end{center}
  $x$ is the predicate variable for $P(x)$ while both $x$ and $y$ are predicate variables for $Q(x,y)$.

  Note that when values are substituted for $x$ and $y$ in $P(x)$ and $Q(x,y)$, these sentences become statements.
  For instance if we substitute $x$ for ``Taylor'' and $y$ for``Boise State University'', $P(x)$ and $Q(x,y)$ become the statements
    \begin{center}
      ``Taylor is a student at the University of North Texas.''
    \end{center}
    and
    \begin{center}
      ``Taylor is a student at Boise State University.''
    \end{center}
    respectively.
  Whence, $P(x)$ and $Q(x,y)$ are predicates
\end{exmp}

In Example \ref{exmp:predicates}, we referred to the example's predicates with their full names, $P(x)$ and $Q(x,y)$.
When there is no room for confusion and the variables or other ornaments are of little importance to the statement being made, we may drop those decorations from the symbol.
That is, the concluding sentence of Example \ref{exmp:predicates} could be written ``Whence, $P$ and $Q$ are predicates.''.


\guard

\guard

\guard

\input{logic/defns/statement.tex}

\begin{defn}
\label{defn:predicate}
  A \emph{predicate} is a sentence which contains a finite number of variables which becomes a statement when a value is assigned to each of its variables.
  The \emph{domain} of a predicate variable is the set of possible values which that variable may take.
\end{defn}


\begin{exmp}
\label{exmp:predicateOverReals}
  Let $P(x)$ denote ``$x^2 > x$'' with $\RR$, the set of real numbers, as the domain of $x$.

  $P(2)$ and $P(-\frac{1}{2})$ are true.
  This is because $P(2)$ and $P(-\frac{1}{2})$ denote the statement ``$4 > 2$'' and ``$\frac{1}{4} > -\frac{1}{2}$'' respectively.
  On the other hand, $P(\frac{1}{2})$, which is the statement ``$\frac{1}{4}>\frac{1}{2}$'' is false.
\end{exmp}

\guard

\guard

\input{logic/defns/statement.tex}

\begin{defn}
\label{defn:predicate}
  A \emph{predicate} is a sentence which contains a finite number of variables which becomes a statement when a value is assigned to each of its variables.
  The \emph{domain} of a predicate variable is the set of possible values which that variable may take.
\end{defn}


\begin{defn}
\label{defn:truthSet}
\index{truth set}
  Let $P(x)$ be a predicate with variable $x$ with domain $D$.
  The \emph{truth set} of $P(x)$ is the set of elements in $D$ so that $P(x)$ is true after substituting that element for $x$.
  That is, it is the set $\set{ x \in D\mid P(x) }$.
\end{defn}


\begin{exmp}
\label{exmp:truthSetOverReals}
  Let $P(x)$ be as in Example \ref{exmp:predicateOverReals}.
  It is not difficult to see that for any real number $x$ with $x<0$, that ${x^2>0>x}$.
  So, for any real number $x$ with $x<0$ $P(x)$ holds.

  Now, if $x$ is a real number with ${0\leq x\leq 1}$, then we can see that ${0\leq x^2\leq x\leq 1}$.
  So, if $x$ is a real number with ${0\leq x\leq 1}$, then $P(x)$ is false.

  Finally, if $x$ is a real number with ${1 < x}$, then we have that ${x < x^2}$.
  Thus, if $x$ is a real number with $1<x$, then $P(x)$ is true.

  Puting this together, we have that the truth set of $P(x)$ are the real numbers $x$ such that $(x<0)\vee(1<x)$.
  % That is, \[ \set{x\in\RR \mid P(x) } = (\infty,0)\cup(1,\infty)\,.\]
\end{exmp}


\guard

\guard

\guard

\input{logic/defns/statement.tex}

\begin{defn}
\label{defn:predicate}
  A \emph{predicate} is a sentence which contains a finite number of variables which becomes a statement when a value is assigned to each of its variables.
  The \emph{domain} of a predicate variable is the set of possible values which that variable may take.
\end{defn}


\begin{defn}
\label{defn:truthSet}
\index{truth set}
  Let $P(x)$ be a predicate with variable $x$ with domain $D$.
  The \emph{truth set} of $P(x)$ is the set of elements in $D$ so that $P(x)$ is true after substituting that element for $x$.
  That is, it is the set $\set{ x \in D\mid P(x) }$.
\end{defn}


\begin{exmp}
\label{exmp:truthSetOverIntegers}
  Let $Q(n)$ be ``$n$ is a factor of $8$.
  Find the truth set of $Q(n)$ where:
  \begin{enumerate}
    \item the domain of $n$ is the set of positive integers, $\ZZ^+$
      \[ \set{ n\in\ZZ^+ \mid Q(n)} = \set{1,2,4,8}\,.\]
    \item the dimain of $n$ is the set of integers, $\ZZ$.
      \[ \set{ n\in\ZZ \mid Q(n)} = \set{-8,-4,-2,-1,1,2,4,8}\,.\]
  \end{enumerate}
\end{exmp}



\guard

\guard

\guard

\begin{defn}
\label{defn:statement}
  A \emph{statement} (or \emph{proposition}) is a sentence that is either true or false, but not both.
\end{defn}

\guard

\input{logic/defns/statement.tex}

\begin{defn}
\label{defn:predicate}
  A \emph{predicate} is a sentence which contains a finite number of variables which becomes a statement when a value is assigned to each of its variables.
  The \emph{domain} of a predicate variable is the set of possible values which that variable may take.
\end{defn}



\begin{defn}
\label{defn:universalStatement}
\index{universal statement}
	An \emph{universal statement} is a statement asserting that that $P(x)$ holds for every $x$ in $D$, written $\forall x\in D(P(x))$, where  $P(x)$ is a predicate and $D$ is the domain of $x$ in $P(x)$.
	$\forall x\in D(P(x))$ is true if, and only if, $P(x)$ is true for every $x$ in $D$.
\end{defn}


\begin{exmp}
\label{exmp:univeralStatementOverFiniteSet}
  Let $D=\set{ 1,2,3,4,5 }$.
	Show that $\forall x\in D(x \leq x^2)$ is true.

	As $D$ is a finite set, and a small one at that, we can check this by brutal force.
	That is, check that $P(x)$ holds for each element in $D$:
	\begin{itemize}
		\item $x=1$,
			So, $x=1 \leq 1 = x^2$.
		\item $x=2$,
			So, $x=2 \leq 4 = x^2$.
		\item $x=3$,
			So, $x=3 \leq 9 = x^2$.
		\item $x=4$,
			So, $x=4 \leq 16 = x^2$.
		\item $x=5$,
			So, $x=5 \leq 25 = x^2$.
	\end{itemize}
	As we have, exhaustively, shown that $x \geq x^2$ holds for each $x\in D$.
	Thus,  $\forall x\in D(x \geq x^2)$ is true.
\end{exmp}


\index{method of exhaustion}
Note that the technique of checking each element of the domain of an universal statement, as in the example above, is not the way to go.
This method is called the \emph{method of exhaustion}, and is aptly named.

\guard

\guard

\guard

\begin{defn}
\label{defn:statement}
  A \emph{statement} (or \emph{proposition}) is a sentence that is either true or false, but not both.
\end{defn}

\guard

\input{logic/defns/statement.tex}

\begin{defn}
\label{defn:predicate}
  A \emph{predicate} is a sentence which contains a finite number of variables which becomes a statement when a value is assigned to each of its variables.
  The \emph{domain} of a predicate variable is the set of possible values which that variable may take.
\end{defn}



\begin{defn}
\label{defn:universalStatement}
\index{universal statement}
	An \emph{universal statement} is a statement asserting that that $P(x)$ holds for every $x$ in $D$, written $\forall x\in D(P(x))$, where  $P(x)$ is a predicate and $D$ is the domain of $x$ in $P(x)$.
	$\forall x\in D(P(x))$ is true if, and only if, $P(x)$ is true for every $x$ in $D$.
\end{defn}


\begin{exmp}
\label{exmp:univeralStatementOverRealsFalse}
	Show that $\forall x\in \RR(x \leq x^2)$ is false.

	To do this, we need to recall that $\forall x\in \RR(x \leq x^2)$ is true, by Definition \ref{defn:universalStatement}, precisely when $x\leq x^2$ for all $x\in \RR$.
	So, to show it is false we need only show that $x \leq x^2$ is false for some $x\in\RR$.
	That is, we need to provide a counter example to the claim.

	For a counter example, consider $x=\frac{1}{2}$ and note that \[ x=\frac{1}{4}\not\leq \frac{1}{16} = \left(\frac{1}{4}\right)^2\,.\]
\end{exmp}

The technique of producing a counter example is the standard way to show that  an universal statement is false.


\guard

\guard

\guard

\begin{defn}
\label{defn:statement}
  A \emph{statement} (or \emph{proposition}) is a sentence that is either true or false, but not both.
\end{defn}

\guard

\input{logic/defns/statement.tex}

\begin{defn}
\label{defn:predicate}
  A \emph{predicate} is a sentence which contains a finite number of variables which becomes a statement when a value is assigned to each of its variables.
  The \emph{domain} of a predicate variable is the set of possible values which that variable may take.
\end{defn}



\begin{defn}
\label{defn:extistentialStatement}
\index{extistential statement}
	An \emph{extistential statement} is a statement asserting that that $P(x)$ holds for any one $x$ in $D$, written $\exists x\in D(P(x))$, where  $P(x)$ is a predicate and $D$ is the domain of $x$ in $P(x)$.
	$\exists x\in D(P(x))$ is true if, and only if, there exists an $x$ in $D$ such that is true.
\end{defn}


\begin{exmp}
\label{exmp:existentialStatementOverFiniteSetFalse}
  Let $D=\set{ 1,2,3,4,5 }$.
	Show that $\exists x\in D(\frac{1}{x} < \frac{1}{x^2})$ is false.

	As $D$ is a finite set, and a small one at that, we can check this exhaustively.
	That is, check that $P(x)$ is false for each element in $D$:
	\begin{itemize}
		\item $x=1$,
			So, $\frac{1}{x} = 1 \geq 1 = x^2$.
		\item $x=2$,
			So, $\frac{1}{x} = \frac{1}{2} \geq \frac{1}{4} = x^2$.
		\item $x=3$,
			So, $\frac{1}{x} = \frac{1}{3} \geq \frac{1}{9} = x^2$.
		\item $x=4$,
			So, $\frac{1}{x} = \frac{1}{4} \geq \frac{1}{16} = x^2$..
		\item $x=5$,
			So, $\frac{1}{x} = \frac{1}{5} \geq \frac{1}{25} = x^2$..
	\end{itemize}
	As we have, exhaustively, shown that $x \not< x^2$ holds for each $x\in D$.
	That is, that $\frac{1}{x} < \frac{1}{x^2}$ is false for each $x\in D$.
	Thus, $\exists x\in D(\frac{1}{x} < \frac{1}{x^2})$ is false.
\end{exmp}

\guard

\guard

\guard

\begin{defn}
\label{defn:statement}
  A \emph{statement} (or \emph{proposition}) is a sentence that is either true or false, but not both.
\end{defn}

\guard

\input{logic/defns/statement.tex}

\begin{defn}
\label{defn:predicate}
  A \emph{predicate} is a sentence which contains a finite number of variables which becomes a statement when a value is assigned to each of its variables.
  The \emph{domain} of a predicate variable is the set of possible values which that variable may take.
\end{defn}



\begin{defn}
\label{defn:extistentialStatement}
\index{extistential statement}
	An \emph{extistential statement} is a statement asserting that that $P(x)$ holds for any one $x$ in $D$, written $\exists x\in D(P(x))$, where  $P(x)$ is a predicate and $D$ is the domain of $x$ in $P(x)$.
	$\exists x\in D(P(x))$ is true if, and only if, there exists an $x$ in $D$ such that is true.
\end{defn}


\begin{exmp}
\label{exmp:existentialStatementOverReals}
\index{witness}
	Show that $\exists x\in \RR(x = x^2)$ is true.

	To do this, we need to recall that $\exists x\in \RR(x = x^2)$ is true, by Definition \ref{defn:extistentialStatement}, precisely when $x= x^2$ for some $x\in \RR$.
	So, we must find a \emph{witness} which sees that $x=x^2$ is true.

	For such a witness, consider $x=1$ and note that$x=1=1=x^2$.
\end{exmp}


\guard

\guard

\guard

\begin{defn}
\label{defn:statement}
  A \emph{statement} (or \emph{proposition}) is a sentence that is either true or false, but not both.
\end{defn}

\guard

\input{logic/defns/statement.tex}

\begin{defn}
\label{defn:predicate}
  A \emph{predicate} is a sentence which contains a finite number of variables which becomes a statement when a value is assigned to each of its variables.
  The \emph{domain} of a predicate variable is the set of possible values which that variable may take.
\end{defn}



\begin{defn}
\label{defn:universalStatement}
\index{universal statement}
	An \emph{universal statement} is a statement asserting that that $P(x)$ holds for every $x$ in $D$, written $\forall x\in D(P(x))$, where  $P(x)$ is a predicate and $D$ is the domain of $x$ in $P(x)$.
	$\forall x\in D(P(x))$ is true if, and only if, $P(x)$ is true for every $x$ in $D$.
\end{defn}

\guard

\guard

\begin{defn}
\label{defn:statement}
  A \emph{statement} (or \emph{proposition}) is a sentence that is either true or false, but not both.
\end{defn}

\guard

\input{logic/defns/statement.tex}

\begin{defn}
\label{defn:predicate}
  A \emph{predicate} is a sentence which contains a finite number of variables which becomes a statement when a value is assigned to each of its variables.
  The \emph{domain} of a predicate variable is the set of possible values which that variable may take.
\end{defn}



\begin{defn}
\label{defn:extistentialStatement}
\index{extistential statement}
	An \emph{extistential statement} is a statement asserting that that $P(x)$ holds for any one $x$ in $D$, written $\exists x\in D(P(x))$, where  $P(x)$ is a predicate and $D$ is the domain of $x$ in $P(x)$.
	$\exists x\in D(P(x))$ is true if, and only if, there exists an $x$ in $D$ such that is true.
\end{defn}


\begin{exmp}
\label{exmp:translateQuantifiedStatements}
  Translate the following statement to a formal statement
	\begin{center}
		``$2$ times any integer is even.''
	\end{center}
	\[ \forall n\in\ZZ( 2n \text{ is even})\]
\end{exmp}


\guard

% This is redundant.
\guard

\guard

\input{logic/defns/statement.tex}
\input{logic/defns/statementForm.tex}
\input{logic/defns/truthValue.tex}


\begin{defn}
\label{defn:conditionalStatement}
\index{conditional}
  Let $p$ and $q$ be statements forms.
  The \emph{conditional statement} ``$p$ implies $q$'', written $p \rightarrow q$, is the statement form that is false precisely when $p$ is true and $q$ is false ( that is, when the statement ``If $p$, then $q$'' is violated ).

  In the conditional $p\rightarrow q$, $p$ is refered to as the \emph{hypothesis} and $q$ is called the \emph{conclusion}.
\end{defn}

\guard

\input{logic/defns/statement.tex}
\input{logic/defns/predicate.tex}


\begin{defn}
\label{defn:universalStatement}
\index{universal statement}
	An \emph{universal statement} is a statement asserting that that $P(x)$ holds for every $x$ in $D$, written $\forall x\in D(P(x))$, where  $P(x)$ is a predicate and $D$ is the domain of $x$ in $P(x)$.
	$\forall x\in D(P(x))$ is true if, and only if, $P(x)$ is true for every $x$ in $D$.
\end{defn}


\begin{defn}
\label{defn:universalConditionalStatement}
\index{universal conditional statement}
	An \emph{universal conditional statement} is a statement asserting that that for any $x\in D$ such that $P(x)$, $Q(x)$ holds as well.
	Written $\forall x\in D(P(x)\rightarrow Q(x) )$, where  $P(x)$ and $Q(x)$ are predicates and $D$ is the domain of $x$ in $P(x)$ and $Q(x)$.
	$\forall x\in D(P(x)\rightarrow Q(x))$ is true if, and only if, $P(x)\rightarrow Q(x)$ is true for every $x$ in $D$.
\end{defn}


\begin{exmp}
\label{exmp:translateQuantifiedConditionalStatements}
  Rewrite the following informally \[ \forall x\in\RR( x>2 \rightarrow x^2 > 4 )\,.\]
	\begin{center}
		``If $x$ is a real number greater than $2$, then the square of $x$ is greater than $4$.''
	\end{center}
\end{exmp}


Note that the statements
\begin{center}
	``For all real numbers $x$ if $x$ is an integer, then $x$ is a rational.''
\end{center}
and
\begin{center}
	``For all integers $x$, $x$ is a rational.''
\end{center}
are equivalent.

In fact, \[\forall x\in D(P(x)\rightarrow Q(x) )\equiv \forall x\in \set{y\in D\mid P(x)}( Q(x) )\]


Let $P(x)$ and $Q(x)$ be predicates with $D$ the common domain of $x$.
We write $P(x)\implies Q(x)$ to mean $\forall x\in D( P(x)\rightarrow Q(x))$ when there is no risk for confusion of the domain $D$.
Similarly, we write $P(x)\iff Q(x)$ to mean $\forall x\in D( P(x)\leftrightarrow Q(x))$
