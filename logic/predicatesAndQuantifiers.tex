\guard

\section{Predicates and Quantifiers}
\label{sec:predicatesAndQuantifiers}

Note the stentence
  \begin{center}
    ``They are a college student.''
  \end{center}
  is not a statement as it may be true or false depending on the value of ``They''.
In this sentence ``They'' is a free variable.
Similarly, ``$x+y>0$'' is not a statement, it has more familiar variables $x$ and $y$.

The predicate refers to the part of the sentence withsome or all of its nouns removed.
That is, in the sentence
  \begin{center}
    ``James is a student at the University of North Texas.'',
  \end{center}
  ``James'' is the subject and
  \begin{center}
    ``is a student at the University of North Texas.''
  \end{center}
  is the predicate.

In logic, predicates are formed in much the same way.
\guard

\guard

\guard

\begin{defn}
\label{defn:statement}
  A \emph{statement} (or \emph{proposition}) is a sentence that is either true or false, but not both.
\end{defn}


\begin{defn}
\label{defn:predicate}
  A \emph{predicate} is a sentence which contains a finite number of variables which becomes a statement when a value is assigned to each of its variables.
  The \emph{domain} of a predicate variable is the set of possible values which that variable may take.
\end{defn}


\begin{exmp}
\label{exmp:predicates}
  Let $P(x)$ denote
    \begin{center}
      ``$x$ is a student at the University of North Texas.''
    \end{center}
    and $Q(x,y)$ denote
    \begin{center}
      ``$x$ is a student at $y$.''.
    \end{center}
  $x$ is the predicate variable for $P(x)$ while both $x$ and $y$ are predicate variables for $Q(x,y)$.

  Note that when values are substituted for $x$ and $y$ in $P(x)$ and $Q(x,y)$, these sentences become statements.
  For instance if we substitute $x$ for ``Taylor'' and $y$ for``Boise State University'', $P(x)$ and $Q(x,y)$ become the statements
    \begin{center}
      ``Taylor is a student at the University of North Texas.''
    \end{center}
    and
    \begin{center}
      ``Taylor is a student at Boise State University.''
    \end{center}
    respectively.
  Whence, $P(x)$ and $Q(x,y)$ are predicates
\end{exmp}

In Example \ref{exmp:predicates}, we referred to the example's predicates with their full names, $P(x)$ and $Q(x,y)$.
When there is no room for confusion and the variables or other ornaments are of little importance to the statement being made, we may drop those decorations from the symbol.
That is, the concluding sentence of Example \ref{exmp:predicates} could be written ``Whence, $P$ and $Q$ are predicates.''.


\guard

\guard

\guard

\input{logic/defns/statement.tex}

\begin{defn}
\label{defn:predicate}
  A \emph{predicate} is a sentence which contains a finite number of variables which becomes a statement when a value is assigned to each of its variables.
  The \emph{domain} of a predicate variable is the set of possible values which that variable may take.
\end{defn}


\begin{exmp}
\label{exmp:predicateOverReals}
  Let $P(x)$ denote ``$x^2 > x$'' with $\RR$, the set of real numbers, as the domain of $x$.

  $P(2)$ and $P(-\frac{1}{2})$ are true.
  This is because $P(2)$ and $P(-\frac{1}{2})$ denote the statement ``$4 > 2$'' and ``$\frac{1}{4} > -\frac{1}{2}$'' respectively.
  On the other hand, $P(\frac{1}{2})$, which is the statement ``$\frac{1}{4}>\frac{1}{2}$'' is false.
\end{exmp}

\guard

\guard

\input{logic/defns/statement.tex}

\begin{defn}
\label{defn:predicate}
  A \emph{predicate} is a sentence which contains a finite number of variables which becomes a statement when a value is assigned to each of its variables.
  The \emph{domain} of a predicate variable is the set of possible values which that variable may take.
\end{defn}


\begin{defn}
\label{defn:truthSet}
\index{truth set}
  Let $P(x)$ be a predicate with variable $x$ with domain $D$.
  The \emph{truth set} of $P(x)$ is the set of elements in $D$ so that $P(x)$ is true after substituting that element for $x$.
  That is, it is the set $\set{ x \in D\mid P(x) }$.
\end{defn}


\begin{exmp}
\label{exmp:truthSetOverReals}
  Let $P(x)$ be as in Example \ref{exmp:predicateOverReals}.
  It is not difficult to see that for any real number $x$ with $x<0$, that ${x^2>0>x}$.
  So, for any real number $x$ with $x<0$ $P(x)$ holds.

  Now, if $x$ is a real number with ${0\leq x\leq 1}$, then we can see that ${0\leq x^2\leq x\leq 1}$.
  So, if $x$ is a real number with ${0\leq x\leq 1}$, then $P(x)$ is false.

  Finally, if $x$ is a real number with ${1 < x}$, then we have that ${x < x^2}$.
  Thus, if $x$ is a real number with $1<x$, then $P(x)$ is true.

  Puting this together, we have that the truth set of $P(x)$ are the real numbers $x$ such that $(x<0)\vee(1<x)$.
  % That is, \[ \set{x\in\RR \mid P(x) } = (\infty,0)\cup(1,\infty)\,.\]
\end{exmp}


\guard

\guard

\guard

\input{logic/defns/statement.tex}

\begin{defn}
\label{defn:predicate}
  A \emph{predicate} is a sentence which contains a finite number of variables which becomes a statement when a value is assigned to each of its variables.
  The \emph{domain} of a predicate variable is the set of possible values which that variable may take.
\end{defn}


\begin{defn}
\label{defn:truthSet}
\index{truth set}
  Let $P(x)$ be a predicate with variable $x$ with domain $D$.
  The \emph{truth set} of $P(x)$ is the set of elements in $D$ so that $P(x)$ is true after substituting that element for $x$.
  That is, it is the set $\set{ x \in D\mid P(x) }$.
\end{defn}


\begin{exmp}
\label{exmp:truthSetOverIntegers}
  Let $Q(n)$ be ``$n$ is a factor of $8$.
  Find the truth set of $Q(n)$ where:
  \begin{enumerate}
    \item the domain of $n$ is the set of positive integers, $\ZZ^+$
      \[ \set{ n\in\ZZ^+ \mid Q(n)} = \set{1,2,4,8}\,.\]
    \item the dimain of $n$ is the set of integers, $\ZZ$.
      \[ \set{ n\in\ZZ \mid Q(n)} = \set{-8,-4,-2,-1,1,2,4,8}\,.\]
  \end{enumerate}
\end{exmp}

