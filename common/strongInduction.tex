\guard
\section{Strong Induction}
\label{sec:strongInduction}

\guard

\guard

\begin{defn}
\label{defn:statement}
  A \emph{statement} (or \emph{proposition}) is a sentence that is either true or false, but not both.
\end{defn}


\begin{defn}
\label{defn:predicate}
  A \emph{predicate} is a sentence which contains a finite number of variables which becomes a statement when a value is assigned to each of its variables.
  The \emph{domain} of a predicate variable is the set of possible values which that variable may take.
\end{defn}


The principle of strong induction is:
\index{weak induction}
Let $P(n)$ some predicate where the domain of $n$ is the integers, or some subset of the integers.
Let $a,b\in\ZZ$ be some fixed values with $a<b$, and suppose that for all $c\geq a$, $c$ is in the domain of $n$ in $P(n)$.
Then, if
\begin{enumerate}
  \item $P(a),P(a+1),\dots,P(b)$ are true and
  \item $\forall k\geq b$ if $\forall i\in\ZZ$ with $a\leq i\leq k$ $P(i)$ holds, then $P(k+1)$ is true,
\end{enumerate}
then $\forall k\geq a(P(k))$ is true.
This is known as \emph{strong induction}.
You'll notice that both the base and inductive steps of strong induction differ slightly from weak induction covered in Section \ref{sec:induction}.

Strong induction allows for proving statements which require checking some finite number of values in the base step.
One case where you may want this power is when proving that some recurrence relation which is defined using more than one previous value.
Further, and more importantly, strong induction differs from its weaker counterpart in that the inductive step makes the assumption that the desired claim holds for all values up to and included some value before moving to the next value.

\guard

\guard

\guard

%TODO: define cross product for sets
%TODO: define axiom of comprehension

\begin{defn}
\label{defn:function}
\index{function}
  Let $A$ and $B$ be sets.
  We say that $f\subseteq A\times B :=\set{(a,b)\mid a\in A,~b\in B}$ is a \emph{function} from $A$ to $B$, written $f:A\to B$ if \[ \forall a\in A\exists!b\in B( (a,b)\in f)\,.\]
  We write $f(a)=b$ to denote $(a,b)\in f$.
  We say that $A$ is the \emph{domain} of $f$, written $\dom(f)$.
  We say that $\set{b\in B\mid \exists a\in A(f(a)=b)}$ is the \emph{range} of $f$, written $\ran(f)$.
\end{defn}


\begin{defn}
\label{defn:sequence}
\index{sequence}
\index{term}
\index{index}
  A \emph{sequence} is a function whose domain is some subset of $\NN$.
  For a sequence $s$, we typically write $s_i$ in place of $s(i)$.
  We refer to an individual value, $s_i$, as a \emph{term} in the sequence and $i$ as the \emph{index} of that term
\end{defn}

\index{explicit formula for sequence}
For a sequence $s$, an \emph{explicit formula} of $s$ is a rule or formula which depends on the index of the sequence.


\begin{exmp}
\label{exmp:recurrenceRelation}
  Define the sequence $s$ as \[ s_0=0,~s_1=4,s_k=6s_{k-1}-5s_{k-2}\,.\]
  We want to show that $\forall n\in\NN$ $s_n = 5^n-1$.

  We do so by strong induction on $n\in\NN$.

  Base step, we check that the claim holds for $n=0$ and $n=1$.
  Here, we note that $s_0 = 0 = 5^0-1$ and $s_1 = 4 = 5^1-1$.
  So, the claim holds at the first two values for $n$.

  Now, we check the inductive step.
  Let $n\in\NN$ and suppose that for all $k\in\NN$ with $k\leq n$ $s_k = 5^k-1$.
  Now, by definition $s_{n+1} = 6 s_n - 5 s_{n-1}$.
  Though, out inductive hypothesis provides that $s_n = 5^n-1 $ and $s_{n-1}=5^{n-1}-1$.
  Thus,
  \begin{align*}
    s_{n+1} &= 6 s_n - 5 s_{n-1} \\
            &= 6 ( 5^n - 1 ) - 5 ( 5^{n-1} - 1 ) \\
            &= 6\cdot 5^n - 6 - 5^n - 5 \\
            &= 5\cdot 5^n - 1 \\
            &= 5^{n+1} - 1\,.
  \end{align*}
  showing that the claim holds at $n+1$.

  Thus, by induction, we have that $\forall n\in\NN$ $s_n = 5^n-1$.
\end{exmp}


\guard

\guard

\guard

\begin{defn}
\label{defn:prime}
\index{prime}
  An integer $1<p$ is said to be \emph{prime} provided for all positive integers $a$ and $b$ if $ab=p$, then $a=p$ or $b=p$.
\end{defn}

\guard

\input{numberTheory/defns/prime.tex}

\begin{defn}
\label{defn:composite}
\index{prime}
  An integer $1<c$ is said to be \emph{composite} if $c$ there exists integers $a$ and $b$ such that $c=ab$ and $1<a,b<c$
\end{defn}


\begin{lem}
\label{lem:notPrimeIsComposite}
  For every integer $n>1$, $n$ is prime if, and only if, $n$ is not composite.
\end{lem}
\begin{proof}
  Fix any integer $n>1$.
  Suppose $n$ is prime.
  By the definition of $n$ prime, the only factors of $n$ are $1$ and $n$.
  Thus, $\not\exists r,s\in\ZZ(n=rs\wedge 1<r,s<n)$.
  Showing that $n$ is not composite.

  Suppose that $n$ is not prime.
  As $n$ is not prime, there exists $r,s\in\ZZ$ such that $n=rs$ and neither $r$ nor $s$ are $1$ or $n$.
  Note, as $n$ is positive, $r$ and $s$ share the same sign.
  With out loss of generality, we may assume $r$ and $s$ are positive as otherwise we may replace them with $-r$ and $-s$.
  Finally, as $rs=n$ and multiplication of positive integers is non-decreasing, $r,s\leq n$.
  As neither $r$ nor $s$ is $1$ and $r,s$ is positive we have that $1<r,s$ and $r,s\leq n$.
  Thus, by Definition \ref{defn:composite}, $n$ is composite.
\end{proof}

\guard

\guard

%TODO define integer
%TODO define integer multiplication

\begin{defn}
\label{defn:divide}
\index{divide}
  An integer $d\not=0$ is said to \emph{divide} an integer $n$ provided, written $d\vert n$, provided there exists an integer $k$ such that $n=dk$.
\end{defn}


\begin{prop}
\label{prop:dividesIsTransitive}
  Let $a,b,c\in\ZZ$.
  If $a\vert b$ and $b\vert c$, then $a\vert c$.
\end{prop}
\begin{proof}
  Fix $a,b,c\in\ZZ$ such that $a\vert b$ and $b\vert c$.
  As $a\vert b$ and $b\vert c$ there exists $r,s\in\ZZ$ such that \[b=ra\text{  and  }c=sb\,.\]
  So,
  \begin{align*}
    c &= sb \\
      &= s(ra) \\
      &= (sr)a \,.
  \end{align*}
  And as $s,r\in\ZZ$ $sr\in\ZZ$.
  Therefore, $a\vert c$.
\end{proof}


\begin{prop}
\label{prop:naturalsLargerThanOneDivisibleByPrime}
  Each integers larger than $1$ is divisible by a prime.
\end{prop}
\begin{proof}
  We prove this by strong induction on $n\in\NN$ with $n>1$.

  Consider the case where $n=2$.
  Then $n$ is prime and divides itself.
  Thus, the claim holds.

  Now, suppose that $n\in\NN$ with $n>1$ is such that every $k$ with $1<k\leq n$, $k$ is divide by a prime.
  Certainly, $n+1$ is either prime or is not prime.

  If $n+1$ is prime, then as before, as $n+1$ divides itself the claim holds.

  Otherwise, by Lemma \ref{lem:notPrimeIsComposite} $n+1$ is composite.
  As $n+1$ is composite, there exists $a, b$ so that $1<a,b<n+1$ and $ab = n+1$.
  Though, as $1<a<n+1$ $1<a \leq n$, and our inductive hypothesis provides that $a$ is divisible by a prime.
  Though, as $a$ divides $n+1$ and $a$ is divisible by a prime, proposition \ref{prop:dividesIsTransitive}, $n+1$ is divisible by a prime.
\end{proof}

