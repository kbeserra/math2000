\guard
\section{Induction}
\label{sec:induction}

\guard

\guard

\begin{defn}
\label{defn:statement}
  A \emph{statement} (or \emph{proposition}) is a sentence which is either true or false, but not both.
\end{defn}


\begin{defn}
\label{defn:predicate}
  A \emph{predicate} is a sentence which contains a finite number of variables which becomes a statement when a value is assigned to each of its variables.
  The \emph{domain} of a predicate variable is the set of possible values which that variable may take.
\end{defn}


The principle of mathematical induction is:
\index{weak induction}
Let $P(n)$ some predicate where the domain of $n$ is the integers, or some subset of the integers.
Let $a\in\ZZ$ be some fixed value, and suppose that for all $c\geq a$, $c$ is in the domain of $n$ in $P(n)$.
Then, if
\begin{enumerate}
  \item $P(a)$ is true and
  \item $\forall k\geq a$ if $P(k)$ is true, then $P(k+1)$ is true,
\end{enumerate}
then $\forall k\geq a(P(k))$ is true.
This is known as \emph{weak induction}.
Proofs by induction generally follow a common pattern.
\index{base step}
\index{inductive step}
\index{inductive hypothesis}
First, the proof shows (1) holds, this is known as the \emph{base step}.
Second, the proof shows (2), which is called the \emph{inductive step}.
During the inductive step, the assumption that $P(k)$ holds for some fixed $k\geq a$ is called the \emph{inductive hypothesis}.

Put simply, if $P(n)$ is some predicate with the variable $n$ an integer, then if we can show that $P(a)$ holds for some starting value and that $P$ holding at some value implies that $P$ holds at the next value, then $P$ holds at every value greater that or equal to the base value.

You can think of this as a line of dominos standing upright so that if you knock down one domino, then that domino knocks down its neighbor.
Mathematical induction is saying that, in this scenario, if you knock down the first domino, then all of the dominos will be knocked down!

We will see many examples of induction in the future.

\guard

\begin{prop}
\label{prop:3And5CentCoins}
  For all integers $n\geq 8$, $n\cent$ can be made with only $3\cent$ and $5\cent$ coins.
\end{prop}
\begin{proof}
  We'll prove this by induction on integers $n\geq 8$.

  Base step. $8\cent$ can be formed using a single $3\cent$ coin and a single $5\cent$ coin.
  Thus, $8\cent$ can be formed using only $3\cent$ and $5\cent$ coins.

  Inductive step. Suppose $n\geq 8$ and $n\cent$ can be formed using only $3\cent$ and $5\cent$ coins.
  Let $t$ be the number of $3\cent$ coins used to form $n\cent$.
  Similarly, let $f$ be the number of $5\cent$ coins used to form $n\cent$.
  Then, out inductive hypothesis gives that $n\cent = t\cdot3\cent + f\cdot5\cent$.
  We'll break into cases depending on the value of $f$.
  Either $f>0$ or $f=0$.

  In the case that $f>0$, then $f-1\geq 0$ and
  \begin{align*}
    (n+1)\cent  &= n\cent + 1\cent \\
                &= (t\cdot3 + f\cdot5 + 1)\cent \\
                &= (t\cdot3 + f\cdot5 + 2\cdot3 - 5)\cent\\
                &= (t+2)3\cent + (f-1)5\cent\,.
  \end{align*}
  Thus, we can form $(n+1)\cent$ using just $3\cent$ and $5\cent$ coins.

  in the case that $f=0$, we have that $n\cent = t\cdot3\cent$.
  As $n\geq 8$, we have that $t\geq 3$.
  So, $t-3\geq 0$.
  Now, similar to before, we have
  \begin{align*}
    (n+1)\cent  &= n\cent + 1\cent \\
                &= (t\cdot3 + 1)\cent \\
                &= (t\cdot3 + 2\cdot5 - 3\cdot3)\cent\\
                &= (t-3)3\cent + 2\cdot5\cent\,.
  \end{align*}
  Thus, we can form $(n+1)\cent$ using just $3\cent$ and $5\cent$ coins.

  So, in either case we can form $(n+1)\cent$ using only $3\cent$ and $5\cent$ coins.
\end{proof}

\guard

\input{common/defns/sumOfSequence.tex}

\begin{prop}
\label{prop:sumOfConsequtiveIntegers}
  \[\forall n\in\NN\quad\sum_{i=0}^n i = \frac{n(n+1)}{2}\,.\]
\end{prop}
\begin{proof}
  We'll prove this by induction on integers $n\in\NN$.

  Base step.
  By Definition \ref{defn:sumOfSequence} and some simple arithmatic, we have that
    \begin{align*}
      \sum_{i=0}^0 i  &= 0 \\
                      &= \frac{ 0(0+1)}{2}\,,
    \end{align*}
  and so the claim holds at $n=0$.

  Inductive step. Let $n\in\NN$ and suppose that $\sum_{i=0}^n i = \frac{n(n+1)}{2}$.
  Then, by Definition \ref{defn:sumOfSequence},
    \[ \sum_{i=0}^{n+1} i = \left( \sum_{i=0}^{n} i \right) + (n+1)\,. \]
  Now, using our inductive hypothesis we have
    \begin{align*}
      \sum_{i=0}^{n+1} i &= \left( \sum_{i=0}^{n} i \right) + (n+1) \\
                         &= \frac{ n(n+1)}{2} + (n+1) \\
                         &= \frac{ n(n+1)}{2} + \frac{2(n+1)}{2} \\
                         &= \frac{ n(n+1)+2(n+1)}{2} \\
                         &= \frac{ (n+1)(n+2) }{2} \,.
    \end{align*}
\end{proof}


\guard

\guard

\input{common/defns/sumOfSequence.tex}

\begin{prop}
\label{prop:sumOfSquares}
  \[\forall n\in\NN\quad\sum_{i=0}^n i^2 = \frac{n(n+1)(2n+1)}{6}\,.\]
\end{prop}
\begin{proof}
  Left as an exercise to the reader.
\end{proof}


\begin{exercise}
\label{exercise:sumOfSquares}
  Prove Proposition \ref{prop:sumOfSquares}.
\end{exercise}

\guard

\guard

\input{common/defns/sumOfSequence.tex}

\begin{prop}
\label{prop:sumOfGeometricSequence}
  For all $r\in\RR$ with $r\not=1$ and for all $n\in\NN$
  \[ n\in\NN\sum_{i=0}^n r^n = \frac{r^{n+1}-1}{r-1} \,.\]
\end{prop}
\begin{proof}
  Left as an exercise to the reader.
\end{proof}


\begin{exercise}
\label{exercise:sumOfGeometricSequence}
  Prove Proposition \ref{prop:sumOfGeometricSequence}.
\end{exercise}



\guard

\guard

%TODO define integer
%TODO define integer multiplication

\begin{defn}
\label{defn:divide}
\index{divide}
  An integer $d\not=0$ is said to \emph{divide} an integer $n$ provided, written $d|n$, provided there exists an integer $k$ such that $n=dk$.
  We will write $a\nmid b$ to mean $\neg(a|b)$.
\end{defn}


\begin{prop}
\label{prop:evenPowersOf2Minus1DivisibleBy3}
  For all $n\in\NN$ $2^{2n}-1$ is divisible by $3$.
\end{prop}
\begin{proof}
  We prove this by induction on $n\in\NN$.

  If $n=0$, then
  \begin{align*}
    2^{2n}-1  &= 2^{2\cdot0}-1 \\
              &= 1 - 1
              &= 0\,.
  \end{align*}
  Though, as $0 = 3\cdot 0$, we have that $2^{2\cdot 0}-1$ is divisible by $3$.

  Now, suppose that $n\in\NN$ is such that $2^{2n}-1$ is divisible by $3$.
  As $2^{2n}-1$ is divisible by $3$, there exists some $a\in\ZZ$ so that $3a = 2^{2n}-1$.
  Now,
  \begin{align*}
    2^{2(n+1)}-1  &= 2^{2n+2} - 1 \\
                  &= 2^{2n}4 - 1 \\
                  &= 2^{2n}(3+1) - 1 \\
                  &= 3\cdot 2^{2n} + 2^{2n} - 1 \\
                  &= 3\cdot 2^{2n} + 3a \\
                  &= 3( 2^{2n} + a )\,.
  \end{align*}
  So, as $2^{2n}+a\in\ZZ$ $3|2^{2(n+1)}-1$.
\end{proof}

\guard

\begin{prop}
\label{prop:nthOddLessThan2ToTheN}
  For all $n\in\NN$ if $n\geq 2$, then $2n+1 < 2^n$.
\end{prop}
\begin{proof}
  We prove this by induction on $n\in\NN$ with $n>2$.

  If $n=3$, then
  \begin{align*}
    2n+1  &= 2\cdot 3 + 1 \\
          &= 7 \\
          &< 8 \\
          &= 2^3 \\
          &= 2^n
  \end{align*}

  Now, suppose that $n\in\NN$ with $n>3$ is such that $2n+1 < 2^n$.
  Then,
  \begin{align*}
    2(n+1)+1  &= 2n + 3 \\
              &= 2n + 1 + 2 \\
              &< 2^n + 2 \\
              &< 2^n + 2^n \\
              &= 2^{n+1}\,.
  \end{align*}
\end{proof}


\guard

\guard

\guard

%TODO: define cross product for sets
%TODO: define axiom of comprehension

\begin{defn}
\label{defn:function}
\index{function}
  Let $A$ and $B$ be sets.
  We say that $f\subseteq A\times B :=\set{(a,b)\mid a\in A,~b\in B}$ is a \emph{function} from $A$ to $B$, written $f:A\to B$ if \[ \forall a\in A\exists!b\in B( (a,b)\in f)\,.\]
  We write $f(a)=b$ to denote $(a,b)\in f$.
  We say that $A$ is the \emph{domain} of $f$, written $\dom(f)$.
  We say that $\set{b\in B\mid \exists a\in A(f(a)=b)}$ is the \emph{range} of $f$, written $\ran(f)$.
\end{defn}


\begin{defn}
\label{defn:sequence}
\index{sequence}
\index{term}
\index{index}
  A \emph{sequence} is a function whose domain is some subset of $\NN$.
  For a sequence $s$, we typically write $s_i$ in place of $s(i)$.
  We refer to an individual value, $s_i$, as a \emph{term} in the sequence and $i$ as the \emph{index} of that term
\end{defn}

\index{explicit formula for sequence}
For a sequence $s$, an \emph{explicit formula} of $s$ is a rule or formula which depends on the index of the sequence.


\begin{exmp}
\label{exmp:showClosedFormFormulaWorks}
  Define the sequence $a$ by $a_0=2$ and $a_{k+1} = 5a_k$.
  \begin{enumerate}
    \item Write the first four terms of the sequence $a$.\\
    \[ a_0 = 2,\,a_1=10,\,a_2=50,\,a_3=250\,.\]
    \item Use induction to show that $a_k=2\cdot 5^k$ for $k\in\NN$.\\
    We first note, that when $k=0$, we have that
    \begin{align*}
      2\cdot 5^0  &= 2 \cdot 1 \\
                  &= 2 \\
                  &= a_0\,.
    \end{align*}

    Now, suppose $k\in\NN$ and $a_k = 2\cdot 5^k$.
    Then, by definition we have,
    \begin{align*}
      a_{k+1} &= 5a_k \\
              &= 5\cdot 2 \cdot 5^k \\
              &= 2 \cdot 5^{k+1}\,.
    \end{align*}
    Thus, by induction, we have that $\forall k\in\NN\,a_k=2\cdot 5^k$.
  \end{enumerate}
\end{exmp}

