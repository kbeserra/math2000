\guard
\section{Suppose Otherwise}
\label{sec:supposeOtherwise}

\guard

\guard

\guard

\begin{defn}
\label{defn:statement}
  A \emph{statement} (or \emph{proposition}) is a sentence that is either true or false, but not both.
\end{defn}


% \guard

\input{logic/defns/statement.tex}
\input{logic/defns/truthValue.tex}


\begin{defn}
\label{defn:negationOfStatement}
  Let $p$ be a statement.
  The \emph{negation} of $p$, written $\neg p$, is the statement with the opposite truth value.
\end{defn}

Note, the symbol $\neg$ is not the only symbol used to denote negation.
For instance, it is not uncommon to see the symbol $\sim$ used in other logic texts.
Further, many c-like programming-languages will use the symbol $!$ for the same meaning.
The Python programming-language deserves a special call-out on this note, it allows for the use of the symbol `not' for its not symbol (which fits nicely with its use of `and' and `or' for its and \& or logical connectives).

% \guard

\input{logic/defns/statement.tex}
\input{logic/defns/truthValue.tex}


\begin{defn}
\label{defn:disjunctionOfStatement}
  Let $p$ and $q$ be statements forms.
  The \emph{disjunction} of $p$ and $q$, written $p \vee q$, is the statement form that is true when either $p$ or $q$ is true and false precisely when $p$ and $q$ are both false.
\end{defn}

% \guard

\input{logic/defns/statement.tex}
\input{logic/defns/truthValue.tex}


\begin{defn}
\label{defn:conjunctionOfStatement}
  Let $p$ and $q$ be statements.
  The \emph{conjunction} of $p$ and $q$, written $p \wedge q$, is the statement that is true precisely when both $p$ and $q$ are true and is otherwise false.
\end{defn}


\begin{defn}
\label{defn:statementForm}
\index{statement form}
  A \emph{statement form} (or \emph{proposition form}) is an expression made up of statement variables and logical connections (such as $\neg$, $\vee$, or $\wedge$) which when substituting statements for statement variables becomes a statement.
\end{defn}


\begin{defn}
\label{defn:tautology}
  A \emph{tautology} is a statement form that is true independent of the truth value assignments of its truth value assignments.
  A statement whose statement form is a tautology is a \emph{tautological statement}.
\end{defn}


The central idea of proving a statement by contradiction is that mathematics must be non-contradictory.
That is, for any statement $\phi$ either $\phi$ is true or $\phi$ is false but not both!
More formally, the statement form $\neg(p\wedge\neg p)$ is a tautology.

It should be noted that proofs by contrition, while valid, are often undesirable.
This is because proofs by contradiction often do not reveal the same insight offered by constructive or direct proofs.
Further, proofs by contradiction can often, if done without care, feel inelegant, much like a sledgehammer.

Suppose we are attempting to prove a statement $\phi$ via contradiction.
The common structure of a proof by contradiction is as follows.
\begin{proof}
  Suppose, towards a contradiction, $\neg\phi$.\\
  Unpack what $\neg\phi$ means.\\
  Perform some work and obtain a formula $\neg\psi$ where we know $\psi$ to be true before we supposed $\neg\phi$.\\
  This gives $\psi$ and $\neg\psi$, a contradiction.\\
  Whence, $\phi$.
\end{proof}
Now, the above is a simple sketch of a near ideal argument.
In practice, this argument can take many forms.
In this section, we will show some examples of this technique to help the reader become more familiar with the process.

\guard

\begin{prop}
\label{prop:integersAreUnbounded}
  There is no greatest integer.
\end{prop}
\begin{proof}
  Suppose, towards a contradiction, that there is a greatest integer.
  That is, there exists some $n\in\ZZ$ such that $\forall m\in\ZZ$ $m\leq n$.
  Consider $n+1$.
  As $\ZZ$ is closed under successors, the operation $\cdot\mapsto\cdot+1$, $n+1\in\ZZ$.
  Though, $n\lneq n+1$ contradicting our assumption of $n$.
  Thus, there is not greatest integer.
\end{proof}

\guard

\guard

%TODO define integer
%TODO define integer multiplication
%TODO define 2

\begin{defn}
\label{defn:even}
\index{even}
  An integer $n$ is said to be \emph{even} if there exists an integer $k$ such that \[ n = 2 k \,.\]
\end{defn}

\guard

\begin{defn}
\label{defn:odd}
\index{odd}
  An integer $n$ is said to be \emph{odd} if there exists an integer $k$ such that \[ n = 2 k + 1\,.\]
\end{defn}


\begin{prop}
\label{prop:noIntegerBothEvenAndOdd}
  No integer is both even and odd.
\end{prop}
\begin{proof}
  Suppose, towards a contradiction, that there is an integer which is both even and odd.
  Let $n\in\ZZ$ be such an integer.
  As $n$ is both even and odd, there exists $a,b\in\ZZ$ such that \[ 2a=n=2b+1\,.\]
  Though,
  \begin{alignat*}{2}
    && n &=n \\
    &\implies\quad & 2a &= 2b + 1 \\
    &\implies\quad & 2a-2b &= 1 \\
    &\implies\quad & a-b &= \frac{1}{2}\\
    &&&\not\in\ZZ
  \end{alignat*}

  Though, as $a,b\in\ZZ$ $a-b\in\ZZ$, a contradiction.
  Whence, there is no integer which is both even and odd.
\end{proof}

\guard

\guard

%TODO define the integers
%TODO define integer multiplication.

\begin{defn}
\label{defn:rational}
\index{rational}
  A real number $r$ is said to be \emph{rational} if, and only if, $\exists a,b\in\ZZ$ with $b\not=0$ and $rb= a$.
\end{defn}

We denote the set of rational numbers $\QQ$.

\guard

\guard

%TODO define the integers
%TODO define integer multiplication.

\begin{defn}
\label{defn:rational}
\index{rational}
  A real number $r$ is said to be \emph{rational} if, and only if, $\exists a,b\in\ZZ$ with $b\not=0$ and $rb= a$.
\end{defn}

We denote the set of rational numbers $\QQ$.


%TODO show the integers are closed under addition and multiplication.
%TODO show multiplication over the integers is commutative.

\begin{prop}
\label{prop:productOfRaionalsIsRational}
  If $r$ and $s$ are rational, then $rs$ is rational.
\end{prop}
\begin{proof}
  Fix $r$ and $s$ rational.
  This means, by Definition \ref{defn:rational}, there exists $a,b,c,d\in\ZZ$ with $b,d\not=0$ such that $sb=a$ and $rd=c$.
  Note that as $a,b,c,d\in\ZZ$, $ac,bd\in\ZZ$ and as $b,d\not=0$ $bd\not=0$.
  Finally, we see that
  \begin{align*}
    bd(sr)  &= bdsr \\
            &= bsdr \\
            &= ac\,.
  \end{align*}
  Thus by Definition \ref{defn:rational}, $sr$ is rational.
\end{proof}

\guard

\guard

%TODO define the integers
%TODO define integer multiplication.

\begin{defn}
\label{defn:rational}
\index{rational}
  A real number $r$ is said to be \emph{rational} if, and only if, $\exists a,b\in\ZZ$ with $b\not=0$ and $rb= a$.
\end{defn}

We denote the set of rational numbers $\QQ$.


%TODO show the integers are closed under addition and multiplication.
%TODO show multiplication over the integers is commutative.

\begin{prop}
\label{prop:sumOfRaionalsIsRational}
  If $r$ and $s$ are rational, then $r+s$ is rational.
\end{prop}
\begin{proof}
  Fix $r$ and $s$ rational.
  This means, by Definition \ref{defn:rational}, there exists $a,b,c,d\in\ZZ$ with $b,d\not=0$ such that $sb=a$ and $rd=c$.
  Note that as $a,b,c,d\in\ZZ$, $ad+bc,bd\in\ZZ$ and as $b,d\not=0$ $bd\not=0$.
  Finally, we see that
  \begin{align*}
    bd(s+r) &= bds + bdr \\
            &= da + bc\,.
  \end{align*}
  Thus, by Definition \ref{defn:rational}, $s+r$ is rational.
\end{proof}


\begin{prop}
\label{prop:sumOfRationalAndIrrationalIsIrrational}
  The sum of a rational number and an irrational number is irrational.
\end{prop}
\begin{proof}
  Suppose towards a contradiction that this is false.
  Thus, there exists a rational $r$ and an irrational $s$ so that $r+s$ is rational.
  Though, as $r$ is rational by Proposition \ref{prop:productOfRaionalsIsRational}, $-r$ is rational.
  Further, as $r+s$ and $-r$ are ration applying Proposition \ref{prop:sumOfRaionalsIsRational}, $r+s+(-r)=s$ is rational, a contradiction.
\end{proof}


% We need the following for the next discussion.
\guard

\guard

\guard

\begin{defn}
\label{defn:statement}
  A \emph{statement} (or \emph{proposition}) is a sentence that is either true or false, but not both.
\end{defn}

\guard

\input{logic/defns/statement.tex}

% \input{logic/defns/negation.tex}
% \input{logic/defns/disjunction.tex}
% \input{logic/defns/conjunction.tex}

\begin{defn}
\label{defn:statementForm}
\index{statement form}
  A \emph{statement form} (or \emph{proposition form}) is an expression made up of statement variables and logical connections (such as $\neg$, $\vee$, or $\wedge$) which when substituting statements for statement variables becomes a statement.
\end{defn}

\guard

\input{logic/defns/statement.tex}

\begin{defn}
\label{defn:statementTruthValue}
\index{truth value}
  The \emph{truth value} of a given statement is true if that sentence is itself true otherwise, the truth value of that statement is false.
\end{defn}



\begin{defn}
\label{defn:conditionalStatement}
\index{conditional}
  Let $p$ and $q$ be statements forms.
  The \emph{conditional statement} ``$p$ implies $q$'', written $p \rightarrow q$, is the statement form that is false precisely when $p$ is true and $q$ is false ( that is, when the statement ``If $p$, then $q$'' is violated ).

  In the conditional $p\rightarrow q$, $p$ is refered to as the \emph{hypothesis} and $q$ is called the \emph{conclusion}.
\end{defn}


\begin{defn}
\label{defn:contrapositive}
\index{contrapositive}
  The \emph{contrapositive} of a conditional statement form $p\rightarrow q$ is the statement form $\neg q \rightarrow \neg p$.
\end{defn}

% This is redundant.
\guard

\guard

\guard

\begin{defn}
\label{defn:statement}
  A \emph{statement} (or \emph{proposition}) is a sentence that is either true or false, but not both.
\end{defn}

\guard

\input{logic/defns/statement.tex}

% \input{logic/defns/negation.tex}
% \input{logic/defns/disjunction.tex}
% \input{logic/defns/conjunction.tex}

\begin{defn}
\label{defn:statementForm}
\index{statement form}
  A \emph{statement form} (or \emph{proposition form}) is an expression made up of statement variables and logical connections (such as $\neg$, $\vee$, or $\wedge$) which when substituting statements for statement variables becomes a statement.
\end{defn}

\guard

\input{logic/defns/statement.tex}

\begin{defn}
\label{defn:statementTruthValue}
\index{truth value}
  The \emph{truth value} of a given statement is true if that sentence is itself true otherwise, the truth value of that statement is false.
\end{defn}



\begin{defn}
\label{defn:conditionalStatement}
\index{conditional}
  Let $p$ and $q$ be statements forms.
  The \emph{conditional statement} ``$p$ implies $q$'', written $p \rightarrow q$, is the statement form that is false precisely when $p$ is true and $q$ is false ( that is, when the statement ``If $p$, then $q$'' is violated ).

  In the conditional $p\rightarrow q$, $p$ is refered to as the \emph{hypothesis} and $q$ is called the \emph{conclusion}.
\end{defn}

\guard

\guard

\begin{defn}
\label{defn:statement}
  A \emph{statement} (or \emph{proposition}) is a sentence that is either true or false, but not both.
\end{defn}

\guard

\input{logic/defns/statement.tex}

\begin{defn}
\label{defn:predicate}
  A \emph{predicate} is a sentence which contains a finite number of variables which becomes a statement when a value is assigned to each of its variables.
  The \emph{domain} of a predicate variable is the set of possible values which that variable may take.
\end{defn}



\begin{defn}
\label{defn:universalStatement}
\index{universal statement}
	An \emph{universal statement} is a statement asserting that that $P(x)$ holds for every $x$ in $D$, written $\forall x\in D(P(x))$, where  $P(x)$ is a predicate and $D$ is the domain of $x$ in $P(x)$.
	$\forall x\in D(P(x))$ is true if, and only if, $P(x)$ is true for every $x$ in $D$.
\end{defn}


\begin{defn}
\label{defn:universalConditionalStatement}
\index{universal conditional statement}
	An \emph{universal conditional statement} is a statement asserting that that for any $x\in D$ such that $P(x)$, $Q(x)$ holds as well.
	Written $\forall x\in D(P(x)\rightarrow Q(x) )$, where  $P(x)$ and $Q(x)$ are predicates and $D$ is the domain of $x$ in $P(x)$ and $Q(x)$.
	$\forall x\in D(P(x)\rightarrow Q(x))$ is true if, and only if, $P(x)\rightarrow Q(x)$ is true for every $x$ in $D$.
\end{defn}


Proof by contrapositive.\\
Suppose we are tasked with proving a statement of the form $\forall x( P(x)\rightarrow Q(x))$.
Suppose further that the predicate $P(x)$ gives little in the way of describing $Q(x)$ (that is, showing $\forall x( P(x)\rightarrow Q(x))$ is hard).
Then, maybe before we attempt a proof via contradiction, we can attack the problem directly from a different direction.
That is, recall that $\forall x( P(x)\rightarrow Q(x))$ is logically equivalent to its contrapositive $\forall x( \neg Q(x)\rightarrow \neg P(x))$.
So, a proof of $\forall x( P(x)\rightarrow Q(x))$ by contradiction is a direct proof of $\forall x( \neg Q(x)\rightarrow \neg P(x))$.
The following lemma is an example of this technique.
\guard

\guard

%TODO define integer
%TODO define integer multiplication
%TODO define 2

\begin{defn}
\label{defn:even}
\index{even}
  An integer $n$ is said to be \emph{even} if there exists an integer $k$ such that \[ n = 2 k \,.\]
\end{defn}

\guard

\guard

\input{numberTheory/defns/divides.tex}

\begin{thm}
\label{thm:quotientRemainderTheorem}
  \textbf{(Quotient-Remainder Theorem)}
  Given an integer $n$ and positive integer $d$, there exists unique integers $q$ and $r$ such that $0\leq r < d$ and $n=qd + r$.
\end{thm}
\begin{proof}
  Fix $n\in\ZZ$ and $d\in\ZZ^+$.

  First, we will show existence.
  Consider the set of non-negative integers of the form $n-dq$.
  This set is clearly non-empty.
  Thus, as $\NN$ is well-ordered, this set contains a least element, say $r=n-dq$.
  Certianly, $r\geq 0$.
  Now, $r<d$ as as otherwise $n-d(q+1)$ contradicts the minimality of $r$.

  Now to show uniqueness.
  Suppose that there exists $q_0,q_1,r_0,r_1\in\ZZ$ with $0\leq r_0,r_1 <d$ and \[n=dq_0+r_0=dq_1+r_1\,.\]
  Note, as $dq_0+r_0=dq_1+r_1$, it suffices to show $r_0=r_1$.
  Rewriting $dq_0+r_0=dq_1+r_1$, we see that $r_0-r_1=d(q_1-q_0)$ showing that $d|(r_0-r_1)$.
  Further, as $0\leq r_1 <d$, $-d<r_1\leq 0$.
  So, $-d < r_0-r_1 <d$.
  Though, as $d|(r_0-r_1)$ and $-d < r_0-r_1 <d$, $r_0-r_1=0$.
\end{proof}


\begin{prop}
\label{prop:integersAreEvenXorOdd}
  Every integer is either even or odd, but not both.
\end{prop}
\begin{proof}
  Fix $n\in\ZZ$ arbitrary.
  As $2\in\ZZ$ is positive and $n\in\ZZ$, the Quotient Remainder Theorem, Theorem \ref{thm:quotientRemainderTheorem}, applies, and there exist $q,r\in\ZZ$ such that $0\leq r<2$ and $n=2q+r$.
  Though, as $r\in\ZZ$ and $0\leq r<2$, $r=0,1$.
  Therefore, $n=2q$ or $n=2q+1$.
  Thus, $n$ is either even or odd.
  Moreover, as the Quotient Remainder Theorem  give that $q$ and $r$ are unique, $n$ cannot be even and odd.
\end{proof}


\begin{lem}
\label{lem:IfSquareOfIntegerIsEvenThenEven}
  For every integer $n$ if $n^2$ is even, then $n$ is even.
\end{lem}
\begin{proof}
  Fix $n\in\ZZ$ and suppose that $n$ is not even.
  As $n$ is not even, we have by Proposition \ref{prop:integersAreEvenXorOdd} that $n$ is odd.
  As $n$ is odd, there exists $k\in\ZZ$ such that $n=2k+1$.
  So,
  \begin{align*}
    n^2 &= (2k+1)^2 \\
        &= 4k^2 + 4k + 1 \\
        &= 2(2k^2+2k) + 1\,,
  \end{align*}
  showing $n^2$ is odd.
  Again, by Proposition \ref{prop:integersAreEvenXorOdd}, as $n^2$ is odd $n^2$ is not even.
\end{proof}

