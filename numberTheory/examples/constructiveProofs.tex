\guard


\guard

%TODO define integer
%TODO define integer multiplication
%TODO define 2

\begin{defn}
\label{defn:even}
\index{even}
  An integer $n$ is said to be \emph{even} if there exists an integer $k$ such that \[ n = 2 k \,.\]
\end{defn}


%TODO prove distribution of
The following proof uses that $\forall a,b,c\in\ZZ(a(b+c)=ab+ac)$ without proof.

The following are some examples utilizing a style of proof called a \emph{constructive proof}.
This style of proof is used to show existential statements, that is to show the existence of some object.
constructive proofs are those proofs which explicitly produce a desired object.

\begin{exmp}
\label{exmp:constructiveProofs}
  \begin{enumerate}
    \item  Show that there exists an even integer which can be written two ways as the sum of two positive integers.\\
      \begin{proof}
        Note that $n=10=2*5$ is even and \[ n = 5*5 = 7+3 \,.\]
      \end{proof}
    \item Let $r,s\in\ZZ$, show that $22r+18s$ is even.
      \begin{proof}
        Fix $r,s\in\ZZ$.
        To show that $22r+18s$ is even, we must show  that there exists (or construct) $k\in\ZZ$ such that \[ 22r + 18s = 2k\,.\]
        Let $k=11r+9s$.
        Note that
        \begin{align*}
          2k  &= 2(11r+9s)\\
              &= 2*11r + 2*9s\\
              &= 22r + 18s
        \end{align*}
        as desired.
      \end{proof}
    \item Show that $\forall a,b\in\ZZ(a^2=b^2 \rightarrow a=b )$ is not true.
    \begin{proof}
      To show that $\forall a,b\in\ZZ(a^2=b^2 \rightarrow a=b )$ is not true, it sufficese to show that there exists $a,b\in\RR$ distinct such that $a^2=b^2$.
      Consider $a=1$ and $b=-1$.
      Note that $a=1\not=-1=b$ but $1^2=1=(-1)^2$.
    \end{proof}
  \end{enumerate}
\end{exmp}
