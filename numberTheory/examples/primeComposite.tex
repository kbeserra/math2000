\guard


%TODO Show that $\ZZ$ is closed under $+$ and $*$.

\guard

\begin{defn}
\label{defn:prime}
\index{prime}
  An integer $1<p$ is said to be \emph{prime} provided for all positive integers $a$ and $b$ if $ab=p$, then $a=p$ or $b=p$.
\end{defn}

\guard

\guard

\begin{defn}
\label{defn:prime}
\index{prime}
  An integer $1<p$ is said to be \emph{prime} provided for all positive integers $a$ and $b$ if $ab=p$, then $a=p$ or $b=p$.
\end{defn}


\begin{defn}
\label{defn:composite}
\index{prime}
  An integer $1<c$ is said to be \emph{composite} if $c$ there exists integers $a$ and $b$ such that $c=ab$ and $1<a,b<c$
\end{defn}


%TODO Show that
This example is incomplete.
It requires proof that $\forall a,b\in\ZZ^+$ $a,b\leq ab$.
That is, multiplication on $\ZZ^+$ is not decreasing.

\begin{exmp}
\label{exmp:primeComposite}
  \begin{enumerate}
    \item Show that $1$ is not prime.\\
      Note, Definition \ref{defn:prime} of prime requires that a prime integer be greater than $1$.
      $1\not<1$, so $1$ is not prime.
    \item Show that $6$ is composite.\\
      Note that $2*3=6$ and $2,3\not=1$ and $2,3\not=6$.
      So $6$ is not prime.
      
    \item Write the first $2$ primes and justify their primeness.\\
      Note that any prime must be larger than $1$.
      So, we begin our search at the first integer after $1$, $2$.

      Note that as  multiplication on $\ZZ^+$ is not decreasing, the only possible factors of $2$ are in the set $\set{1,2}$.
      As $1*1\not=2$ and $2*2\not=2$ but $1*2=2*1=2$, we have that $2$ is prime.

      The only positive integer less than $3$ but not $1$ or $3$ is $2$.
      Though, we note that none of $1*2$, $2*2$, and $2*3$ are not $3$.
      Thus, the only integers whose product is $3$ are $1$ and $3$.

  \end{enumerate}
\end{exmp}
