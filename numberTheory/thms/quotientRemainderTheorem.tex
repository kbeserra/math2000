\guard

\guard

%TODO define integer
%TODO define integer multiplication

\begin{defn}
\label{defn:divide}
\index{divide}
  An integer $d\not=0$ is said to \emph{divide} an integer $n$ provided, written $d\vert n$, provided there exists an integer $k$ such that $n=dk$.
\end{defn}


\begin{thm}
\label{thm:quotientRemainderTheorem}
  \textbf{(Quotient-Remainder Theorem)}
  Given an integer $n$ and positive integer $d$, there exists unique integers $q$ and $r$ such that $0\leq r < d$ and $n=qd + r$.
\end{thm}
\begin{proof}
  Fix $n\in\ZZ$ and $d\in\ZZ^+$.

  First, we will show existence.
  Consider the set of non-negative integers of the form $n-dq$.
  This set is clearly non-empty.
  Thus, as $\NN$ is well-ordered, this set contains a least element, say $r=n-dq$.
  Certianly, $r\geq 0$.
  Now, $r<d$ as as otherwise $n-d(q+1)$ contradicts the minimality of $r$.

  Now to show uniqueness.
  Suppose that there exists $q_0,q_1,r_0,r_1\in\ZZ$ with $0\leq r_0,r_1 <d$ and \[n=dq_0+r_0=dq_1+r_1\,.\]
  Note, as $dq_0+r_0=dq_1+r_1$, it suffices to show $r_0=r_1$.
  Rewriting $dq_0+r_0=dq_1+r_1$, we see that $r_0-r_1=d(q_1-q_0)$ showing that $d|(r_0-r_1)$.
  Further, as $0\leq r_1 <d$, $-d<r_1\leq 0$.
  So, $-d < r_0-r_1 <d$.
  Though, as $d|(r_0-r_1)$ and $-d < r_0-r_1 <d$, $r_0-r_1=0$.
\end{proof}
