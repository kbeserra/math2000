\guard

\guard

\begin{defn}
\label{defn:prime}
\index{prime}
  An integer $1<p$ is said to be \emph{prime} provided for all positive integers $a$ and $b$ if $ab=p$, then $a=p$ or $b=p$.
\end{defn}

\guard

%TODO define integer
%TODO define integer multiplication

\begin{defn}
\label{defn:divide}
\index{divide}
  An integer $d\not=0$ is said to \emph{divide} an integer $n$ provided, written $d\vert n$, provided there exists an integer $k$ such that $n=dk$.
\end{defn}


\guard

\guard

\begin{defn}
\label{defn:prime}
\index{prime}
  An integer $1<p$ is said to be \emph{prime} provided for all positive integers $a$ and $b$ if $ab=p$, then $a=p$ or $b=p$.
\end{defn}

\guard

\guard

\begin{defn}
\label{defn:prime}
\index{prime}
  An integer $1<p$ is said to be \emph{prime} provided for all positive integers $a$ and $b$ if $ab=p$, then $a=p$ or $b=p$.
\end{defn}


\begin{defn}
\label{defn:composite}
\index{prime}
  An integer $1<c$ is said to be \emph{composite} if $c$ there exists integers $a$ and $b$ such that $c=ab$ and $1<a,b<c$
\end{defn}

\guard

\guard

\begin{defn}
\label{defn:prime}
\index{prime}
  An integer $1<p$ is said to be \emph{prime} provided for all positive integers $a$ and $b$ if $ab=p$, then $a=p$ or $b=p$.
\end{defn}

\guard

\input{numberTheory/defns/prime.tex}

\begin{defn}
\label{defn:composite}
\index{prime}
  An integer $1<c$ is said to be \emph{composite} if $c$ there exists integers $a$ and $b$ such that $c=ab$ and $1<a,b<c$
\end{defn}


\begin{lem}
\label{lem:notPrimeIsComposite}
  For every integer $n>1$, $n$ is prime if, and only if, $n$ is not composite.
\end{lem}
\begin{proof}
  Fix any integer $n>1$.
  Suppose $n$ is prime.
  By the definition of $n$ prime, the only factors of $n$ are $1$ and $n$.
  Thus, $\not\exists r,s\in\ZZ(n=rs\wedge 1<r,s<n)$.
  Showing that $n$ is not composite.

  Suppose that $n$ is not prime.
  As $n$ is not prime, there exists $r,s\in\ZZ$ such that $n=rs$ and neither $r$ nor $s$ are $1$ or $n$.
  Note, as $n$ is positive, $r$ and $s$ share the same sign.
  With out loss of generality, we may assume $r$ and $s$ are positive as otherwise we may replace them with $-r$ and $-s$.
  Finally, as $rs=n$ and multiplication of positive integers is non-decreasing, $r,s\leq n$.
  As neither $r$ nor $s$ is $1$ and $r,s$ is positive we have that $1<r,s$ and $r,s\leq n$.
  Thus, by Definition \ref{defn:composite}, $n$ is composite.
\end{proof}


\begin{lem}
\label{lem:positiveNotUnitIntegerPrimeOrProductOfPrimes}
  Every integer $n>1$ is either prime or a product of primes.
\end{lem}
\begin{proof}
  We show this via strong induction on $n$.
  For the base case, where $n=2$, we note that $2$ is prime.
  Thus, the claim holds at $n=2$.

  Now, suppose that $n>1$ and every integer $m>n$ is either prime or a product of primes.
  If $n$ is prime, there is nothing to show.
  Otherwise $n$ is not prime, so by Lemma \ref{lem:notPrimeIsComposite} $n$ is composite.
  Thus, there exists $r,s\in\ZZ$ such that $1<r,s<n$.
  Though, as $1<r,s<n$ our inductive hypothesis applies to $r$ and $s$, so $r$ and $s$ are a product of primes.
  Whence, $n=rs$ is a product of primes.
\end{proof}


\guard

%TODO define integer
%TODO define integer multiplication

\begin{defn}
\label{defn:divide}
\index{divide}
  An integer $d\not=0$ is said to \emph{divide} an integer $n$ provided, written $d\vert n$, provided there exists an integer $k$ such that $n=dk$.
\end{defn}


\begin{thm}
\label{thm:fundamentalTheoremOfArithmetic}
  \textbf{(Fundamental Theorem Of Arithmetic)}
  Given any integer $n>1$, there exists a positive integer $k$, distinct primes $p_1,p_2,\dots,p_k$ and positive integers $a_1,a_2,\dots,a_k$ such that \[ n=p_1^{a_1}p_2^{a_2}\cdots p_k^{a_k}\] and any other expression is identical up to order.
\end{thm}
\begin{proof}
  First, we will show existence of such a factorization.
  Let $n>1$ be an arbitrary integer.
  If $n$ is prime, we are done.
  Otherwise, by Lemma \ref{lem:positiveNotUnitIntegerPrimeOrProductOfPrimes}, $n$ is a product of primes.
  Say $n = q_1q_2\cdots q_{k'}$ for some $k'\in\NN$.
  Certainly, it need not be the case that the $k_i$s are unique.
  Let $k\leq k'$ be the number of unique primes within $q_1,q_2,\dots q_{k'}$.
  Enumerate the unique primes within $q_1,q_2,\dots q_{k'}$ as $p_1,p_2,\dots,p_k$.
  Finally, for each $i$ let $a_i$ be the number of times $p_i$ occurs within  $q_1,q_2,\dots q_{k'}$.
  Then,
  \begin{align*}
    n &= q_1q_2\cdots q_{k'} \\
      &= p_1^{a_1}p_2^{a_2}\cdots p_k^{a_k}
  \end{align*}
  as desired.
  Though, as $1<n\in\ZZ$ was arbitrary, such a factorization exists for all integers larger than $1$.


  Now, we show uniqueness.
  Suppose, towards a contradiction that there exists an integer which can be factored as above in two distinct ways.
  As there exists such an integer, as $\NN$ is well-ordered by $<$ we may let $n$ be the least $n>1$ with two distinct prime factorizations as above.
  Note, this implies $n$ is not prime.
  Say \[n=p_1^{a_1}p_2^{a_2}\cdots p_k^{a_k} =q_1^{b_1}q_2^{b_2}\cdots q_l^{b_l}\,.\]
  Note, as $n$ is not prime $k,l\geq 2$.
  As $n=p_1^{a_1}p_2^{a_2}\cdots p_k^{a_k}$, $p_1$ divides $n$.
  So, $n$ divides $q_1^{b_1}q_2^{b_2}\cdots q_l^{b_l}$, and so there exists some $i<l$ so that $p_1\vert q_i$.
  Without loss of generalit, we may reorder the $q_i$s so that $p_1\vert q_1$.
  As $q_1$ is prime and $p_1\vert q_1$, $p_1=q_1$.
  Thus \[ p_1^{a_1-1}p_2^{a_2}\cdots p_k^{a_k} = q_1^{b_1-1}q_2^{b_2}\cdots q_l^{b_l}\] are distinct prime factorizations for an integer below $n$. contradicting the minimality of $n$.
  Thus, no such $n$ may exist.
\end{proof}
