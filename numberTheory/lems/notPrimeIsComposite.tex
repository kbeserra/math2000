\guard

\guard

\begin{defn}
\label{defn:prime}
\index{prime}
  An integer $1<p$ is said to be \emph{prime} provided for all positive integers $a$ and $b$ if $ab=p$, then $a=p$ or $b=p$.
\end{defn}

\guard

\guard

\begin{defn}
\label{defn:prime}
\index{prime}
  An integer $1<p$ is said to be \emph{prime} provided for all positive integers $a$ and $b$ if $ab=p$, then $a=p$ or $b=p$.
\end{defn}


\begin{defn}
\label{defn:composite}
\index{prime}
  An integer $1<c$ is said to be \emph{composite} if $c$ there exists integers $a$ and $b$ such that $c=ab$ and $1<a,b<c$
\end{defn}


\begin{lem}
\label{lem:notPrimeIsComposite}
  For every integer $n>1$, $n$ is prime if, and only if, $n$ is not composite.
\end{lem}
\begin{proof}
  Fix any integer $n>1$.
  Suppose $n$ is prime.
  By the definition of $n$ prime, the only factors of $n$ are $1$ and $n$.
  Thus, $\not\exists r,s\in\ZZ(n=rs\wedge 1<r,s<n)$.
  Showing that $n$ is not composite.

  Suppose that $n$ is not prime.
  As $n$ is not prime, there exists $r,s\in\ZZ$ such that $n=rs$ and neither $r$ nor $s$ are $1$ or $n$.
  Note, as $n$ is positive, $r$ and $s$ share the same sign.
  With out loss of generality, we may assume $r$ and $s$ are positive as otherwise we may replace them with $-r$ and $-s$.
  Finally, as $rs=n$ and multiplication of positive integers is non-decreasing, $r,s\leq n$.
  As neither $r$ nor $s$ is $1$ and $r,s$ is positive we have that $1<r,s$ and $r,s\leq n$.
  Thus, by Definition \ref{defn:composite}, $n$ is composite.
\end{proof}
