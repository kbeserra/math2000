\guard

\guard

\begin{defn}
\label{defn:prime}
\index{prime}
  An integer $1<p$ is said to be \emph{prime} provided for all positive integers $a$ and $b$ if $ab=p$, then $a=p$ or $b=p$.
\end{defn}

\guard

%TODO define integer
%TODO define integer multiplication

\begin{defn}
\label{defn:divide}
\index{divide}
  An integer $d\not=0$ is said to \emph{divide} an integer $n$ provided, written $d\vert n$, provided there exists an integer $k$ such that $n=dk$.
\end{defn}

\guard

\guard

%TODO define integer
%TODO define integer multiplication

\begin{defn}
\label{defn:divide}
\index{divide}
  An integer $d\not=0$ is said to \emph{divide} an integer $n$ provided, written $d\vert n$, provided there exists an integer $k$ such that $n=dk$.
\end{defn}


\begin{prop}
\label{prop:dividesIsTransitive}
  Let $a,b,c\in\ZZ$.
  If $a\vert b$ and $b\vert c$, then $a\vert c$.
\end{prop}
\begin{proof}
  Fix $a,b,c\in\ZZ$ such that $a\vert b$ and $b\vert c$.
  As $a\vert b$ and $b\vert c$ there exists $r,s\in\ZZ$ such that \[b=ra\text{  and  }c=sb\,.\]
  So,
  \begin{align*}
    c &= sb \\
      &= s(ra) \\
      &= (sr)a \,.
  \end{align*}
  And as $s,r\in\ZZ$ $sr\in\ZZ$.
  Therefore, $a\vert c$.
\end{proof}

\guard

\guard

\begin{defn}
\label{defn:prime}
\index{prime}
  An integer $1<p$ is said to be \emph{prime} provided for all positive integers $a$ and $b$ if $ab=p$, then $a=p$ or $b=p$.
\end{defn}

\guard

\guard

\begin{defn}
\label{defn:prime}
\index{prime}
  An integer $1<p$ is said to be \emph{prime} provided for all positive integers $a$ and $b$ if $ab=p$, then $a=p$ or $b=p$.
\end{defn}


\begin{defn}
\label{defn:composite}
\index{prime}
  An integer $1<c$ is said to be \emph{composite} if $c$ there exists integers $a$ and $b$ such that $c=ab$ and $1<a,b<c$
\end{defn}


\begin{lem}
\label{lem:notPrimeIsComposite}
  For every integer $n>1$, $n$ is prime if, and only if, $n$ is not composite.
\end{lem}
\begin{proof}
  Fix any integer $n>1$.
  Suppose $n$ is prime.
  By the definition of $n$ prime, the only factors of $n$ are $1$ and $n$.
  Thus, $\not\exists r,s\in\ZZ(n=rs\wedge 1<r,s<n)$.
  Showing that $n$ is not composite.

  Suppose that $n$ is not prime.
  As $n$ is not prime, there exists $r,s\in\ZZ$ such that $n=rs$ and neither $r$ nor $s$ are $1$ or $n$.
  Note, as $n$ is positive, $r$ and $s$ share the same sign.
  With out loss of generality, we may assume $r$ and $s$ are positive as otherwise we may replace them with $-r$ and $-s$.
  Finally, as $rs=n$ and multiplication of positive integers is non-decreasing, $r,s\leq n$.
  As neither $r$ nor $s$ is $1$ and $r,s$ is positive we have that $1<r,s$ and $r,s\leq n$.
  Thus, by Definition \ref{defn:composite}, $n$ is composite.
\end{proof}

%TODO argue that not prime is composite

\begin{prop}
\label{prop:everyPositiveNonUnitIntegerDivisibleByAPrime}
  For any integer $n$ if $n>1$, then $n$ is divisible by a prime.
\end{prop}
\begin{proof}
  Fix $n\in\ZZ$ such that $n>1$.
  If $n$ is prime, then there is nothing to show.
  So, suppose that $n$ is not prime.

  As $n$ is not prime, it is composite, and there exists integers $r_0,s_0$ such that $n=r_0s_0$ and $1<r_0,s_0<n$.
  As $n=r_0s_0$, $r_0\vert n$.

  If $r_0$ is prime, we are dode.
  Otherwise, there exists $r_1,s_1\in\ZZ$ such that $1<r_1,s_1<r_0$ and $r_0=r_1s_1$.
  As $r_0=r_1s_1$, we have that $r_1\vert r_0$.
  Though, as $r_1\vert r_0$ and $r_0\vert n$, by Proposition \ref{prop:dividesIsTransitive}, $r_1\vert n$.

  If $r_1$ is prime, we are done.
  Otherwise, we continue in this fashion producing ever smaller integers which divide $n$.
  This process must terminate, as each successive factor is greater than $1$ but less than $n$, and there are only finitely many integers between $1$ and $n$.
\end{proof}
