\section{Some Basics Definitions of Integers}


\guard


\guard

%TODO define integer
%TODO define integer multiplication
%TODO define 2

\begin{defn}
\label{defn:even}
\index{even}
  An integer $n$ is said to be \emph{even} if there exists an integer $k$ such that \[ n = 2 k \,.\]
\end{defn}

\guard

\begin{defn}
\label{defn:odd}
\index{odd}
  An integer $n$ is said to be \emph{odd} if there exists an integer $k$ such that \[ n = 2 k + 1\,.\]
\end{defn}


%TODO Show that $\ZZ$ is closed under $+$ and $*$.
The following exercise is incomplete, it requires the fact that $\ZZ$ is closed under $+$ and $*$.
Currently, that fact is asserted without proof.

\begin{exmp}
\label{exmp:evenOdd}
  Use the definitions for even and odd, Definitions \ref{defn:even} and \ref{defn:odd}, to justify that:
  \begin{enumerate}
    \item $0$ is even.\\
      Note that $0\in\ZZ$ and $0 = 2(0)$, thus by Definition \ref{defn:even} $0$ is even.
    \item $-301$ is odd.\\
      Note that $-151\in\ZZ$ and $-301=2(-151) + 1$, thus by Definition \ref{defn:odd} $-301$ is odd.
    \item if $a,b\in\ZZ$, then $6a^2b$ is even.\\
      Fix $a,b\in\ZZ$.
      Note that as $\ZZ$ is closed under $*$ and $3\in\ZZ$, $3*a^2b=3aab\in\ZZ$.
      Similarly, as $6\in\ZZ$ $6a^2b\in\ZZ$
      Thus, by Definition \ref{defn:even} $6a^2b=2(3a^2b)$ is even.
  \end{enumerate}
\end{exmp}

\guard


%TODO Show that $\ZZ$ is closed under $+$ and $*$.

\guard

\begin{defn}
\label{defn:prime}
\index{prime}
  An integer $1<p$ is said to be \emph{prime} provided for all positive integers $a$ and $b$ if $ab=p$, then $a=p$ or $b=p$.
\end{defn}

\guard

\guard

\begin{defn}
\label{defn:prime}
\index{prime}
  An integer $1<p$ is said to be \emph{prime} provided for all positive integers $a$ and $b$ if $ab=p$, then $a=p$ or $b=p$.
\end{defn}


\begin{defn}
\label{defn:composite}
\index{prime}
  An integer $1<c$ is said to be \emph{composite} if $c$ there exists integers $a$ and $b$ such that $c=ab$ and $1<a,b<c$
\end{defn}


%TODO Show that
This example is incomplete.
It requires proof that $\forall a,b\in\ZZ^+$ $a,b\leq ab$.
That is, multiplication on $\ZZ^+$ is not decreasing.

\begin{exmp}
\label{exmp:primeComposite}
  \begin{enumerate}
    \item Show that $1$ is not prime.\\
      Note, Definition \ref{defn:prime} of prime requires that a prime integer be greater than $1$.
      $1\not<1$, so $1$ is not prime.
    \item Show that $6$ is composite.\\
      Note that $2*3=6$ and $2,3\not=1$ and $2,3\not=6$.
      So $6$ is not prime.
      
    \item Write the first $2$ primes and justify their primeness.\\
      Note that any prime must be larger than $1$.
      So, we begin our search at the first integer after $1$, $2$.

      Note that as  multiplication on $\ZZ^+$ is not decreasing, the only possible factors of $2$ are in the set $\set{1,2}$.
      As $1*1\not=2$ and $2*2\not=2$ but $1*2=2*1=2$, we have that $2$ is prime.

      The only positive integer less than $3$ but not $1$ or $3$ is $2$.
      Though, we note that none of $1*2$, $2*2$, and $2*3$ are not $3$.
      Thus, the only integers whose product is $3$ are $1$ and $3$.

  \end{enumerate}
\end{exmp}


\guard


\guard

%TODO define integer
%TODO define integer multiplication
%TODO define 2

\begin{defn}
\label{defn:even}
\index{even}
  An integer $n$ is said to be \emph{even} if there exists an integer $k$ such that \[ n = 2 k \,.\]
\end{defn}


%TODO prove distribution of
The following proof uses that $\forall a,b,c\in\ZZ(a(b+c)=ab+ac)$ without proof.

The following are some examples utilizing a style of proof called a \emph{constructive proof}.
This style of proof is used to show existential statements, that is to show the existence of some object.
constructive proofs are those proofs which explicitly produce a desired object.

\begin{exmp}
\label{exmp:constructiveProofs}
  \begin{enumerate}
    \item  Show that there exists an even integer which can be written two ways as the sum of two positive integers.\\
      \begin{proof}
        Note that $n=10=2*5$ is even and \[ n = 5*5 = 7+3 \,.\]
      \end{proof}
    \item Let $r,s\in\ZZ$, show that $22r+18s$ is even.
      \begin{proof}
        Fix $r,s\in\ZZ$.
        To show that $22r+18s$ is even, we must show  that there exists (or construct) $k\in\ZZ$ such that \[ 22r + 18s = 2k\,.\]
        Let $k=11r+9s$.
        Note that
        \begin{align*}
          2k  &= 2(11r+9s)\\
              &= 2*11r + 2*9s\\
              &= 22r + 18s
        \end{align*}
        as desired.
      \end{proof}
    \item Show that $\forall a,b\in\ZZ(a^2=b^2 \rightarrow a=b )$ is not true.
    \begin{proof}
      To show that $\forall a,b\in\ZZ(a^2=b^2 \rightarrow a=b )$ is not true, it sufficese to show that there exists $a,b\in\RR$ distinct such that $a^2=b^2$.
      Consider $a=1$ and $b=-1$.
      Note that $a=1\not=-1=b$ but $1^2=1=(-1)^2$.
    \end{proof}
  \end{enumerate}
\end{exmp}


Direct proofs:

\guard

\guard

%TODO define integer
%TODO define integer multiplication
%TODO define 2

\begin{defn}
\label{defn:even}
\index{even}
  An integer $n$ is said to be \emph{even} if there exists an integer $k$ such that \[ n = 2 k \,.\]
\end{defn}


\begin{prop}
\label{prop:sumOfEvenIsEven}
  The sum of two even integers is even.
\end{prop}
\begin{proof}
  Let $a$ and $b$ be arbitrary even integers.
  As $a$ and $b$ are even there exists $k,l\in\ZZ$ such that $a=2k$ and $b=2l$.
  Then, we note that
  \begin{align*}
    a + b &= 2k + 2l \\
          &= 2( k + l )\,.
  \end{align*}
  Though, as $k,l\in\ZZ$, $k+l\in\ZZ$, set $m=k+l$.
  Then we note that $m\in\ZZ$ and $a+b = 2m$, and whence $a+b$ is even.
\end{proof}


\guard

\guard

\guard

%TODO define integer
%TODO define integer multiplication
%TODO define 2

\begin{defn}
\label{defn:even}
\index{even}
  An integer $n$ is said to be \emph{even} if there exists an integer $k$ such that \[ n = 2 k \,.\]
\end{defn}

\guard

\begin{defn}
\label{defn:odd}
\index{odd}
  An integer $n$ is said to be \emph{odd} if there exists an integer $k$ such that \[ n = 2 k + 1\,.\]
\end{defn}


\begin{prop}
\label{prop:sumOfOddIsEven}
  The sum of two odd integers is even.
\end{prop}
\begin{proof}
  Left as an exercise to the reader.
\end{proof}


\begin{exercise}
\label{exercise:sumOfOddIsEven}
  Prove Proposition \ref{prop:sumOfOddIsEven}.
\end{exercise}

\guard

\guard

\guard

%TODO define integer
%TODO define integer multiplication
%TODO define 2

\begin{defn}
\label{defn:even}
\index{even}
  An integer $n$ is said to be \emph{even} if there exists an integer $k$ such that \[ n = 2 k \,.\]
\end{defn}

\guard

\begin{defn}
\label{defn:odd}
\index{odd}
  An integer $n$ is said to be \emph{odd} if there exists an integer $k$ such that \[ n = 2 k + 1\,.\]
\end{defn}


\begin{prop}
\label{prop:sumOfEvenAndOddIsOdd}
  The sum of an even integer and an odd integer is odd.
\end{prop}
\begin{proof}
  Left as an exercise to the reader.
\end{proof}


\begin{exercise}
\label{exercise:sumOfEvenAndOddIsOdd}
  Prove Proposition \ref{prop:sumOfEvenAndOddIsOdd}.
\end{exercise}

\guard

\guard

\guard

\begin{defn}
\label{defn:odd}
\index{odd}
  An integer $n$ is said to be \emph{odd} if there exists an integer $k$ such that \[ n = 2 k + 1\,.\]
\end{defn}


\begin{prop}
\label{prop:productOfOddIsOdd}
  The product of two odd integers is odd.
\end{prop}
\begin{proof}
  Left as an exercise to the reader.
\end{proof}


\begin{exercise}
\label{exercise:productOfOddIsOdd}
  Prove Proposition \ref{prop:productOfOddIsOdd}.
\end{exercise}

\guard

\guard

\guard

%TODO define integer
%TODO define integer multiplication
%TODO define 2

\begin{defn}
\label{defn:even}
\index{even}
  An integer $n$ is said to be \emph{even} if there exists an integer $k$ such that \[ n = 2 k \,.\]
\end{defn}

\guard

\begin{defn}
\label{defn:odd}
\index{odd}
  An integer $n$ is said to be \emph{odd} if there exists an integer $k$ such that \[ n = 2 k + 1\,.\]
\end{defn}


\begin{prop}
\label{prop:productOfEvenAndOddIsEven}
  The product of an even integer and an odd integer is even.
\end{prop}
\begin{proof}
  Left as an exercise to the reader.
\end{proof}


\begin{exercise}
\label{exercise:productOfEvenAndOddIsEven}
  Prove Proposition \ref{prop:productOfEvenAndOddIsEven}.
\end{exercise}

\guard

\guard

\guard

%TODO define integer
%TODO define integer multiplication
%TODO define 2

\begin{defn}
\label{defn:even}
\index{even}
  An integer $n$ is said to be \emph{even} if there exists an integer $k$ such that \[ n = 2 k \,.\]
\end{defn}


\begin{prop}
\label{prop:sumOfEvenIsEven}
  The sum of two even integers is even.
\end{prop}
\begin{proof}
  Let $a$ and $b$ be arbitrary even integers.
  As $a$ and $b$ are even there exists $k,l\in\ZZ$ such that $a=2k$ and $b=2l$.
  Then, we note that
  \begin{align*}
    a + b &= 2k + 2l \\
          &= 2( k + l )\,.
  \end{align*}
  Though, as $k,l\in\ZZ$, $k+l\in\ZZ$, set $m=k+l$.
  Then we note that $m\in\ZZ$ and $a+b = 2m$, and whence $a+b$ is even.
\end{proof}

\guard

\guard

%TODO define integer
%TODO define integer multiplication
%TODO define 2

\begin{defn}
\label{defn:even}
\index{even}
  An integer $n$ is said to be \emph{even} if there exists an integer $k$ such that \[ n = 2 k \,.\]
\end{defn}

\guard

\begin{defn}
\label{defn:odd}
\index{odd}
  An integer $n$ is said to be \emph{odd} if there exists an integer $k$ such that \[ n = 2 k + 1\,.\]
\end{defn}


\begin{prop}
\label{prop:sumOfOddIsEven}
  The sum of two odd integers is even.
\end{prop}
\begin{proof}
  Left as an exercise to the reader.
\end{proof}

\guard

\guard

%TODO define integer
%TODO define integer multiplication
%TODO define 2

\begin{defn}
\label{defn:even}
\index{even}
  An integer $n$ is said to be \emph{even} if there exists an integer $k$ such that \[ n = 2 k \,.\]
\end{defn}

\guard

\begin{defn}
\label{defn:odd}
\index{odd}
  An integer $n$ is said to be \emph{odd} if there exists an integer $k$ such that \[ n = 2 k + 1\,.\]
\end{defn}


\begin{prop}
\label{prop:sumOfEvenAndOddIsOdd}
  The sum of an even integer and an odd integer is odd.
\end{prop}
\begin{proof}
  Left as an exercise to the reader.
\end{proof}

\guard

\guard

\begin{defn}
\label{defn:odd}
\index{odd}
  An integer $n$ is said to be \emph{odd} if there exists an integer $k$ such that \[ n = 2 k + 1\,.\]
\end{defn}


\begin{prop}
\label{prop:productOfOddIsOdd}
  The product of two odd integers is odd.
\end{prop}
\begin{proof}
  Left as an exercise to the reader.
\end{proof}

\guard

\guard

%TODO define integer
%TODO define integer multiplication
%TODO define 2

\begin{defn}
\label{defn:even}
\index{even}
  An integer $n$ is said to be \emph{even} if there exists an integer $k$ such that \[ n = 2 k \,.\]
\end{defn}

\guard

\begin{defn}
\label{defn:odd}
\index{odd}
  An integer $n$ is said to be \emph{odd} if there exists an integer $k$ such that \[ n = 2 k + 1\,.\]
\end{defn}


\begin{prop}
\label{prop:productOfEvenAndOddIsEven}
  The product of an even integer and an odd integer is even.
\end{prop}
\begin{proof}
  Left as an exercise to the reader.
\end{proof}


\begin{exercise}
\label{exercise:showExpressionIsInteger}
  Use Propositions \ref{prop:sumOfEvenIsEven}. \ref{prop:sumOfOddIsEven}, \ref{prop:sumOfEvenAndOddIsOdd}, \ref{prop:productOfOddIsOdd}, and \ref{prop:productOfEvenAndOddIsEven} to show that if $a\in\ZZ$ is even and $b\in\ZZ$ is odd, then \[ \frac{a^2+b^2+1}{2}\] is an integer.
\end{exercise}


\guard

\guard

%TODO define the integers
%TODO define integer multiplication.

\begin{defn}
\label{defn:rational}
\index{rational}
  A real number $r$ is said to be \emph{rational} if, and only if, $\exists a,b\in\ZZ$ with $b\not=0$ and $rb= a$.
\end{defn}

We denote the set of rational numbers $\QQ$.


%TODO fill this out.
\begin{exmp}
\label{exmp:someRationals}
  \begin{enumerate}
    \item Show $\frac{10}{3}$, $-\frac{5}{34}$, $0.28$, are rational. \\
    \item is $0$ rational?
    \item is $0.\overline{21}$ rational?
    \item if $m,n$ are rational, show $\frac{m+n}{nm}$ is rational.
  \end{enumerate}
\end{exmp}

\guard

\guard

%TODO define the integers
%TODO define integer multiplication.

\begin{defn}
\label{defn:rational}
\index{rational}
  A real number $r$ is said to be \emph{rational} if, and only if, $\exists a,b\in\ZZ$ with $b\not=0$ and $rb= a$.
\end{defn}

We denote the set of rational numbers $\QQ$.


\begin{prop}
\label{prop:integersAreRational}
  $\forall n\in\ZZ( n\in\QQ )$.
\end{prop}
\begin{proof}
  Fix $n\in\ZZ$ arbitrary.
  To show $n$ is rational, i.e. $n\in\QQ$, it suffices to show that there exists $a,b\in\ZZ$ such that $n=\frac{a}{b}$.
  Note $n,1\in\ZZ$, $1\not=0$, and $1n = n$.
  Thus by Definition \ref{defn:rational}, $n$ is rational.
\end{proof}


\guard

\guard

\guard

%TODO define the integers
%TODO define integer multiplication.

\begin{defn}
\label{defn:rational}
\index{rational}
  A real number $r$ is said to be \emph{rational} if, and only if, $\exists a,b\in\ZZ$ with $b\not=0$ and $rb= a$.
\end{defn}

We denote the set of rational numbers $\QQ$.


\begin{prop}
\label{prop:integersAreRational}
  $\forall n\in\ZZ( n\in\QQ )$.
\end{prop}
\begin{proof}
  Fix $n\in\ZZ$ arbitrary.
  To show $n$ is rational, i.e. $n\in\QQ$, it suffices to show that there exists $a,b\in\ZZ$ such that $n=\frac{a}{b}$.
  Note $n,1\in\ZZ$, $1\not=0$, and $1n = n$.
  Thus by Definition \ref{defn:rational}, $n$ is rational.
\end{proof}

\guard

\guard

%TODO define the integers
%TODO define integer multiplication.

\begin{defn}
\label{defn:rational}
\index{rational}
  A real number $r$ is said to be \emph{rational} if, and only if, $\exists a,b\in\ZZ$ with $b\not=0$ and $rb= a$.
\end{defn}

We denote the set of rational numbers $\QQ$.


%TODO show the integers are closed under addition and multiplication.
%TODO show multiplication over the integers is commutative.

\begin{prop}
\label{prop:sumOfRaionalsIsRational}
  If $r$ and $s$ are rational, then $r+s$ is rational.
\end{prop}
\begin{proof}
  Fix $r$ and $s$ rational.
  This means, by Definition \ref{defn:rational}, there exists $a,b,c,d\in\ZZ$ with $b,d\not=0$ such that $sb=a$ and $rd=c$.
  Note that as $a,b,c,d\in\ZZ$, $ad+bc,bd\in\ZZ$ and as $b,d\not=0$ $bd\not=0$.
  Finally, we see that
  \begin{align*}
    bd(s+r) &= bds + bdr \\
            &= da + bc\,.
  \end{align*}
  Thus, by Definition \ref{defn:rational}, $s+r$ is rational.
\end{proof}


\begin{prop}
\label{prop:doubleOfRationalIsRational}
  The double of a rational is rational.
\end{prop}
\begin{proof}
  Fix $r\in\QQ$.
  As $r\in\QQ$ we have by Proposition \ref{prop:sumOfRaionalsIsRational}, the double of $r$, $2r=r+r$ is rational.
\end{proof}

\guard

\guard

%TODO define the integers
%TODO define integer multiplication.

\begin{defn}
\label{defn:rational}
\index{rational}
  A real number $r$ is said to be \emph{rational} if, and only if, $\exists a,b\in\ZZ$ with $b\not=0$ and $rb= a$.
\end{defn}

We denote the set of rational numbers $\QQ$.


%TODO show the integers are closed under addition and multiplication.
%TODO show multiplication over the integers is commutative.

\begin{prop}
\label{prop:productOfRaionalsIsRational}
  If $r$ and $s$ are rational, then $rs$ is rational.
\end{prop}
\begin{proof}
  Fix $r$ and $s$ rational.
  This means, by Definition \ref{defn:rational}, there exists $a,b,c,d\in\ZZ$ with $b,d\not=0$ such that $sb=a$ and $rd=c$.
  Note that as $a,b,c,d\in\ZZ$, $ac,bd\in\ZZ$ and as $b,d\not=0$ $bd\not=0$.
  Finally, we see that
  \begin{align*}
    bd(sr)  &= bdsr \\
            &= bsdr \\
            &= ac\,.
  \end{align*}
  Thus by Definition \ref{defn:rational}, $sr$ is rational.
\end{proof}



\guard

\guard

%TODO define integer
%TODO define integer multiplication

\begin{defn}
\label{defn:divide}
\index{divide}
  An integer $d\not=0$ is said to \emph{divide} an integer $n$ provided, written $d\vert n$, provided there exists an integer $k$ such that $n=dk$.
\end{defn}


%TODO fill this out.
\begin{exmp}
\label{exmp:everyIntegerDivides0}
  If $n\in\ZZ$, then does $n$ divide $0$?\\
  Yes, as for any integer $n$, $0\in\ZZ$ and $0=0k$.
  Whenc,e $n\vert 0$.
\end{exmp}


\guard

\guard

%TODO define integer
%TODO define integer multiplication

\begin{defn}
\label{defn:divide}
\index{divide}
  An integer $d\not=0$ is said to \emph{divide} an integer $n$ provided, written $d\vert n$, provided there exists an integer $k$ such that $n=dk$.
\end{defn}

\guard

\guard

%TODO define integer
%TODO define integer multiplication

\begin{defn}
\label{defn:divide}
\index{divide}
  An integer $d\not=0$ is said to \emph{divide} an integer $n$ provided, written $d\vert n$, provided there exists an integer $k$ such that $n=dk$.
\end{defn}


%TODO show multiplication is of positive integers is non-decreasing.

\begin{prop}
\label{prop:positiveDividesImpliesNotGreater}
  For any integers $a$ and $b$ if $a$ and $b$ are positive and $a\vert b$, then $a\leq b$.
\end{prop}
\begin{proof}
  Fix $a,b\in\ZZ$ such that $a$ and $b$ are positive and $a\vert b$.
  As $a\vert b$ there exists $k\in\ZZ$ such that $b=ak$.
  As $a,b>0$, $\frac{b}{a}>0$, thus \[ 0 < \frac{b}{a}=k\,.\]
  Whence, $k>0$.
  As $k\in\ZZ$ and $k>0$, $k\geq 1$.
  Finally, we have \[ a \leq ka = b \].
\end{proof}



\begin{prop}
\label{prop:divisorsOf1}
  The divisors of $1$ are $1$ and $-1$.
\end{prop}
\begin{proof}
  Note that $1=1\cdot 1=-1\cdot-1$.
  So, $-1,1\vert 1$.

  Now, it suffices to show that $\forall m\in\ZZ( m\vert 1\rightarrow m=1,-1)$.
  Suppose $m\in\ZZ$ and $m\vert 1$.
  As $m\vert 1$ there exists $n\in\ZZ$ such that $mn=1$
  We note that either $m$ and $n$ are both positive, or both negative.
  Case 1. $m$ and $n$ are both positive.
    Then Proposition \ref{prop:positiveDividesImpliesNotGreater}, $m\leq 1$.
    But, $m$ is positive, so $0<m\leq 1$.
    Finally, as $m\in\ZZ$ and $0<m\leq 1$, $m=1$.
  Case 2. $m$ and $n$ are both negative.
    Then $(-m)(-n)=mn=1$.
    So, $-m$ and $-n$ are positive.
    Thus, the previous case gives that $-m=1$.
    Whence, $m=-1$.


\end{proof}

\guard

\guard

\begin{defn}
\label{defn:prime}
\index{prime}
  An integer $1<p$ is said to be \emph{prime} provided for all positive integers $a$ and $b$ if $ab=p$, then $a=p$ or $b=p$.
\end{defn}

\guard

%TODO define integer
%TODO define integer multiplication

\begin{defn}
\label{defn:divide}
\index{divide}
  An integer $d\not=0$ is said to \emph{divide} an integer $n$ provided, written $d\vert n$, provided there exists an integer $k$ such that $n=dk$.
\end{defn}

\guard

\guard

%TODO define integer
%TODO define integer multiplication

\begin{defn}
\label{defn:divide}
\index{divide}
  An integer $d\not=0$ is said to \emph{divide} an integer $n$ provided, written $d\vert n$, provided there exists an integer $k$ such that $n=dk$.
\end{defn}


\begin{prop}
\label{prop:dividesIsTransitive}
  Let $a,b,c\in\ZZ$.
  If $a\vert b$ and $b\vert c$, then $a\vert c$.
\end{prop}
\begin{proof}
  Fix $a,b,c\in\ZZ$ such that $a\vert b$ and $b\vert c$.
  As $a\vert b$ and $b\vert c$ there exists $r,s\in\ZZ$ such that \[b=ra\text{  and  }c=sb\,.\]
  So,
  \begin{align*}
    c &= sb \\
      &= s(ra) \\
      &= (sr)a \,.
  \end{align*}
  And as $s,r\in\ZZ$ $sr\in\ZZ$.
  Therefore, $a\vert c$.
\end{proof}


%TODO argue that not prime is composite

\begin{prop}
\label{prop:everyPositiveNonUnitIntegerDivisibleByAPrime}
  For any integer $n$ if $n>1$, then $n$ is divisible by a prime.
\end{prop}
\begin{proof}
  Fix $n\in\ZZ$ such that $n>1$.
  If $n$ is prime, then there is nothing to show.
  So, suppose that $n$ is not prime.

  As $n$ is not prime, it is composite, and there exists integers $r_0,s_0$ such that $n=r_0s_0$ and $1<r_0,s_0<n$.
  As $n=r_0s_0$, $r_0\vert n$.

  If $r_0$ is prime, we are dode.
  Otherwise, there exists $r_1,s_1\in\ZZ$ such that $1<r_1,s_1<r_0$ and $r_0=r_1s_1$.
  As $r_0=r_1s_1$, we have that $r_1\vert r_0$.
  Though, as $r_1\vert r_0$ and $r_0\vert n$, by Proposition \ref{prop:dividesIsTransitive}, $r_1\vert n$.

  If $r_1$ is prime, we are done.
  Otherwise, we continue in this fashion producing ever smaller integers which divide $n$.
  This process must terminate, as each successive factor is greater than $1$ but less than $n$, and there are only finitely many integers between $1$ and $n$.
\end{proof}


The proof of this next lemma requires a technique which we have not covered, yet.
This techniques is called strong induction.
Strong induction is a strengthening of regular induction.
Regular induction is a proof technique to show a claim holds over a well-ordered sets (which do not contain a limit -- this is outside the scope of this class).
You show that a claim holds at a base cases.
After the base case you show the successor case; that if the claim holds at a specific place, then it necessarily holds at the next place.
After you show these two things,
Strong induction differs from regular, weak, induction in the successor case.
In strong induction, we suppose that the claim holds everywhere at and bellow a point and then show that it holds at the next point.

The above exposition is not necessary.
Though, as induction is one of the more valuable skills learned in this course, I want to get a little jump start the the subject.
\guard

\guard

\begin{defn}
\label{defn:prime}
\index{prime}
  An integer $1<p$ is said to be \emph{prime} provided for all positive integers $a$ and $b$ if $ab=p$, then $a=p$ or $b=p$.
\end{defn}

\guard

\guard

\begin{defn}
\label{defn:prime}
\index{prime}
  An integer $1<p$ is said to be \emph{prime} provided for all positive integers $a$ and $b$ if $ab=p$, then $a=p$ or $b=p$.
\end{defn}


\begin{defn}
\label{defn:composite}
\index{prime}
  An integer $1<c$ is said to be \emph{composite} if $c$ there exists integers $a$ and $b$ such that $c=ab$ and $1<a,b<c$
\end{defn}

\guard

\guard

\begin{defn}
\label{defn:prime}
\index{prime}
  An integer $1<p$ is said to be \emph{prime} provided for all positive integers $a$ and $b$ if $ab=p$, then $a=p$ or $b=p$.
\end{defn}

\guard

\input{numberTheory/defns/prime.tex}

\begin{defn}
\label{defn:composite}
\index{prime}
  An integer $1<c$ is said to be \emph{composite} if $c$ there exists integers $a$ and $b$ such that $c=ab$ and $1<a,b<c$
\end{defn}


\begin{lem}
\label{lem:notPrimeIsComposite}
  For every integer $n>1$, $n$ is prime if, and only if, $n$ is not composite.
\end{lem}
\begin{proof}
  Fix any integer $n>1$.
  Suppose $n$ is prime.
  By the definition of $n$ prime, the only factors of $n$ are $1$ and $n$.
  Thus, $\not\exists r,s\in\ZZ(n=rs\wedge 1<r,s<n)$.
  Showing that $n$ is not composite.

  Suppose that $n$ is not prime.
  As $n$ is not prime, there exists $r,s\in\ZZ$ such that $n=rs$ and neither $r$ nor $s$ are $1$ or $n$.
  Note, as $n$ is positive, $r$ and $s$ share the same sign.
  With out loss of generality, we may assume $r$ and $s$ are positive as otherwise we may replace them with $-r$ and $-s$.
  Finally, as $rs=n$ and multiplication of positive integers is non-decreasing, $r,s\leq n$.
  As neither $r$ nor $s$ is $1$ and $r,s$ is positive we have that $1<r,s$ and $r,s\leq n$.
  Thus, by Definition \ref{defn:composite}, $n$ is composite.
\end{proof}


\begin{lem}
\label{lem:positiveNotUnitIntegerPrimeOrProductOfPrimes}
  Every integer $n>1$ is either prime or a product of primes.
\end{lem}
\begin{proof}
  We show this via strong induction on $n$.
  For the base case, where $n=2$, we note that $2$ is prime.
  Thus, the claim holds at $n=2$.

  Now, suppose that $n>1$ and every integer $m>n$ is either prime or a product of primes.
  If $n$ is prime, there is nothing to show.
  Otherwise $n$ is not prime, so by Lemma \ref{lem:notPrimeIsComposite} $n$ is composite.
  Thus, there exists $r,s\in\ZZ$ such that $1<r,s<n$.
  Though, as $1<r,s<n$ our inductive hypothesis applies to $r$ and $s$, so $r$ and $s$ are a product of primes.
  Whence, $n=rs$ is a product of primes.
\end{proof}


The proof of the next theorem requires the use of the fact that $\NN$ is well-ordered by $<$.
This means two things.
First, $\NN$ is totally ordered by $<$, that is $\forall a,b\in\NN$ $a<b\vee a=b\vee b<a$.
Second, every nonempty subset of $\NN$ has a $<$ least element.
\guard

\guard

\begin{defn}
\label{defn:prime}
\index{prime}
  An integer $1<p$ is said to be \emph{prime} provided for all positive integers $a$ and $b$ if $ab=p$, then $a=p$ or $b=p$.
\end{defn}

\guard

%TODO define integer
%TODO define integer multiplication

\begin{defn}
\label{defn:divide}
\index{divide}
  An integer $d\not=0$ is said to \emph{divide} an integer $n$ provided, written $d\vert n$, provided there exists an integer $k$ such that $n=dk$.
\end{defn}


\guard

\guard

\begin{defn}
\label{defn:prime}
\index{prime}
  An integer $1<p$ is said to be \emph{prime} provided for all positive integers $a$ and $b$ if $ab=p$, then $a=p$ or $b=p$.
\end{defn}

\guard

\input{numberTheory/defns/prime.tex}

\begin{defn}
\label{defn:composite}
\index{prime}
  An integer $1<c$ is said to be \emph{composite} if $c$ there exists integers $a$ and $b$ such that $c=ab$ and $1<a,b<c$
\end{defn}

\guard

\input{numberTheory/defns/prime.tex}
\input{numberTheory/defns/composite.tex}

\begin{lem}
\label{lem:notPrimeIsComposite}
  For every integer $n>1$, $n$ is prime if, and only if, $n$ is not composite.
\end{lem}
\begin{proof}
  Fix any integer $n>1$.
  Suppose $n$ is prime.
  By the definition of $n$ prime, the only factors of $n$ are $1$ and $n$.
  Thus, $\not\exists r,s\in\ZZ(n=rs\wedge 1<r,s<n)$.
  Showing that $n$ is not composite.

  Suppose that $n$ is not prime.
  As $n$ is not prime, there exists $r,s\in\ZZ$ such that $n=rs$ and neither $r$ nor $s$ are $1$ or $n$.
  Note, as $n$ is positive, $r$ and $s$ share the same sign.
  With out loss of generality, we may assume $r$ and $s$ are positive as otherwise we may replace them with $-r$ and $-s$.
  Finally, as $rs=n$ and multiplication of positive integers is non-decreasing, $r,s\leq n$.
  As neither $r$ nor $s$ is $1$ and $r,s$ is positive we have that $1<r,s$ and $r,s\leq n$.
  Thus, by Definition \ref{defn:composite}, $n$ is composite.
\end{proof}


\begin{lem}
\label{lem:positiveNotUnitIntegerPrimeOrProductOfPrimes}
  Every integer $n>1$ is either prime or a product of primes.
\end{lem}
\begin{proof}
  We show this via strong induction on $n$.
  For the base case, where $n=2$, we note that $2$ is prime.
  Thus, the claim holds at $n=2$.

  Now, suppose that $n>1$ and every integer $m>n$ is either prime or a product of primes.
  If $n$ is prime, there is nothing to show.
  Otherwise $n$ is not prime, so by Lemma \ref{lem:notPrimeIsComposite} $n$ is composite.
  Thus, there exists $r,s\in\ZZ$ such that $1<r,s<n$.
  Though, as $1<r,s<n$ our inductive hypothesis applies to $r$ and $s$, so $r$ and $s$ are a product of primes.
  Whence, $n=rs$ is a product of primes.
\end{proof}


\guard

%TODO define integer
%TODO define integer multiplication

\begin{defn}
\label{defn:divide}
\index{divide}
  An integer $d\not=0$ is said to \emph{divide} an integer $n$ provided, written $d\vert n$, provided there exists an integer $k$ such that $n=dk$.
\end{defn}


\begin{thm}
\label{thm:fundementalTheoremOfArithmetic}
  \textbf{(Fundemental Theorem Of Arithmetic)}
  Given any integer $n>1$, there exists a positive integer $k$, distinct primes $p_1,p_2,\dots,p_k$ and positive integers $a_1,a_2,\dots,a_k$ such that \[ n=p_1^{a_1}p_2^{a_2}\cdots p_k^{a_k}\] and any other expression is identical up to order.
\end{thm}
\begin{proof}
  First, we will show existence of such a factorization.
  Let $n>1$ be an arbitrary integer.
  If $n$ is prime, we are done.
  Otherwise, by Lemma \ref{lem:positiveNotUnitIntegerPrimeOrProductOfPrimes}, $n$ is a product of primes.
  Say $n = q_1q_2\cdots q_{k'}$ for some $k'\in\NN$.
  Certainly, it need not be the case that the $k_i$s are unique.
  Let $k\leq k'$ be the number of unique primes within $q_1,q_2,\dots q_{k'}$.
  Enumerate the unique primes within $q_1,q_2,\dots q_{k'}$ as $p_1,p_2,\dots,p_k$.
  Finally, for each $i$ let $a_i$ be the number of times $p_i$ occurs within  $q_1,q_2,\dots q_{k'}$.
  Then,
  \begin{align*}
    n &= q_1q_2\cdots q_{k'} \\
      &= n=p_1^{a_1}p_2^{a_2}\cdots p_k^{a_k}
  \end{align*}
  as desired.
  Though, as $1<n\in\ZZ$ was arbitrary, such a factorization exists for all integers larger than $1$.


  Now, we show uniqueness.
  Suppose, towards a contradiction that there exists an integer which can be factored as above in two distinct ways.
  As there exists such an integer, as $\NN$ is well-ordered by $<$ we may let $n$ be the least $n>1$ with two distinct prime factorizations as above.
  Note, this implies $n$ is not prime.
  Say \[n=p_1^{a_1}p_2^{a_2}\cdots p_k^{a_k} =q_1^{b_1}q_2^{b_2}\cdots q_l^{b_l}\,.\]
  Note, as $n$ is not prime $k,l\geq 2$.
  As $n=p_1^{a_1}p_2^{a_2}\cdots p_k^{a_k}$, $p_1$ divides $n$.
  So, $n$ divides $q_1^{b_1}q_2^{b_2}\cdots q_l^{b_l}$, and so there exists some $i<l$ so that $p_1\vert q_i$.
  Without loss of generalit, we may reorder the $q_i$s so that $p_1\vert q_1$.
  As $q_1$ is prime and $p_1\vert q_1$, $p_1=q_1$.
  Thus \[ p_1^{a_1-1}p_2^{a_2}\cdots p_k^{a_k} = q_1^{b_1-1}q_2^{b_2}\cdots q_l^{b_l}\] are distinct prime factorizations for an integer below $n$. contradicting the minimality of $n$.
  Thus, no such $n$ may exist.
\end{proof}


% \guard

\guard

%TODO define integer
%TODO define integer multiplication

\begin{defn}
\label{defn:divide}
\index{divide}
  An integer $d\not=0$ is said to \emph{divide} an integer $n$ provided, written $d\vert n$, provided there exists an integer $k$ such that $n=dk$.
\end{defn}


\begin{thm}
\label{thm:quotientRemainderTheorem}
  \textbf{(Quotient-Remainder Theorem)}
  Given an integer $n$ and positive integer $d$, there exists unique integers $q$ and $r$ such that $0\leq r < d$ and $n=qd + r$.
\end{thm}
\begin{proof}
  Fix $n\in\ZZ$ and $d\in\ZZ^+$.

  First, we will show existence.
  Consider the set of non-negative integers of the form $n-dq$.
  This set is clearly non-empty.
  Thus, as $\NN$ is well-ordered, this set contains a least element, say $r=n-dq$.
  Certianly, $r\geq 0$.
  Now, $r<d$ as as otherwise $n-d(q+1)$ contradicts the minimality of $r$.

  Now to show uniqueness.
  Suppose that there exists $q_0,q_1,r_0,r_1\in\ZZ$ with $0\leq r_0,r_1 <d$ and \[n=dq_0+r_0=dq_1+r_1\,.\]
  Note, as $dq_0+r_0=dq_1+r_1$, it suffices to show $r_0=r_1$.
  Rewriting $dq_0+r_0=dq_1+r_1$, we see that $r_0-r_1=d(q_1-q_0)$ showing that $d|(r_0-r_1)$.
  Further, as $0\leq r_1 <d$, $-d<r_1\leq 0$.
  So, $-d < r_0-r_1 <d$.
  Though, as $d|(r_0-r_1)$ and $-d < r_0-r_1 <d$, $r_0-r_1=0$.
\end{proof}

