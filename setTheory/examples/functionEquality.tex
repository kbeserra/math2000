\guard

\guard

\guard

\guard

%TODO define integer
%TODO define integer multiplication

\begin{defn}
\label{defn:divide}
\index{divide}
  An integer $d\not=0$ is said to \emph{divide} an integer $n$ provided, written $d\vert n$, provided there exists an integer $k$ such that $n=dk$.
\end{defn}


\begin{thm}
\label{thm:quotientRemainderTheorem}
  \textbf{(Quotient-Remainder Theorem)}
  Given an integer $n$ and positive integer $d$, there exists unique integers $q$ and $r$ such that $0\leq r < d$ and $n=qd + r$.
\end{thm}
\begin{proof}
  Fix $n\in\ZZ$ and $d\in\ZZ^+$.

  First, we will show existence.
  Consider the set of non-negative integers of the form $n-dq$.
  This set is clearly non-empty.
  Thus, as $\NN$ is well-ordered, this set contains a least element, say $r=n-dq$.
  Certianly, $r\geq 0$.
  Now, $r<d$ as as otherwise $n-d(q+1)$ contradicts the minimality of $r$.

  Now to show uniqueness.
  Suppose that there exists $q_0,q_1,r_0,r_1\in\ZZ$ with $0\leq r_0,r_1 <d$ and \[n=dq_0+r_0=dq_1+r_1\,.\]
  Note, as $dq_0+r_0=dq_1+r_1$, it suffices to show $r_0=r_1$.
  Rewriting $dq_0+r_0=dq_1+r_1$, we see that $r_0-r_1=d(q_1-q_0)$ showing that $d|(r_0-r_1)$.
  Further, as $0\leq r_1 <d$, $-d<r_1\leq 0$.
  So, $-d < r_0-r_1 <d$.
  Though, as $d|(r_0-r_1)$ and $-d < r_0-r_1 <d$, $r_0-r_1=0$.
\end{proof}


\begin{defn}
\label{defn:mod}
\label{defn:div}
\index{mod}
\index{div}
  Let $n$ be an integer and $d$ a positive integer.
  Then the Quotient Remainder Theorem, Theorem \ref{thm:quotientRemainderTheorem}, applies, and there exists unique integers $q$ and $r$ such that $0\leq r<d$ and $n=qd+r$.
  We define \[n\bmod d = r\text~~{ and }~~n\bdiv d = q\] to denote the integer quotient and remainder after division by $d$ respectively.
\end{defn}

\guard

\guard

\guard

\input{setTheory/defns/tuple.tex}

\begin{defn}
\label{defn:cartesianProduct}
\index{cartesian product}
  Let $A$ and $B$ be given.
  The \emph{caresian product} of $A$ and $B$, written $A\times B$, is the set of all ordered pairs $(a,b)$ where $a\in A$ and $b\in B$.
  \[ A\times B := \set{ (a,b)\mid a\in A,b\in B}\,.\]
\end{defn}


\begin{defn}
\label{defn:function}
\index{function}
\index{domain}
\index{codomain}
  Let $X$ and $Y$ be sets.
  We say that $f\subseteq X\times Y$ is a \emph{function} from a set $X$ to a set $Y$, denoted $f:X\to Y$, is a relation from $X$ to $Y$ such that every element of $x$ is related to exactly one element in $Y$.
  For $x\in X$ and $y\in Y$, $y$ is the unique value related to $x$ if, and only if, $(x,y)\in f$.
  In this case, we say that $f(x)=y$ or $f:x\mapsto y$.
\end{defn}

\guard

\guard

\input{setTheory/defns/cartesianProduct.tex}

\begin{defn}
\label{defn:function}
\index{function}
\index{domain}
\index{codomain}
  Let $X$ and $Y$ be sets.
  We say that $f\subseteq X\times Y$ is a \emph{function} from a set $X$ to a set $Y$, denoted $f:X\to Y$, is a relation from $X$ to $Y$ such that every element of $x$ is related to exactly one element in $Y$.
  For $x\in X$ and $y\in Y$, $y$ is the unique value related to $x$ if, and only if, $(x,y)\in f$.
  In this case, we say that $f(x)=y$ or $f:x\mapsto y$.
\end{defn}


\begin{defn}
\label{defn:functionDomain}
\index{domain}
  For a given function $f:X\to Y$, we call $X$ the \emph{domain} of $f$, written $\dom(f)$.
\end{defn}

\guard

\guard

\input{setTheory/defns/cartesianProduct.tex}

\begin{defn}
\label{defn:function}
\index{function}
\index{domain}
\index{codomain}
  Let $X$ and $Y$ be sets.
  We say that $f\subseteq X\times Y$ is a \emph{function} from a set $X$ to a set $Y$, denoted $f:X\to Y$, is a relation from $X$ to $Y$ such that every element of $x$ is related to exactly one element in $Y$.
  For $x\in X$ and $y\in Y$, $y$ is the unique value related to $x$ if, and only if, $(x,y)\in f$.
  In this case, we say that $f(x)=y$ or $f:x\mapsto y$.
\end{defn}


\begin{defn}
\label{defn:functionRange}
\index{range}
\index{codomain}
  For a given function $f:X\to Y$, we call the set \[ \set{ y\in Y \mid \exists x\in X( f(x) = y)}\] the \emph{range} of $f$, and denote this set $\ran(f)$.
  We call the set $Y$ to \emph{codomain} of $f$.
\end{defn}


\begin{prop}
\label{prop:functionEquality}
  Given functions $f,g:X\to Y$, we say that $f=g$ if, and only if, $\forall x\in X(f(x)=g(x))$.
\end{prop}
\begin{proof}
  Let$f,g:X \to Y$ be arbitrary functions.
  As $f$ and $g$ share domain and co-domain, $f,g\subseteq X\times Y$.

  Suppose $\forall x\in X(f(x) = g(x))$.
  Then for any $(x,y)\in X\times Y$,
  \begin{align*}
    (x,y)\in f  &\iff f(x) = y \\
                &\iff g(x) = y \\
                &\iff (x,y)\in g\,.
  \end{align*}
  Thus, $f=g$ as sets.

  For the converse, suppose $f=g$ as sets.
  Fix any $x\in X$.
  Then, for any $y\in Y$
  \begin{align*}
    f(x)=y  &\iff (x,y)\in f \\
            &\iff (x,y)\in g \\
            &\iff g(x) = y\,.
  \end{align*}
  Thus $f(x)=y$ if, and only if, $g(x) = y$.

  Though, for any $x\in X$, there is a unique $y$ such that $f(x)=y$.
  Thus $\forall x\in X( f(x) = g(x) )$.
\end{proof}


\begin{exmp}
\label{exmp:functionEquality}
  Set $X = {0,1,2}$ and defin $f,g:X\to Y$ as \[ f(x) = (x^2+x+1)\mod 3\,,\] and \[ g(x) = (x+1)^2 \mod 3\,.\]
  Does $f=g$?

  As
  \begin{align*}
    f(0)  &= 1  &&= g(0) \\
    f(1)  &= 0  &&= g(1) \\
    f(2)  &= 1  &&= g(2)
  \end{align*}
  Proposition \ref{prop:functionEquality} gives that $f=g$.


  For any function $a,b:\RR\to\RR$, define $(a+b):\RR\to\RR$ as $\forall x\in RR$ $(a+b)(x) := a(x) + b(x)$.
  Let $f,g:\RR\to\RR$ be arbitrary.
  Does $f+g = g+f$.

  Again, yes.
  As $\forall x\in \RR$
  \begin{align*}
    (f+g)(x)  &= f(x) + g(x) \\
              &= g(x) + f(x) \\
              &= (g+f)(x)\,.
  \end{align*}
  So, using the same proposition as before, we have that $f=g$.
\end{exmp}
