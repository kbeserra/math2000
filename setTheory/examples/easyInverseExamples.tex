\guard

\guard

\guard

\guard

\input{setTheory/defns/function.tex}

\begin{defn}
\label{defn:injective}
\index{injective}
  A function $f:X\to Y$ is said to be \emph{injective} (or \emph{one-to-one}) provided $\forall x,y\in X$ if $x\not=y$, then $f(x)\not= f(y)$.
\end{defn}


\begin{exmp}
\label{exmp:easyInjectiveExamples}
  \begin{enumerate}
    \item Show that $f:\RR\to\RR$ defined as $f(x) = 4x-1$ is injective.\\
    Like many proofs of infectivity, we will show the contrapositive of Definition \ref{defn:injective}.
    Fix $x,y\in\RR$ and suppose that $f(x)=f(y)$.
    Then,
    \begin{alignat*}{2}
      && f(x)&=f(y) \\
      &\Rightarrow\quad& 4x-1 &= 4y-1 \\
      &\Rightarrow\quad& 4x &= 4y \\
      &\Rightarrow\quad& x &= y \,.\\
    \end{alignat*}
    Thus, $\forall x,y\in\RR(f(x)=f(y)\rightarrow x=y)$.

    \item Show that $g:\RR\to\RR$ defined by $g(x)=x^2$ is not injective.\\
    To do this, we need only show that there exists distinct $x,y\in\RR$ such that $g(x)=g(y)$.
    Note that $-1,1\in\RR$ and $-1\not= 1$, but $g(1)=g(-1)$.

    Note, we can change the domain of $g$ to make the function injective.
    How?
  \end{enumerate}

\end{exmp}

\guard

\guard

\input{setTheory/defns/function.tex}

\begin{defn}
\label{defn:surjective}
\index{surjective}
  A function $f:X\to Y$ is said to be \emph{surjective} (or \emph{onto}) provided $\forall y\in Y\exists x\in X$ $f(x)=y$.
\end{defn}


\begin{exmp}
\label{exmp:easySurjectiveExamples}
  \begin{enumerate}
    \item Show that $f:\RR\to\RR$ defined as $f(x) = 4x-1$ is surjective.\\
    Fix any $y\in \RR$.
    Our job is to find some $x\in\RR$ such that $f(x)=y$.
    Given our selection of $y$, set $x = \frac{y+1}{4}$.
    We claim that this $x$ works,
    \begin{align*}
      f(x)  &=  4x-1 \\
            &=  4\frac{y+1}{4} - 1 \\
            &=  y+1 - 1 \\
            &=  y\,.
    \end{align*}
    Thus, for each $y\in\RR$ we can find some $x\in\RR$ such that $f(x)=y$.
    Thus, $f$ is surjective.

    \item Show that $g:\RR\to\RR$ defined by $g(x)=x^2$ is not surjective.\\
    To do this, we need only produce some $y\in\RR$ such that there does not exist a $x\in\RR$ such that $g(x)=y$.
    Consider $y=-1$.
    Note, $\forall x\in\RR$ $x^2\geq 0$, thus there does not exist an $x\in\RR$ such that $g(x)=-1$.
    Thus, $g$ is not surjective.

    Note, we can change the codomain of $g$ to make the function surjective.
    How? 
  \end{enumerate}

\end{exmp}


\guard

\input{setTheory/defns/injection.tex}
\input{setTheory/defns/surjection.tex}

\begin{defn}
\label{defn:bijection}
\index{bijection}
  A function $f:X\to Y$ is said to be \emph{bijection} provided $f$ is injective and surjective.
\end{defn}


\begin{exmp}
\label{exmp:easyBijectionExamples}
  \begin{enumerate}
    \item Show that $f(x) = 4x-1$ is a bijection.\\

    In Examples \ref{exmp:easyInjectiveExamples} and \ref{exmp:easySurjectiveExamples} respectively, we showed that $f$ is injective and surjective.
    So, as $f$ is both an injection and a surjection, $f$ is a bijection.

    \item Show that $f:\RR^2\to\RR^2$ defined by \[ g(x,y) = (x+y,x-y)\] is a bijection.\\

    First, we show that $g$ is injective.
    Fix $x,y,a,b\in\RR$ such that $g(x,y) = g(a,b)$.
    Then by definiton of $g$, we have that
    \begin{alignat*}{2}
      && g(x,y)&=g(a,b) \\
      &\Rightarrow\quad& (x+y,x-y) &= (a+b,a-b) \\
      &\Rightarrow\quad& (x+y = a+b)\wedge (x-y=a-b)\,.
    \end{alignat*}
    Adding these two equalities gives that $ 2x = 2a$.
    So, $x=a$.
    Now, as $x=a$ and $x+y = a+b$, we have that $y=b$.
    Thus, $(x,y)=(a,b)$.
    So $g$ is injective.

    Now, to show that $g$ is surjective.
    Fix $u,v\in\RR$.
    Let $x = \frac{u+v}{2}$ and $y=\frac{u-v}{2}$.
    Then, we can note that
    \begin{align*}
      g(x,y)  &= \left( \frac{u+v}{2}+\frac{u-v}{2}, \frac{u+v}{2}-\frac{u-v}{2} \right)\\
              &= \left( \frac{u+v+u-v}{2}, \frac{u+v-u+v}{2} \right)\\
              &= \left( \frac{2u}{2}, \frac{2v}{2} \right)\\
              &= \left( u, v \right)
    \end{align*}
    So, $g$ is injective.
  \end{enumerate}

\end{exmp}

\guard

\guard

\guard

\input{setTheory/defns/injection.tex}
\input{setTheory/defns/surjection.tex}

\begin{defn}
\label{defn:bijection}
\index{bijection}
  A function $f:X\to Y$ is said to be \emph{bijection} provided $f$ is injective and surjective.
\end{defn}


\begin{prop}
\label{prop:existenceOfInverseFunction}
  Let $f:X\to Y$ be a bijection.
  Then, there exists a unique function $f^{-1}:Y\to X$ defined by $\forall y\in Y$ $f^{-1}(y)=x$ where $x$ is the unique element of $X$ such that $f(x)=y$.
  That is, $\forall x\in X\forall y\in Y$ \[ f^{-1}(y)=x\quad\iff\quad f(x)=y\,.\]
\end{prop}
\begin{proof}

\end{proof}


\begin{defn}
\label{defn:bijection}
\index{bijection}
  $f^{-1}$ in Proposition \ref{prop:existenceOfInverseFunction} is called the \emph{inverse} of $f$.
\end{defn}


\begin{exmp}
\label{exmp:easyBijectionExamples}
  \begin{enumerate}
    \item Compute $f^{-1}$ given $f(x) = 4x-1$.\\

    First, we performa sanity check and ask if $f^{-1}$ makes sense given our function $f$.
    As we showed in Example \ref{exmp:easyBijectionExamples}, $f$ as above is a bijection.
    So, this by Proposition \ref{prop:existenceOfInverseFunction} $f^{-1}$ does exists.
    In fact, we have computed the inverse already, in Example \ref{exmp:easySurjectiveExamples}, it is \[ f^{-1}(y) = \frac{y+1}{4}\,.\]

    \item Compute the inverse of the function $g(x,y) = (x+y,x-y)$.\\

    Again, we have already shown that this $g$ is a bijection.
    Further, in Example \ref{exmp:easyBijectionExamples}, while showing that $g$ was a surjection, we nearly computed the inverse of $g$.
    Putting together the work we did there, it is not difficult to see that \[ g^{-1}(u,v) = \left( \frac{ u+v}{2}, \frac{u-v}{2} \right)\,.\]

  \end{enumerate}

\end{exmp}
