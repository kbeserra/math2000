\guard

\guard

\guard

\begin{defn}
\label{defn:set}
\index{set}
  A \emph{set} is a collection of things.
\label{defn:membership}
\index{$\in$}
  The symbol $\in$ is used denote the \emph{membership} relation on sets.
  For $S$ a set and $x$ arbitrary, we say $x\in S$ precisely when $x$ is a member (or element) of $S$.
\end{defn}


\begin{defn}
\label{defn:subset}
\index{subset}
  The \emph{subset} relation, $\subseteq$, is defined as \[ x\subseteq y\iff\forall z(z\in x\rightarrow z\in y)\,.\]
  We use $x\subset y$, or $x\subsetneq y$, to abbreviate $x\subseteq y\wedge x\not=y$.
\index{subset!proper}
  If $x\subsetneq y$, we say that $x$ is a \emph{proper subset} of $y$.
\end{defn}


\begin{exmp}
\label{exmp:earlySubsetExample}
  Set \[ A := \set{ n\in\ZZ \mid \exists r\in\ZZ( n = 6r )} \] and \[ B := \set{ n\in\ZZ\mid \exists s\in\ZZ( n=3s )}\,. \]
  Note \[ A = \set{ \dots, -18,-12,-6,0,6,12,18,\dots} \] and \[ B = \set{ \dots,-9,-6,-3,0,3,6,9,\dots }\,.\]

  First, let us show that $A\subseteq B$.
  That is, $\forall x( x\in A \rightarrow x\in B)$.
  Let $x$ be arbitrary, and suppose that $x\in A$.
  From the definition of $A$, $x\in A$ means that $\exists r\in\ZZ( x= 6r)$.
  Let $r\in\ZZ$ be such that $x=6r$.
  Set $s=2r$, and observe that
  \begin{align*}
    x &= 6r \\
      &= 3\cdot2 r\\
      &= 3s\,.
  \end{align*}
  So, $x=3s$ for some $s$, showing that $x\in B$.
  Finally, as $x$ was arbitrary in $A$, we have that $A\subseteq B$.


  Now, the reverse is not true.
  That is, $B\not\subseteq A$.
  This is becuase there is some element of $B$ which is not an element of $A$.
  One such element is $3$.
  Note, as $3=3\cdot 1$, $3\in B$.
  Though, as there does not exist a $r\in\ZZ$ such that $3=6r$, $3\not\in A$.
  So, $B\not\subseteq A$.

  Thus, $A$ is a proper subset of $B$. 
\end{exmp}
