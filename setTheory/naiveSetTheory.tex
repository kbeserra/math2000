\guard
\section{Naive Set Theory}
\label{sec:naiveSetTheory}

\guard

\begin{defn}
\label{defn:set}
\index{set}
  A \emph{set} is a collection of things.
\label{defn:membership}
\index{$\in$}
  The symbol $\in$ is used denote the \emph{membership} relation on sets.
  For $S$ a set and $x$ arbitrary, we say $x\in S$ precisely when $x$ is a member (or element) of $S$.
\end{defn}

\guard

\guard

\begin{defn}
\label{defn:set}
\index{set}
  A \emph{set} is a collection of things.
\label{defn:membership}
\index{$\in$}
  The symbol $\in$ is used denote the \emph{membership} relation on sets.
  For $S$ a set and $x$ arbitrary, we say $x\in S$ precisely when $x$ is a member (or element) of $S$.
\end{defn}


In this section, and in those that follow, everything is a set.


\guard

\guard

\guard

\input{setTheory/defns/set.tex}

In this section, and in those that follow, everything is a set.


\begin{defn}
\label{defn:extensionality}
\index{Axiom of Extensionality}
  \[ \forall x\forall y(\forall z(z\in x\leftrightarrow z\in y)\rightarrow x=y) \]
  If $A$ and $B$ have the same elements, then $A=B$.
\end{defn}


\begin{exmp}
\label{exmp:setEquality}
  Let $A=\set{1}$, $B=\set{1,2}$, and $C=\set{1,1}$.\\

  As $2\in B$ but $2\not\in A$, $A\not= B$.\\

  Clearly, as every element of $A$ is an element of $C$.
  The reverse is also true; that is, every element of $C$ is an element of $B$.
  Thus, $A=C$.
\end{exmp}


\guard

\guard

\guard

\input{logic/defns/statement.tex}
\input{logic/defns/truthValue.tex}


\begin{defn}
\label{defn:conjunctionOfStatement}
  Let $p$ and $q$ be statements.
  The \emph{conjunction} of $p$ and $q$, written $p \wedge q$, is the statement that is true precisely when both $p$ and $q$ are true and is otherwise false.
\end{defn}

\guard

\input{logic/defns/statement.tex}
\input{logic/defns/statementForm.tex}
\input{logic/defns/truthValue.tex}


\begin{defn}
\label{defn:conditionalStatement}
\index{conditional}
  Let $p$ and $q$ be statements forms.
  The \emph{conditional statement} ``$p$ implies $q$'', written $p \rightarrow q$, is the statement form that is false precisely when $p$ is true and $q$ is false ( that is, when the statement ``If $p$, then $q$'' is violated ).

  In the conditional $p\rightarrow q$, $p$ is refered to as the \emph{hypothesis} and $q$ is called the \emph{conclusion}.
\end{defn}

\guard

\begin{defn}
\label{defn:set}
\index{set}
  A \emph{set} is a collection of things.
\label{defn:membership}
\index{$\in$}
  The symbol $\in$ is used denote the \emph{membership} relation on sets.
  For $S$ a set and $x$ arbitrary, we say $x\in S$ precisely when $x$ is a member (or element) of $S$.
\end{defn}


\begin{defn}
\label{defn:subset}
\index{subset}
\index{$\subseteq$}
  The \emph{subset} relation, $\subseteq$, is defined as \[ x\subseteq y\iff\forall z(z\in x\rightarrow z\in y)\,.\]
\index{subset!proper}
\index{$\subset$}
\index{$\subsetneq$}
  We use $x\subset y$, or $x\subsetneq y$, to abbreviate $x\subseteq y\wedge x\not=y$.
  If $x\subsetneq y$, we say that $x$ is a \emph{proper subset} of $y$.
\index{super set}
\index{$\supseteq$}
  We use $A\supseteq B$ to mean $B\subseteq A$, and is read $A$ is a super set of $B$.
  The $\supset$ and $\supsetneq$ variations are similar to the subset variations.
\end{defn}


\begin{exmp}
\label{exmp:earlySubsetExample}
  Set \[ A := \set{ n\in\ZZ \mid \exists r\in\ZZ( n = 6r )} \] and \[ B := \set{ n\in\ZZ\mid \exists s\in\ZZ( n=3s )}\,. \]
  Note \[ A = \set{ \dots, -18,-12,-6,0,6,12,18,\dots} \] and \[ B = \set{ \dots,-9,-6,-3,0,3,6,9,\dots }\,.\]

  First, let us show that $A\subseteq B$.
  That is, $\forall x( x\in A \rightarrow x\in B)$.
  Let $x$ be arbitrary, and suppose that $x\in A$.
  From the definition of $A$, $x\in A$ means that $\exists r\in\ZZ( x= 6r)$.
  Let $r\in\ZZ$ be such that $x=6r$.
  Set $s=2r$, and observe that
  \begin{align*}
    x &= 6r \\
      &= 3\cdot2 r\\
      &= 3s\,.
  \end{align*}
  So, $x=3s$ for some $s$, showing that $x\in B$.
  Finally, as $x$ was arbitrary in $A$, we have that $A\subseteq B$.


  Now, the reverse is not true.
  That is, $B\not\subseteq A$.
  This is becuase there is some element of $B$ which is not an element of $A$.
  One such element is $3$.
  Note, as $3=3\cdot 1$, $3\in B$.
  Though, as there does not exist a $r\in\ZZ$ such that $3=6r$, $3\not\in A$.
  So, $B\not\subseteq A$.

  Thus, $A$ is a proper subset of $B$. 
\end{exmp}


\guard

\guard

\guard

\begin{defn}
\label{defn:set}
\index{set}
  A \emph{set} is a collection of things.
\label{defn:membership}
\index{$\in$}
  The symbol $\in$ is used denote the \emph{membership} relation on sets.
  For $S$ a set and $x$ arbitrary, we say $x\in S$ precisely when $x$ is a member (or element) of $S$.
\end{defn}


\begin{defn}
\label{defn:emptyset}
\index{emptyset}
\index{$\emptyset$}
  The set which contains nothing is called the \emph{emptyset}.
  We use $\emptyset$ to denote the emptyset.
\end{defn}

\guard

\guard

\input{logic/defns/statement.tex}
\input{logic/defns/truthValue.tex}


\begin{defn}
\label{defn:conjunctionOfStatement}
  Let $p$ and $q$ be statements.
  The \emph{conjunction} of $p$ and $q$, written $p \wedge q$, is the statement that is true precisely when both $p$ and $q$ are true and is otherwise false.
\end{defn}

\guard

\input{logic/defns/statement.tex}
\input{logic/defns/statementForm.tex}
\input{logic/defns/truthValue.tex}


\begin{defn}
\label{defn:conditionalStatement}
\index{conditional}
  Let $p$ and $q$ be statements forms.
  The \emph{conditional statement} ``$p$ implies $q$'', written $p \rightarrow q$, is the statement form that is false precisely when $p$ is true and $q$ is false ( that is, when the statement ``If $p$, then $q$'' is violated ).

  In the conditional $p\rightarrow q$, $p$ is refered to as the \emph{hypothesis} and $q$ is called the \emph{conclusion}.
\end{defn}

\guard

\begin{defn}
\label{defn:set}
\index{set}
  A \emph{set} is a collection of things.
\label{defn:membership}
\index{$\in$}
  The symbol $\in$ is used denote the \emph{membership} relation on sets.
  For $S$ a set and $x$ arbitrary, we say $x\in S$ precisely when $x$ is a member (or element) of $S$.
\end{defn}


\begin{defn}
\label{defn:subset}
\index{subset}
\index{$\subseteq$}
  The \emph{subset} relation, $\subseteq$, is defined as \[ x\subseteq y\iff\forall z(z\in x\rightarrow z\in y)\,.\]
\index{subset!proper}
\index{$\subset$}
\index{$\subsetneq$}
  We use $x\subset y$, or $x\subsetneq y$, to abbreviate $x\subseteq y\wedge x\not=y$.
  If $x\subsetneq y$, we say that $x$ is a \emph{proper subset} of $y$.
\index{super set}
\index{$\supseteq$}
  We use $A\supseteq B$ to mean $B\subseteq A$, and is read $A$ is a super set of $B$.
  The $\supset$ and $\supsetneq$ variations are similar to the subset variations.
\end{defn}


\begin{prop}
\label{prop:emptysetIsSubsetOfEverySet}
  For any $A$, $\emptyset\subseteq A$.
\end{prop}
\begin{proof}
  This holds trivially.
\end{proof}


Note, to show that two sets $A$ and $B$ are equal, it suffices to show that $A\subseteq B$ and $B\subseteq A$.
\guard

\guard

\guard

\input{setTheory/defns/set.tex}

In this section, and in those that follow, everything is a set.


\begin{defn}
\label{defn:extensionality}
\index{Axiom of Extensionality}
  \[ \forall x\forall y(\forall z(z\in x\leftrightarrow z\in y)\rightarrow x=y) \]
  If $A$ and $B$ have the same elements, then $A=B$.
\end{defn}

\guard

\guard

\input{logic/defns/statement.tex}
\input{logic/defns/truthValue.tex}


\begin{defn}
\label{defn:conjunctionOfStatement}
  Let $p$ and $q$ be statements.
  The \emph{conjunction} of $p$ and $q$, written $p \wedge q$, is the statement that is true precisely when both $p$ and $q$ are true and is otherwise false.
\end{defn}

\guard

\input{logic/defns/statement.tex}
\input{logic/defns/statementForm.tex}
\input{logic/defns/truthValue.tex}


\begin{defn}
\label{defn:conditionalStatement}
\index{conditional}
  Let $p$ and $q$ be statements forms.
  The \emph{conditional statement} ``$p$ implies $q$'', written $p \rightarrow q$, is the statement form that is false precisely when $p$ is true and $q$ is false ( that is, when the statement ``If $p$, then $q$'' is violated ).

  In the conditional $p\rightarrow q$, $p$ is refered to as the \emph{hypothesis} and $q$ is called the \emph{conclusion}.
\end{defn}

\guard

\begin{defn}
\label{defn:set}
\index{set}
  A \emph{set} is a collection of things.
\label{defn:membership}
\index{$\in$}
  The symbol $\in$ is used denote the \emph{membership} relation on sets.
  For $S$ a set and $x$ arbitrary, we say $x\in S$ precisely when $x$ is a member (or element) of $S$.
\end{defn}


\begin{defn}
\label{defn:subset}
\index{subset}
\index{$\subseteq$}
  The \emph{subset} relation, $\subseteq$, is defined as \[ x\subseteq y\iff\forall z(z\in x\rightarrow z\in y)\,.\]
\index{subset!proper}
\index{$\subset$}
\index{$\subsetneq$}
  We use $x\subset y$, or $x\subsetneq y$, to abbreviate $x\subseteq y\wedge x\not=y$.
  If $x\subsetneq y$, we say that $x$ is a \emph{proper subset} of $y$.
\index{super set}
\index{$\supseteq$}
  We use $A\supseteq B$ to mean $B\subseteq A$, and is read $A$ is a super set of $B$.
  The $\supset$ and $\supsetneq$ variations are similar to the subset variations.
\end{defn}


\begin{exmp}
\label{exmp:setEqualityBySubsets}
  Let \[ A := \set{ 2a \mid a\in\ZZ } \] and \[ B := \set{ 2b-2\mid b\in\ZZ }\,. \]

  Fix $x\in A$.
  As $x\in A$, $x=2a$ for some $a\in\ZZ$.
  As $a\in\ZZ$, $a+1\in\ZZ$.
  Set $b = a+1$, and observe that
  \begin{align*}
    x &= 2a \\
      &= 2a + 2 - 2 \\
      &= 2(a+1) - 2 \\
      &= 2b - 2\,,
  \end{align*}
  showing that that $x\in B$.
  So, as $x$ was arbitrary, $A\subseteq B$.

  Now, fix $x\in B$.
  As $x\in B$, $x = 2b-2$ for some $b\in \ZZ$.
  Set $a = b-1$, and as $b\in\ZZ$ $a\in\ZZ$ as well.
  Finally,
  \begin{align*}
    x &= 2b-2 \\
      &= 2b - 2 + 2 - 2 \\
      &= 2(b-1) \\
      &= 2a\,,
  \end{align*}
  showing that $x\in A$.
  So, as $x$ was arbitrary, $B\subseteq A$.

  Thus, as $A\subseteq B$ and $B\subseteq A$, $A=B$.
\end{exmp}


\guard

\guard

\guard

\input{logic/defns/statement.tex}
\input{logic/defns/truthValue.tex}


\begin{defn}
\label{defn:conjunctionOfStatement}
  Let $p$ and $q$ be statements.
  The \emph{conjunction} of $p$ and $q$, written $p \wedge q$, is the statement that is true precisely when both $p$ and $q$ are true and is otherwise false.
\end{defn}

\guard

\input{logic/defns/statement.tex}
\input{logic/defns/statementForm.tex}
\input{logic/defns/truthValue.tex}


\begin{defn}
\label{defn:conditionalStatement}
\index{conditional}
  Let $p$ and $q$ be statements forms.
  The \emph{conditional statement} ``$p$ implies $q$'', written $p \rightarrow q$, is the statement form that is false precisely when $p$ is true and $q$ is false ( that is, when the statement ``If $p$, then $q$'' is violated ).

  In the conditional $p\rightarrow q$, $p$ is refered to as the \emph{hypothesis} and $q$ is called the \emph{conclusion}.
\end{defn}

\guard

\begin{defn}
\label{defn:set}
\index{set}
  A \emph{set} is a collection of things.
\label{defn:membership}
\index{$\in$}
  The symbol $\in$ is used denote the \emph{membership} relation on sets.
  For $S$ a set and $x$ arbitrary, we say $x\in S$ precisely when $x$ is a member (or element) of $S$.
\end{defn}


\begin{defn}
\label{defn:subset}
\index{subset}
\index{$\subseteq$}
  The \emph{subset} relation, $\subseteq$, is defined as \[ x\subseteq y\iff\forall z(z\in x\rightarrow z\in y)\,.\]
\index{subset!proper}
\index{$\subset$}
\index{$\subsetneq$}
  We use $x\subset y$, or $x\subsetneq y$, to abbreviate $x\subseteq y\wedge x\not=y$.
  If $x\subsetneq y$, we say that $x$ is a \emph{proper subset} of $y$.
\index{super set}
\index{$\supseteq$}
  We use $A\supseteq B$ to mean $B\subseteq A$, and is read $A$ is a super set of $B$.
  The $\supset$ and $\supsetneq$ variations are similar to the subset variations.
\end{defn}


\begin{prop}
\label{prop:subsetRelationsIsTransitive}
  For any $A$, $B$, and $C$.
  If $A\subseteq B$ and $B\subseteq C$, then $A\subseteq C$.
\end{prop}
\begin{proof}
  Fix $A$, $B$, and $C$ such that $A\subseteq B$ and $B\subseteq C$.
  Fix $a\in A$.
  As $A\subseteq B$, $a\in B$.
  As $B\subseteq C$, $a\in C$.
  But, $a\in A$ was arbitrary, so $A\subseteq C$.
\end{proof}





\guard

\guard

\guard

\begin{defn}
\label{defn:set}
\index{set}
  A \emph{set} is a collection of things.
\label{defn:membership}
\index{$\in$}
  The symbol $\in$ is used denote the \emph{membership} relation on sets.
  For $S$ a set and $x$ arbitrary, we say $x\in S$ precisely when $x$ is a member (or element) of $S$.
\end{defn}


\begin{defn}
\label{defn:union}
\index{union}
  Let $A$ and $B$ sets.
  The \emph{union} of $A$ and $B$, $A\cup B$ is the set which contains everything which is in $A$ or in $B$.
  \[ A\cup B := \set{ x \mid x\in A \vee x\in B}\,.\]
\end{defn}

\guard

\guard

\begin{defn}
\label{defn:set}
\index{set}
  A \emph{set} is a collection of things.
\label{defn:membership}
\index{$\in$}
  The symbol $\in$ is used denote the \emph{membership} relation on sets.
  For $S$ a set and $x$ arbitrary, we say $x\in S$ precisely when $x$ is a member (or element) of $S$.
\end{defn}


\begin{defn}
\label{defn:intersection}
\index{intersection}
  Let $A$ and $B$ sets.
  The \emph{intersection} of $A$ and $B$, $A\cap B$ is the set which contains everything which is in $A$ and in $B$.
  \[ A\cap B := \set{ x \mid x\in A \wedge x\in B}\,.\]
\end{defn}

\guard

\guard

\begin{defn}
\label{defn:set}
\index{set}
  A \emph{set} is a collection of things.
\label{defn:membership}
\index{$\in$}
  The symbol $\in$ is used denote the \emph{membership} relation on sets.
  For $S$ a set and $x$ arbitrary, we say $x\in S$ precisely when $x$ is a member (or element) of $S$.
\end{defn}


\begin{defn}
\label{defn:difference}
\index{set difference}
  Let $A$ and $B$ sets.
  The \emph{difference} of $A$ and $B$, $A\setminus B$ is the set which contains everything which is in $A$ and not in $B$.
  \[ A\setminus B := \set{ x \in A \mid x\not\in B}\,.\]
\end{defn}


\begin{exmp}
\label{exmp:setOperations}
  Let $A=\set{a,c,e,g}$ and $B=\set{d,e,f,g}$.\\
  Then
  \begin{align*}
    A\cup B &= \set{ a,c,d,e,g,f}\,, \\
    A\cap B &= \set{ e, g}\,, \\
    A\setminus B &= \set{ a, c, }\,, \\
    B\setminus A &= \set{ d, f }\,.
  \end{align*}
\end{exmp}


\guard

%TODO: these exercises.

\guard

\guard

\input{logic/defns/statement.tex}
\input{logic/defns/truthValue.tex}


\begin{defn}
\label{defn:conjunctionOfStatement}
  Let $p$ and $q$ be statements.
  The \emph{conjunction} of $p$ and $q$, written $p \wedge q$, is the statement that is true precisely when both $p$ and $q$ are true and is otherwise false.
\end{defn}

\guard

\input{logic/defns/statement.tex}
\input{logic/defns/statementForm.tex}
\input{logic/defns/truthValue.tex}


\begin{defn}
\label{defn:conditionalStatement}
\index{conditional}
  Let $p$ and $q$ be statements forms.
  The \emph{conditional statement} ``$p$ implies $q$'', written $p \rightarrow q$, is the statement form that is false precisely when $p$ is true and $q$ is false ( that is, when the statement ``If $p$, then $q$'' is violated ).

  In the conditional $p\rightarrow q$, $p$ is refered to as the \emph{hypothesis} and $q$ is called the \emph{conclusion}.
\end{defn}

\guard

\begin{defn}
\label{defn:set}
\index{set}
  A \emph{set} is a collection of things.
\label{defn:membership}
\index{$\in$}
  The symbol $\in$ is used denote the \emph{membership} relation on sets.
  For $S$ a set and $x$ arbitrary, we say $x\in S$ precisely when $x$ is a member (or element) of $S$.
\end{defn}


\begin{defn}
\label{defn:subset}
\index{subset}
\index{$\subseteq$}
  The \emph{subset} relation, $\subseteq$, is defined as \[ x\subseteq y\iff\forall z(z\in x\rightarrow z\in y)\,.\]
\index{subset!proper}
\index{$\subset$}
\index{$\subsetneq$}
  We use $x\subset y$, or $x\subsetneq y$, to abbreviate $x\subseteq y\wedge x\not=y$.
  If $x\subsetneq y$, we say that $x$ is a \emph{proper subset} of $y$.
\index{super set}
\index{$\supseteq$}
  We use $A\supseteq B$ to mean $B\subseteq A$, and is read $A$ is a super set of $B$.
  The $\supset$ and $\supsetneq$ variations are similar to the subset variations.
\end{defn}


\begin{exmp}
\label{exmp:setOperationsForSequencesOfSets}
  For each $i\in\ZZ^{+}$, set \[ A_i:=\set{ x\in\RR\mid -\frac{1}{i}<x<\frac{1}{i}} = (-\frac{1}{i},\frac{1}{i})\,.\]
  Compute:
  \begin{enumerate}
    \item $A_1 \cup A_2 \cup A_3 $

    \item $A_1 \cap A_2 \cap A_3 $

    \item $\bigcup_{i\in\ZZ^+} A_i$

    \item $\bigcap_{i\in\ZZ^+} A_i$
  \end{enumerate}
\end{exmp}


\guard

\guard

\guard

\input{logic/defns/statement.tex}
\input{logic/defns/truthValue.tex}


\begin{defn}
\label{defn:conjunctionOfStatement}
  Let $p$ and $q$ be statements.
  The \emph{conjunction} of $p$ and $q$, written $p \wedge q$, is the statement that is true precisely when both $p$ and $q$ are true and is otherwise false.
\end{defn}

\guard

\input{logic/defns/statement.tex}
\input{logic/defns/statementForm.tex}
\input{logic/defns/truthValue.tex}


\begin{defn}
\label{defn:conditionalStatement}
\index{conditional}
  Let $p$ and $q$ be statements forms.
  The \emph{conditional statement} ``$p$ implies $q$'', written $p \rightarrow q$, is the statement form that is false precisely when $p$ is true and $q$ is false ( that is, when the statement ``If $p$, then $q$'' is violated ).

  In the conditional $p\rightarrow q$, $p$ is refered to as the \emph{hypothesis} and $q$ is called the \emph{conclusion}.
\end{defn}

\guard

\begin{defn}
\label{defn:set}
\index{set}
  A \emph{set} is a collection of things.
\label{defn:membership}
\index{$\in$}
  The symbol $\in$ is used denote the \emph{membership} relation on sets.
  For $S$ a set and $x$ arbitrary, we say $x\in S$ precisely when $x$ is a member (or element) of $S$.
\end{defn}


\begin{defn}
\label{defn:subset}
\index{subset}
\index{$\subseteq$}
  The \emph{subset} relation, $\subseteq$, is defined as \[ x\subseteq y\iff\forall z(z\in x\rightarrow z\in y)\,.\]
\index{subset!proper}
\index{$\subset$}
\index{$\subsetneq$}
  We use $x\subset y$, or $x\subsetneq y$, to abbreviate $x\subseteq y\wedge x\not=y$.
  If $x\subsetneq y$, we say that $x$ is a \emph{proper subset} of $y$.
\index{super set}
\index{$\supseteq$}
  We use $A\supseteq B$ to mean $B\subseteq A$, and is read $A$ is a super set of $B$.
  The $\supset$ and $\supsetneq$ variations are similar to the subset variations.
\end{defn}

\guard

\guard

\begin{defn}
\label{defn:set}
\index{set}
  A \emph{set} is a collection of things.
\label{defn:membership}
\index{$\in$}
  The symbol $\in$ is used denote the \emph{membership} relation on sets.
  For $S$ a set and $x$ arbitrary, we say $x\in S$ precisely when $x$ is a member (or element) of $S$.
\end{defn}


\begin{defn}
\label{defn:union}
\index{union}
  Let $A$ and $B$ sets.
  The \emph{union} of $A$ and $B$, $A\cup B$ is the set which contains everything which is in $A$ or in $B$.
  \[ A\cup B := \set{ x \mid x\in A \vee x\in B}\,.\]
\end{defn}


\begin{prop}
\label{prop:unionIsAdditive}
  For any $A$ and $B$, $A\subseteq A\cup B$.
\end{prop}
\begin{proof}
  Let $A$ and $B$ be arbitrary.
  Fix any $x\in A$.
  As $x\in A$, $x\in A$ or $x\in B$.
  Thus, $x\in A\cup B$.
  But, $x$ was arbitrary, so $A\subseteq A\cup B$.
\end{proof}

\guard

\guard

\guard

\begin{defn}
\label{defn:set}
\index{set}
  A \emph{set} is a collection of things.
\label{defn:membership}
\index{$\in$}
  The symbol $\in$ is used denote the \emph{membership} relation on sets.
  For $S$ a set and $x$ arbitrary, we say $x\in S$ precisely when $x$ is a member (or element) of $S$.
\end{defn}


\begin{defn}
\label{defn:emptyset}
\index{emptyset}
\index{$\emptyset$}
  The set which contains nothing is called the \emph{emptyset}.
  We use $\emptyset$ to denote the emptyset.
\end{defn}

\guard

\guard

\input{logic/defns/statement.tex}
\input{logic/defns/truthValue.tex}


\begin{defn}
\label{defn:conjunctionOfStatement}
  Let $p$ and $q$ be statements.
  The \emph{conjunction} of $p$ and $q$, written $p \wedge q$, is the statement that is true precisely when both $p$ and $q$ are true and is otherwise false.
\end{defn}

\guard

\input{logic/defns/statement.tex}
\input{logic/defns/statementForm.tex}
\input{logic/defns/truthValue.tex}


\begin{defn}
\label{defn:conditionalStatement}
\index{conditional}
  Let $p$ and $q$ be statements forms.
  The \emph{conditional statement} ``$p$ implies $q$'', written $p \rightarrow q$, is the statement form that is false precisely when $p$ is true and $q$ is false ( that is, when the statement ``If $p$, then $q$'' is violated ).

  In the conditional $p\rightarrow q$, $p$ is refered to as the \emph{hypothesis} and $q$ is called the \emph{conclusion}.
\end{defn}

\guard

\begin{defn}
\label{defn:set}
\index{set}
  A \emph{set} is a collection of things.
\label{defn:membership}
\index{$\in$}
  The symbol $\in$ is used denote the \emph{membership} relation on sets.
  For $S$ a set and $x$ arbitrary, we say $x\in S$ precisely when $x$ is a member (or element) of $S$.
\end{defn}


\begin{defn}
\label{defn:subset}
\index{subset}
\index{$\subseteq$}
  The \emph{subset} relation, $\subseteq$, is defined as \[ x\subseteq y\iff\forall z(z\in x\rightarrow z\in y)\,.\]
\index{subset!proper}
\index{$\subset$}
\index{$\subsetneq$}
  We use $x\subset y$, or $x\subsetneq y$, to abbreviate $x\subseteq y\wedge x\not=y$.
  If $x\subsetneq y$, we say that $x$ is a \emph{proper subset} of $y$.
\index{super set}
\index{$\supseteq$}
  We use $A\supseteq B$ to mean $B\subseteq A$, and is read $A$ is a super set of $B$.
  The $\supset$ and $\supsetneq$ variations are similar to the subset variations.
\end{defn}

\guard

\guard

\begin{defn}
\label{defn:set}
\index{set}
  A \emph{set} is a collection of things.
\label{defn:membership}
\index{$\in$}
  The symbol $\in$ is used denote the \emph{membership} relation on sets.
  For $S$ a set and $x$ arbitrary, we say $x\in S$ precisely when $x$ is a member (or element) of $S$.
\end{defn}


\begin{defn}
\label{defn:intersection}
\index{intersection}
  Let $A$ and $B$ sets.
  The \emph{intersection} of $A$ and $B$, $A\cap B$ is the set which contains everything which is in $A$ and in $B$.
  \[ A\cap B := \set{ x \mid x\in A \wedge x\in B}\,.\]
\end{defn}


\guard

\guard

\input{setTheory/defns/set.tex}

\begin{defn}
\label{defn:emptyset}
\index{emptyset}
\index{$\emptyset$}
  The set which contains nothing is called the \emph{emptyset}.
  We use $\emptyset$ to denote the emptyset.
\end{defn}

\guard

\input{logic/defns/conjunction.tex}
\input{logic/defns/conditional.tex}
\input{setTheory/defns/set.tex}

\begin{defn}
\label{defn:subset}
\index{subset}
\index{$\subseteq$}
  The \emph{subset} relation, $\subseteq$, is defined as \[ x\subseteq y\iff\forall z(z\in x\rightarrow z\in y)\,.\]
\index{subset!proper}
\index{$\subset$}
\index{$\subsetneq$}
  We use $x\subset y$, or $x\subsetneq y$, to abbreviate $x\subseteq y\wedge x\not=y$.
  If $x\subsetneq y$, we say that $x$ is a \emph{proper subset} of $y$.
\index{super set}
\index{$\supseteq$}
  We use $A\supseteq B$ to mean $B\subseteq A$, and is read $A$ is a super set of $B$.
  The $\supset$ and $\supsetneq$ variations are similar to the subset variations.
\end{defn}


\begin{prop}
\label{prop:emptysetIsSubsetOfEverySet}
  For any $A$, $\emptyset\subseteq A$.
\end{prop}
\begin{proof}
  This holds trivially.
\end{proof}

\guard

\guard

\input{logic/defns/conjunction.tex}
\input{logic/defns/conditional.tex}
\input{setTheory/defns/set.tex}

\begin{defn}
\label{defn:subset}
\index{subset}
\index{$\subseteq$}
  The \emph{subset} relation, $\subseteq$, is defined as \[ x\subseteq y\iff\forall z(z\in x\rightarrow z\in y)\,.\]
\index{subset!proper}
\index{$\subset$}
\index{$\subsetneq$}
  We use $x\subset y$, or $x\subsetneq y$, to abbreviate $x\subseteq y\wedge x\not=y$.
  If $x\subsetneq y$, we say that $x$ is a \emph{proper subset} of $y$.
\index{super set}
\index{$\supseteq$}
  We use $A\supseteq B$ to mean $B\subseteq A$, and is read $A$ is a super set of $B$.
  The $\supset$ and $\supsetneq$ variations are similar to the subset variations.
\end{defn}

\guard

\input{setTheory/defns/set.tex}

\begin{defn}
\label{defn:intersection}
\index{intersection}
  Let $A$ and $B$ sets.
  The \emph{intersection} of $A$ and $B$, $A\cap B$ is the set which contains everything which is in $A$ and in $B$.
  \[ A\cap B := \set{ x \mid x\in A \wedge x\in B}\,.\]
\end{defn}


\begin{prop}
\label{prop:intersectionIsSubadditive}
  For any $A$ and $B$, $A\cap B\subseteq B$.
\end{prop}
\begin{proof}
  Let $A$ and $B$ be arbitrary.
  Fix any $x\in A\cap B$.
  As $x\in A\cap B$, $x\in A$ and $x\in B$.
  Thus, $x\in A\cap B$.
  But, $x$ was arbitrary, so $A\cap B\subseteq B$.
\end{proof}


\begin{prop}
\label{prop:intersectionWithEmptysetIsEmpty}
  For any $A$, $A\cap\emptyset=\emptyset$.
\end{prop}
\begin{proof}
  Let $A$ be arbitrary.
  Applying Proposition \ref{prop:intersectionIsSubadditive}, we have that $A\cap\emptyset\subseteq \emptyset$.
  Further, Proposition \ref{prop:emptysetIsSubsetOfEverySet}, we have that $\emptyset\subseteq A\cap\emptyset$.
  Thus, $A\cap\emptyset=\emptyset$.
\end{proof}


\guard

\guard

\guard

\input{logic/defns/statement.tex}
\input{logic/defns/truthValue.tex}


\begin{defn}
\label{defn:conjunctionOfStatement}
  Let $p$ and $q$ be statements.
  The \emph{conjunction} of $p$ and $q$, written $p \wedge q$, is the statement that is true precisely when both $p$ and $q$ are true and is otherwise false.
\end{defn}

\guard

\input{logic/defns/statement.tex}
\input{logic/defns/statementForm.tex}
\input{logic/defns/truthValue.tex}


\begin{defn}
\label{defn:conditionalStatement}
\index{conditional}
  Let $p$ and $q$ be statements forms.
  The \emph{conditional statement} ``$p$ implies $q$'', written $p \rightarrow q$, is the statement form that is false precisely when $p$ is true and $q$ is false ( that is, when the statement ``If $p$, then $q$'' is violated ).

  In the conditional $p\rightarrow q$, $p$ is refered to as the \emph{hypothesis} and $q$ is called the \emph{conclusion}.
\end{defn}

\guard

\begin{defn}
\label{defn:set}
\index{set}
  A \emph{set} is a collection of things.
\label{defn:membership}
\index{$\in$}
  The symbol $\in$ is used denote the \emph{membership} relation on sets.
  For $S$ a set and $x$ arbitrary, we say $x\in S$ precisely when $x$ is a member (or element) of $S$.
\end{defn}


\begin{defn}
\label{defn:subset}
\index{subset}
\index{$\subseteq$}
  The \emph{subset} relation, $\subseteq$, is defined as \[ x\subseteq y\iff\forall z(z\in x\rightarrow z\in y)\,.\]
\index{subset!proper}
\index{$\subset$}
\index{$\subsetneq$}
  We use $x\subset y$, or $x\subsetneq y$, to abbreviate $x\subseteq y\wedge x\not=y$.
  If $x\subsetneq y$, we say that $x$ is a \emph{proper subset} of $y$.
\index{super set}
\index{$\supseteq$}
  We use $A\supseteq B$ to mean $B\subseteq A$, and is read $A$ is a super set of $B$.
  The $\supset$ and $\supsetneq$ variations are similar to the subset variations.
\end{defn}

\guard

\guard

\begin{defn}
\label{defn:set}
\index{set}
  A \emph{set} is a collection of things.
\label{defn:membership}
\index{$\in$}
  The symbol $\in$ is used denote the \emph{membership} relation on sets.
  For $S$ a set and $x$ arbitrary, we say $x\in S$ precisely when $x$ is a member (or element) of $S$.
\end{defn}


\begin{defn}
\label{defn:intersection}
\index{intersection}
  Let $A$ and $B$ sets.
  The \emph{intersection} of $A$ and $B$, $A\cap B$ is the set which contains everything which is in $A$ and in $B$.
  \[ A\cap B := \set{ x \mid x\in A \wedge x\in B}\,.\]
\end{defn}

\guard

\guard

\begin{defn}
\label{defn:set}
\index{set}
  A \emph{set} is a collection of things.
\label{defn:membership}
\index{$\in$}
  The symbol $\in$ is used denote the \emph{membership} relation on sets.
  For $S$ a set and $x$ arbitrary, we say $x\in S$ precisely when $x$ is a member (or element) of $S$.
\end{defn}


\begin{defn}
\label{defn:union}
\index{union}
  Let $A$ and $B$ sets.
  The \emph{union} of $A$ and $B$, $A\cup B$ is the set which contains everything which is in $A$ or in $B$.
  \[ A\cup B := \set{ x \mid x\in A \vee x\in B}\,.\]
\end{defn}


\guard

\guard

\input{logic/defns/conjunction.tex}
\input{logic/defns/conditional.tex}
\input{setTheory/defns/set.tex}

\begin{defn}
\label{defn:subset}
\index{subset}
\index{$\subseteq$}
  The \emph{subset} relation, $\subseteq$, is defined as \[ x\subseteq y\iff\forall z(z\in x\rightarrow z\in y)\,.\]
\index{subset!proper}
\index{$\subset$}
\index{$\subsetneq$}
  We use $x\subset y$, or $x\subsetneq y$, to abbreviate $x\subseteq y\wedge x\not=y$.
  If $x\subsetneq y$, we say that $x$ is a \emph{proper subset} of $y$.
\index{super set}
\index{$\supseteq$}
  We use $A\supseteq B$ to mean $B\subseteq A$, and is read $A$ is a super set of $B$.
  The $\supset$ and $\supsetneq$ variations are similar to the subset variations.
\end{defn}

\guard

\input{setTheory/defns/set.tex}

\begin{defn}
\label{defn:intersection}
\index{intersection}
  Let $A$ and $B$ sets.
  The \emph{intersection} of $A$ and $B$, $A\cap B$ is the set which contains everything which is in $A$ and in $B$.
  \[ A\cap B := \set{ x \mid x\in A \wedge x\in B}\,.\]
\end{defn}


\begin{prop}
\label{prop:intersectionIsSubadditive}
  For any $A$ and $B$, $A\cap B\subseteq B$.
\end{prop}
\begin{proof}
  Let $A$ and $B$ be arbitrary.
  Fix any $x\in A\cap B$.
  As $x\in A\cap B$, $x\in A$ and $x\in B$.
  Thus, $x\in A\cap B$.
  But, $x$ was arbitrary, so $A\cap B\subseteq B$.
\end{proof}

\guard

\guard

\input{logic/defns/conjunction.tex}
\input{logic/defns/conditional.tex}
\input{setTheory/defns/set.tex}

\begin{defn}
\label{defn:subset}
\index{subset}
\index{$\subseteq$}
  The \emph{subset} relation, $\subseteq$, is defined as \[ x\subseteq y\iff\forall z(z\in x\rightarrow z\in y)\,.\]
\index{subset!proper}
\index{$\subset$}
\index{$\subsetneq$}
  We use $x\subset y$, or $x\subsetneq y$, to abbreviate $x\subseteq y\wedge x\not=y$.
  If $x\subsetneq y$, we say that $x$ is a \emph{proper subset} of $y$.
\index{super set}
\index{$\supseteq$}
  We use $A\supseteq B$ to mean $B\subseteq A$, and is read $A$ is a super set of $B$.
  The $\supset$ and $\supsetneq$ variations are similar to the subset variations.
\end{defn}

\guard

\input{setTheory/defns/set.tex}

\begin{defn}
\label{defn:union}
\index{union}
  Let $A$ and $B$ sets.
  The \emph{union} of $A$ and $B$, $A\cup B$ is the set which contains everything which is in $A$ or in $B$.
  \[ A\cup B := \set{ x \mid x\in A \vee x\in B}\,.\]
\end{defn}


\begin{prop}
\label{prop:unionIsAdditive}
  For any $A$ and $B$, $A\subseteq A\cup B$.
\end{prop}
\begin{proof}
  Let $A$ and $B$ be arbitrary.
  Fix any $x\in A$.
  As $x\in A$, $x\in A$ or $x\in B$.
  Thus, $x\in A\cup B$.
  But, $x$ was arbitrary, so $A\subseteq A\cup B$.
\end{proof}


\begin{prop}
\label{prop:subsetRelationAndUnionsIntersections}
  For any $A$ and $B$.
  If $A\subseteq B$, then
  \begin{enumerate}
    \item $A\cap B = A$ and
    \item $A\cup B = B$.
  \end{enumerate}
\end{prop}
\begin{proof}
  Fix $A$ and $B$.
  Suppose that $A\subseteq B$.
  \begin{enumerate}
    \item By Proposition \ref{prop:intersectionIsSubadditive}, $A\cap B\subseteq A$.
      It remains to show that $A\subseteq A\cap B$.
      Fix any $a\in A$.
      As $a\in A$ and $A\subseteq B$, $a\in B$.
      Thus, $a\in A$ and $a\in B$.
      So, $a\in A\cap B$, showing that $A\subseteq A\cap B$.

    \item By Proposition \ref{prop:unionIsAdditive}, $B\subseteq A\cup B$.
      It remains to show that $A\cup B\subseteq B$.
      Fix any $b\in A\cup B$.
      By the Definition \ref{defn:union}, $b\in A$ or $b\in B$.
      If $b\in A$, then as $A\subseteq B$ $b\in B$.
      So, in either case we have that $b\in B$.
      Whence, $A\cap B\subseteq B$.
      Showing that $A\cap B= B$
  \end{enumerate}
\end{proof}

\guard

\guard

\guard

\input{logic/defns/statement.tex}
\input{logic/defns/truthValue.tex}


\begin{defn}
\label{defn:conjunctionOfStatement}
  Let $p$ and $q$ be statements.
  The \emph{conjunction} of $p$ and $q$, written $p \wedge q$, is the statement that is true precisely when both $p$ and $q$ are true and is otherwise false.
\end{defn}

\guard

\input{logic/defns/statement.tex}
\input{logic/defns/statementForm.tex}
\input{logic/defns/truthValue.tex}


\begin{defn}
\label{defn:conditionalStatement}
\index{conditional}
  Let $p$ and $q$ be statements forms.
  The \emph{conditional statement} ``$p$ implies $q$'', written $p \rightarrow q$, is the statement form that is false precisely when $p$ is true and $q$ is false ( that is, when the statement ``If $p$, then $q$'' is violated ).

  In the conditional $p\rightarrow q$, $p$ is refered to as the \emph{hypothesis} and $q$ is called the \emph{conclusion}.
\end{defn}

\guard

\begin{defn}
\label{defn:set}
\index{set}
  A \emph{set} is a collection of things.
\label{defn:membership}
\index{$\in$}
  The symbol $\in$ is used denote the \emph{membership} relation on sets.
  For $S$ a set and $x$ arbitrary, we say $x\in S$ precisely when $x$ is a member (or element) of $S$.
\end{defn}


\begin{defn}
\label{defn:subset}
\index{subset}
\index{$\subseteq$}
  The \emph{subset} relation, $\subseteq$, is defined as \[ x\subseteq y\iff\forall z(z\in x\rightarrow z\in y)\,.\]
\index{subset!proper}
\index{$\subset$}
\index{$\subsetneq$}
  We use $x\subset y$, or $x\subsetneq y$, to abbreviate $x\subseteq y\wedge x\not=y$.
  If $x\subsetneq y$, we say that $x$ is a \emph{proper subset} of $y$.
\index{super set}
\index{$\supseteq$}
  We use $A\supseteq B$ to mean $B\subseteq A$, and is read $A$ is a super set of $B$.
  The $\supset$ and $\supsetneq$ variations are similar to the subset variations.
\end{defn}

\guard

\guard

\begin{defn}
\label{defn:set}
\index{set}
  A \emph{set} is a collection of things.
\label{defn:membership}
\index{$\in$}
  The symbol $\in$ is used denote the \emph{membership} relation on sets.
  For $S$ a set and $x$ arbitrary, we say $x\in S$ precisely when $x$ is a member (or element) of $S$.
\end{defn}


\begin{defn}
\label{defn:union}
\index{union}
  Let $A$ and $B$ sets.
  The \emph{union} of $A$ and $B$, $A\cup B$ is the set which contains everything which is in $A$ or in $B$.
  \[ A\cup B := \set{ x \mid x\in A \vee x\in B}\,.\]
\end{defn}

\guard

\guard

\begin{defn}
\label{defn:set}
\index{set}
  A \emph{set} is a collection of things.
\label{defn:membership}
\index{$\in$}
  The symbol $\in$ is used denote the \emph{membership} relation on sets.
  For $S$ a set and $x$ arbitrary, we say $x\in S$ precisely when $x$ is a member (or element) of $S$.
\end{defn}


\begin{defn}
\label{defn:intersection}
\index{intersection}
  Let $A$ and $B$ sets.
  The \emph{intersection} of $A$ and $B$, $A\cap B$ is the set which contains everything which is in $A$ and in $B$.
  \[ A\cap B := \set{ x \mid x\in A \wedge x\in B}\,.\]
\end{defn}


\begin{prop}
\label{prop:unionDistributionOverIntersection}
  For any $A$, $B$, and $C$, \[ A\cup( B\cap C) = (A\cup B) \cap (A\cup C)\,. \]
\end{prop}
\begin{proof}
  Let $A$, $B$, and $C$ be arbitrary.

  ($\subseteq$) Fix $x\in A\cup( B\cap C)$.
    By Definition \ref{defn:union},  $x\in A$ pr $x\in(B\cap C)$.

    If $x\in A$, then $x\in A\cup B$ and $x\in A\cup C$.
    Thus, $x\in (A\cup B) \cap (A\cup C)$.

    If $x\in B\cap C$, then $x\in B$ and $x\in C$.
    As $x\in B$ and $x\in C$, $x\in A\cup B$ and $x\in A\cup C$.
    So, $x\in (A\cup B) \cap (A\cup C)$.

    But, $x\in  A\cup( B\cap C)$ was arbitrary, so \[ A\cup( B\cap C) \subseteq (A\cup B) \cap (A\cup C)\,.\]

  ($\supseteq$) Fix $x\in (A\cup B)\cap (A\cup B)$.
  Then, $x\in A\cup B$ and $x\in A\cup C$.
  Either $x$ is in $A$ or it is not.

  If $x$ is in $A$, when $x\in A\cup(B\cap C)$, and we are done.

  Suppose $x$ is not in $A$.
  As $x\in A\cup B$ and $x$ is not in $A$, $x\in B$.
  Similarly, as $x\in A\cup B$ and $x\not\in A$, $x\in C$.
  Thus, $x\in B\cap C$.
  Whence, $x\in A\cup(B\cap C)$.
\end{proof}


\guard

\guard

\guard

\input{setTheory/defns/set.tex}

\begin{defn}
\label{defn:emptyset}
\index{emptyset}
\index{$\emptyset$}
  The set which contains nothing is called the \emph{emptyset}.
  We use $\emptyset$ to denote the emptyset.
\end{defn}


\begin{defn}
\label{defn:disjoint}
\index{disjoint}
  Sets $A$ and $B$ are said to be \emph{disjoint} when $ A\cap B=\emptyset$
\end{defn}

\guard

\guard

\input{numberTheory/defns/divides.tex}

\begin{thm}
\label{thm:quotientRemainderTheorem}
  \textbf{(Quotient-Remainder Theorem)}
  Given an integer $n$ and positive integer $d$, there exists unique integers $q$ and $r$ such that $0\leq r < d$ and $n=qd + r$.
\end{thm}
\begin{proof}
  Fix $n\in\ZZ$ and $d\in\ZZ^+$.

  First, we will show existence.
  Consider the set of non-negative integers of the form $n-dq$.
  This set is clearly non-empty.
  Thus, as $\NN$ is well-ordered, this set contains a least element, say $r=n-dq$.
  Certianly, $r\geq 0$.
  Now, $r<d$ as as otherwise $n-d(q+1)$ contradicts the minimality of $r$.

  Now to show uniqueness.
  Suppose that there exists $q_0,q_1,r_0,r_1\in\ZZ$ with $0\leq r_0,r_1 <d$ and \[n=dq_0+r_0=dq_1+r_1\,.\]
  Note, as $dq_0+r_0=dq_1+r_1$, it suffices to show $r_0=r_1$.
  Rewriting $dq_0+r_0=dq_1+r_1$, we see that $r_0-r_1=d(q_1-q_0)$ showing that $d|(r_0-r_1)$.
  Further, as $0\leq r_1 <d$, $-d<r_1\leq 0$.
  So, $-d < r_0-r_1 <d$.
  Though, as $d|(r_0-r_1)$ and $-d < r_0-r_1 <d$, $r_0-r_1=0$.
\end{proof}


\begin{prop}
\label{prop:integersAreEvenXorOdd}
  Every integer is either even or odd, but not both.
\end{prop}
\begin{proof}
  Fix $n\in\ZZ$ arbitrary.
  As $2\in\ZZ$ is positive and $n\in\ZZ$, the Quotient Remainder Theorem, Theorem \ref{thm:quotientRemainderTheorem}, applies, and there exist $q,r\in\ZZ$ such that $0\leq r<2$ and $n=2q+r$.
  Though, as $r\in\ZZ$ and $0\leq r<2$, $r=0,1$.
  Therefore, $n=2q$ or $n=2q+1$.
  Thus, $n$ is either even or odd.
  Moreover, as the Quotient Remainder Theorem  give that $q$ and $r$ are unique, $n$ cannot be even and odd.
\end{proof}


\begin{exmp}
\label{exmp:setOfEvenAndOddIntegersAreDisjoint}
  Let $E$ and $O$ be the set of even and odd integers respectively.
  $E$ and $O$ are disjoint.

  To show this, consider any $n\in\ZZ$.
  Using Proposition \ref{prop:integersAreEvenXorOdd}, $n$ is either even or odd, but not both.
  In the case that $n$ is even, $n\in E$ and $n\not\in O$.
  Similarly, if $n$ is odd, then $n\in O$ and $n\not\in O$.
  Thus \[E\cap O=\set{ n\in\ZZ\mid n\in O \wedge n\in E}=\emptyset\,.\]
  So, $E$ and $O$ are disjoint.
\end{exmp}


\guard

\guard

\guard

\input{setTheory/defns/set.tex}

\begin{defn}
\label{defn:union}
\index{union}
  Let $A$ and $B$ sets.
  The \emph{union} of $A$ and $B$, $A\cup B$ is the set which contains everything which is in $A$ or in $B$.
  \[ A\cup B := \set{ x \mid x\in A \vee x\in B}\,.\]
\end{defn}

\guard

\input{logic/defns/universalConditionalStatement.tex}
\input{setTheory/defns/disjoint.tex}

\begin{defn}
\label{defn:mutuallyDisjointCollection}
\index{disjoint!mutually}
  $\mathcal{A}$ is said to be a \emph{mutually disjoint} collection provided $\forall A,B\in\mathcal{A}$ $A$ and $B$ are disjoint.
\end{defn}


\begin{defn}
\label{defn:partition}
\index{partition}
  \sloppy $\mathcal{A}$ is said to be a \emph{partition} of a set $X$ provided ${\bigcup_{A\in\mathcal{A}} A = X}$ and $\mathcal{A}$ is a mutually disjoint collection.
\end{defn}

\guard

\guard

\input{setTheory/defns/set.tex}

\begin{defn}
\label{defn:union}
\index{union}
  Let $A$ and $B$ sets.
  The \emph{union} of $A$ and $B$, $A\cup B$ is the set which contains everything which is in $A$ or in $B$.
  \[ A\cup B := \set{ x \mid x\in A \vee x\in B}\,.\]
\end{defn}


\begin{prop}
\label{prop:unionIsCommutative}
  For any $A$ and $B$, $A\cup B = B\cup A$.
\end{prop}
\begin{proof}
  \begin{align*}
    A\cup B &= \set{ x\mid x\in A \vee x\in B } \\
            &= \set{ x\mid x\in B \vee x\in A } \\
            &= B\cup A
  \end{align*}
\end{proof}


\begin{exmp}
\label{exmp:easyParitions}
  Let $A=\set{ 1,2,3,4,5,6}$, $A_1 = \set{1,2}$, $A_2=\set{3,4}$, and $A_3=\set{5,6}$.
    Note, as $A_1\cap A_2=A_1\cap A_3=A_2\cap A_3=\emptyset$ $\set{A_1,A_2,A_3}$ is a mutually disjoint collection.
    Finally as $A_1\cup A_2\cup A_3 = A$, $\set{A_1,A_2,A_3}$ is a partition of $X$.
\end{exmp}

\guard

\guard

\guard

\input{setTheory/defns/set.tex}

\begin{defn}
\label{defn:union}
\index{union}
  Let $A$ and $B$ sets.
  The \emph{union} of $A$ and $B$, $A\cup B$ is the set which contains everything which is in $A$ or in $B$.
  \[ A\cup B := \set{ x \mid x\in A \vee x\in B}\,.\]
\end{defn}

\guard

\input{logic/defns/universalConditionalStatement.tex}
\input{setTheory/defns/disjoint.tex}

\begin{defn}
\label{defn:mutuallyDisjointCollection}
\index{disjoint!mutually}
  $\mathcal{A}$ is said to be a \emph{mutually disjoint} collection provided $\forall A,B\in\mathcal{A}$ $A$ and $B$ are disjoint.
\end{defn}


\begin{defn}
\label{defn:partition}
\index{partition}
  \sloppy $\mathcal{A}$ is said to be a \emph{partition} of a set $X$ provided ${\bigcup_{A\in\mathcal{A}} A = X}$ and $\mathcal{A}$ is a mutually disjoint collection.
\end{defn}


\begin{exercise}
\label{exercise:paritionOfIntegers}
  Let
  \begin{align*}
    T_0 &= \set{n\in\ZZ \mid \exists k\in\ZZ(n=3k)} \\
    T_1 &= \set{n\in\ZZ \mid \exists k\in\ZZ(n=3k+1)} \\
    T_2 &= \set{n\in\ZZ \mid \exists k\in\ZZ(n=3k+2)} \,.
  \end{align*}
  Show that $\set{ T_0, T_1, T_2}$ is a partition of $\ZZ$.
\end{exercise}


\guard

\guard

\guard

\input{setTheory/defns/set.tex}

\begin{defn}
\label{defn:tuple}
\index{tuple}
  Let $n\in\ZZ^+$ and $x_1,x_2,\dots, x_n$ be given.
  Then, the \emph{$n$-tuple}, often just \emph{tuple}, $(x_1,x_2,\dots,x_n)$ is the collection containing $x_1,x_2,\dots,x_n$ together with their order.
\label{defn:orderedPair}
\index{orderedPair}
  A $2$-tuple is called an \emph{ordered pair}.
\label{defn:orderedTriple}
\index{ordered Triple}
  A $3$-tuple is called an \emph{ordered triple}.
\end{defn}


\begin{defn}
\label{defn:cartesianProduct}
\index{cartesian product}
  Let $A$ and $B$ be given.
  The \emph{caresian product} of $A$ and $B$, written $A\times B$, is the set of all ordered pairs $(a,b)$ where $a\in A$ and $b\in B$.
  \[ A\times B := \set{ (a,b)\mid a\in A,b\in B}\,.\]
\end{defn}


\begin{exmp}
\label{exmp:cartesianProductEasy}
  \begin{align*}
    \set{1,2}\times\set{a,b} := \set{ (1,a),(1,b),(2,a),(2,b)}\,.
  \end{align*}

\end{exmp}


\guard

\guard

\guard

\begin{defn}
\label{defn:set}
\index{set}
  A \emph{set} is a collection of things.
\label{defn:membership}
\index{$\in$}
  The symbol $\in$ is used denote the \emph{membership} relation on sets.
  For $S$ a set and $x$ arbitrary, we say $x\in S$ precisely when $x$ is a member (or element) of $S$.
\end{defn}

\guard

\input{logic/defns/conjunction.tex}
\input{logic/defns/conditional.tex}
\input{setTheory/defns/set.tex}

\begin{defn}
\label{defn:subset}
\index{subset}
\index{$\subseteq$}
  The \emph{subset} relation, $\subseteq$, is defined as \[ x\subseteq y\iff\forall z(z\in x\rightarrow z\in y)\,.\]
\index{subset!proper}
\index{$\subset$}
\index{$\subsetneq$}
  We use $x\subset y$, or $x\subsetneq y$, to abbreviate $x\subseteq y\wedge x\not=y$.
  If $x\subsetneq y$, we say that $x$ is a \emph{proper subset} of $y$.
\index{super set}
\index{$\supseteq$}
  We use $A\supseteq B$ to mean $B\subseteq A$, and is read $A$ is a super set of $B$.
  The $\supset$ and $\supsetneq$ variations are similar to the subset variations.
\end{defn}


\begin{defn}
\label{defn:powerSet}
\index{power set}
  Let $A$ be a set.
  The set of all subsets of $A$, $\mathcal{P}$, is called the \emph{power set} of $A$.
  \[ \mathcal{P} := \set{ B\mid B\subseteq A}\,.\]
\end{defn}


\begin{exmp}
\label{exmp:powerSetEasy}
  \begin{align*}
    \mathcal{P}(\set{a,b}) &= \set{ \set{}, \set{a}, \set{b}, \set{a,b} }\,.
  \end{align*}
  Note, $a,b\not\in\mathcal{P}$, though $a,b\in A$.
\end{exmp}

\guard

\guard

\guard

\begin{defn}
\label{defn:set}
\index{set}
  A \emph{set} is a collection of things.
\label{defn:membership}
\index{$\in$}
  The symbol $\in$ is used denote the \emph{membership} relation on sets.
  For $S$ a set and $x$ arbitrary, we say $x\in S$ precisely when $x$ is a member (or element) of $S$.
\end{defn}


\begin{defn}
\label{defn:emptyset}
\index{emptyset}
\index{$\emptyset$}
  The set which contains nothing is called the \emph{emptyset}.
  We use $\emptyset$ to denote the emptyset.
\end{defn}

\guard

\guard

\begin{defn}
\label{defn:set}
\index{set}
  A \emph{set} is a collection of things.
\label{defn:membership}
\index{$\in$}
  The symbol $\in$ is used denote the \emph{membership} relation on sets.
  For $S$ a set and $x$ arbitrary, we say $x\in S$ precisely when $x$ is a member (or element) of $S$.
\end{defn}

\guard

\input{logic/defns/conjunction.tex}
\input{logic/defns/conditional.tex}
\input{setTheory/defns/set.tex}

\begin{defn}
\label{defn:subset}
\index{subset}
\index{$\subseteq$}
  The \emph{subset} relation, $\subseteq$, is defined as \[ x\subseteq y\iff\forall z(z\in x\rightarrow z\in y)\,.\]
\index{subset!proper}
\index{$\subset$}
\index{$\subsetneq$}
  We use $x\subset y$, or $x\subsetneq y$, to abbreviate $x\subseteq y\wedge x\not=y$.
  If $x\subsetneq y$, we say that $x$ is a \emph{proper subset} of $y$.
\index{super set}
\index{$\supseteq$}
  We use $A\supseteq B$ to mean $B\subseteq A$, and is read $A$ is a super set of $B$.
  The $\supset$ and $\supsetneq$ variations are similar to the subset variations.
\end{defn}


\begin{defn}
\label{defn:powerSet}
\index{power set}
  Let $A$ be a set.
  The set of all subsets of $A$, $\mathcal{P}$, is called the \emph{power set} of $A$.
  \[ \mathcal{P} := \set{ B\mid B\subseteq A}\,.\]
\end{defn}



\begin{prop}
\label{prop:sizeOfFinitePowerset}
  For all $n\in\NN$, if $X$ is a set with $n$ elements, then $\mathcal{P}$ has $2^n$ elements.
\end{prop}
\begin{proof}
  We prove this by induction on $n$.

  In the case where $n=0$.
  Then, the only set with $0$ elements is the emptyset.
  We now see \[\mathcal{P}(\emptyset) = \set{\emptyset} \,\] has $2^0=1$ elements.

  Now, suppose $n\in\NN$ and the power set of any set with $n$ elements has  $2^n$ elements.
  Let $X$ be a set with $n+1$ many elements.
  As $X$ has $n+1$ many elements, there exists some $x\in X$.
  Fix any $x\in X$.

  Note, $X\setminus\set{x}$ has $n$ elements.
  So, by our inductive hypothesis $\mathcal{P}(X\setminus\set{x})$ has $2^n$ elements.

  $\mathcal{P}(X\setminus\set{x})$ contains half of the elements in $\mathcal{P}(X)$.
  In particular, it is missing all the subsets of $X$ which contain $x$.
  Note \[ \mathcal{P}(X) = \mathcal{P}(X\setminus\set{x}) \cup \set{ A\cup\set{x} \mid A\in \mathcal{P}(X\setminus\set{x}) }\,.\]
  As $\mathcal{P}(X\setminus\set{x})$ has $2^n$ elements, $\set{ A\cup\set{x} \mid A\in \mathcal{P}(X\setminus\set{x}) }$ has $2^n$ many elements.
  Further, as each element of $\mathcal{P}(X\setminus\set{x})$ does not contain $x$ and each element of $\set{ A\cup\set{x} \mid A\in \mathcal{P}(X\setminus\set{x}) }$ does, those two sets are disjoint.
  This, $\mathcal{P}(X)$ has $2\cdot 2^n=2^{n+1}$ elements.
\end{proof}

