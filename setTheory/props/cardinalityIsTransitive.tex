\guard

\guard

\guard

\input{setTheory/defns/injection.tex}
\input{setTheory/defns/surjection.tex}

\begin{defn}
\label{defn:bijection}
\index{bijection}
  A function $f:X\to Y$ is said to be \emph{bijection} provided $f$ is injective and surjective.
\end{defn}

\guard

\guard

\input{setTheory/defns/function.tex}

\begin{defn}
\label{defn:injective}
\index{injective}
  A function $f:X\to Y$ is said to be \emph{injective} (or \emph{one-to-one}) provided $\forall x,y\in X$ if $x\not=y$, then $f(x)\not= f(y)$.
\end{defn}

\guard

\input{setTheory/defns/function.tex}

\begin{defn}
\label{defn:functionComposition}
\index{function!composition}
  Let $f:X\to Y$ and $g:Y\to Z$.
  The \emph{composition} of $f$ and $g$, written $g\circ f:X\to Z$, is defined by $(g\circ f )(x) := g(f(x))$ for each $x\in X$.
\end{defn}

\guard

\input{setTheory/defns/injective.tex}
\input{setTheory/defns/functionComposition.tex}
\input{setTheory/props/compositionOfInjectionIsInjection.tex}

\begin{prop}
\label{prop:compositionOfInjectionIsInjection}
  If $f:X\to Y$ and $g:Y\to Z$ are both injective, then $g\circ f$ is injective.
\end{prop}
\begin{proof}
  Fix any $x,y\in X$ and suppose that $x\not=y$.
  As $f$ is injective and $x\not=y$, $f(x)\not=f(y)$.
  As $g$ is injective, and $f(x),f(y)\in Y$ are distinct, $g(f(x))\not= g(f(y))$.
  Thus, $g\circ f$ is injective.
\end{proof}


\begin{prop}
\label{prop:compositionOfInjectionIsInjection}
  If $f:X\to Y$ and $g:Y\to Z$ are both injective, then $g\circ f$ is injective.
\end{prop}
\begin{proof}
  Fix any $x,y\in X$ and suppose that $x\not=y$.
  As $f$ is injective and $x\not=y$, $f(x)\not=f(y)$.
  As $g$ is injective, and $f(x),f(y)\in Y$ are distinct, $g(f(x))\not= g(f(y))$.
  Thus, $g\circ f$ is injective.
\end{proof}

\guard

\guard

\input{setTheory/defns/function.tex}

\begin{defn}
\label{defn:surjective}
\index{surjective}
  A function $f:X\to Y$ is said to be \emph{surjective} (or \emph{onto}) provided $\forall y\in Y\exists x\in X$ $f(x)=y$.
\end{defn}

\guard

\input{setTheory/defns/function.tex}

\begin{defn}
\label{defn:functionComposition}
\index{function!composition}
  Let $f:X\to Y$ and $g:Y\to Z$.
  The \emph{composition} of $f$ and $g$, written $g\circ f:X\to Z$, is defined by $(g\circ f )(x) := g(f(x))$ for each $x\in X$.
\end{defn}


\begin{prop}
\label{prop:compositionOfSurjectionIsSurjection}
  If $f:X\to Y$ and $g:Y\to Z$ are both surjective, then $g\circ f$ is surjective.
\end{prop}
\begin{proof}
  Fix any $z\in Z$.
  As $g$ is surjective, there exists some $y\in Y$ such that $g(y)=z$.
  As $f$ is surjective, there exists some $x\in X$ such that $f(x)=z$.
  Now, we note that
  \begin{align*}
    (g\circ f)(x) &= g(f(x)) \\
                  &= g(y) \\
                  &= z\,.
  \end{align*}
  Showing that $g\circ f$ is surjective.
\end{proof}


\begin{prop}
\label{prop:compositionOfBijectionsIsBijection}
  If $f:X\to Y$ and $g:Y\to Z$ are both bijective, then $g\circ f$ is bijective.
\end{prop}
\begin{proof}
  As $f$ and $g$ are bijections, they are both surjective and injective.
  Thus, by applying Propositions \ref{prop:compositionOfSurjectionIsSurjection} and \ref{prop:compositionOfInjectionIsInjection}, we have that $g\circ f$ is a surjective and injective.
  Thus, as $g\circ f$ is a bijection.
\end{proof}

\guard

\guard

\input{setTheory/defns/injection.tex}
\input{setTheory/defns/surjection.tex}

\begin{defn}
\label{defn:bijection}
\index{bijection}
  A function $f:X\to Y$ is said to be \emph{bijection} provided $f$ is injective and surjective.
\end{defn}


\begin{defn}
\label{defn:sameCardinality}
\index{cardinality!equal}
  For sets $A$ and $B$, $A$ is said to have the \emph{same cardinality} as $B$, written $\vert A\vert = \vert B\vert$, provided there exists a bijection from $A$ to $B$.
\end{defn}


\begin{prop}
\label{prop:cardinalityIsTransitive}
  For any sets $A$, $B$, and $C$ if $A$ has the same cardinality as $B$ and $B$ has the same cardinality as $C$, then $A$ has the same cardinality as $C$.
\end{prop}
\begin{proof}
  Fix sets $A$, $B$, and $C$, and suppose that $A$ has the same cardinality as $B$ and $B$ has the same cardinality as $C$.
  As $A$ has the same cardinality as $B$, there exists a bijection $f$ from $A$ to $B$.
  As $B$ has the same cardinality as $C$, there exists a bijection $g$ from $B$ to $C$.
  As $f$ and $g$ are bijections, applying Proposition \ref{prop:compositionOfBijectionsIsBijection}, $g\circ f$ is a bijection.
  Thus, $g\circ f$ sees that $A$ has the same cardinality as $B$.
\end{proof}
