\guard

\guard

\guard

\begin{defn}
\label{defn:set}
\index{set}
  A \emph{set} is a collection of things.
\label{defn:membership}
\index{$\in$}
  The symbol $\in$ is used denote the \emph{membership} relation on sets.
  For $S$ a set and $x$ arbitrary, we say $x\in S$ precisely when $x$ is a member (or element) of $S$.
\end{defn}


\begin{defn}
\label{defn:subset}
\index{subset}
  The \emph{subset} relation, $\subseteq$, is defined as \[ x\subseteq y\iff\forall z(z\in x\rightarrow z\in y)\,.\]
  We use $x\subset y$, or $x\subsetneq y$, to abbreviate $x\subseteq y\wedge x\not=y$.
\index{subset!proper}
  If $x\subsetneq y$, we say that $x$ is a \emph{proper subset} of $y$.
\end{defn}

\guard

\guard

\begin{defn}
\label{defn:set}
\index{set}
  A \emph{set} is a collection of things.
\label{defn:membership}
\index{$\in$}
  The symbol $\in$ is used denote the \emph{membership} relation on sets.
  For $S$ a set and $x$ arbitrary, we say $x\in S$ precisely when $x$ is a member (or element) of $S$.
\end{defn}


\begin{defn}
\label{defn:union}
\index{union}
  Let $A$ and $B$ sets.
  The \emph{union} of $A$ and $B$, $A\cup B$ is the set which contains everything which is in $A$ or in $B$.
  \[ A\cup B := \set{ x \mid x\in A \vee x\in B}\,.\]
\end{defn}

\guard

\guard

\begin{defn}
\label{defn:set}
\index{set}
  A \emph{set} is a collection of things.
\label{defn:membership}
\index{$\in$}
  The symbol $\in$ is used denote the \emph{membership} relation on sets.
  For $S$ a set and $x$ arbitrary, we say $x\in S$ precisely when $x$ is a member (or element) of $S$.
\end{defn}


\begin{defn}
\label{defn:intersection}
\index{intersection}
  Let $A$ and $B$ sets.
  The \emph{intersection} of $A$ and $B$, $A\cap B$ is the set which contains everything which is in $A$ and in $B$.
  \[ A\cap B := \set{ x \mid x\in A \wedge x\in B}\,.\]
\end{defn}


\begin{prop}
\label{prop:unionDistributionOverIntersection}
  For any $A$, $B$, and $C$, \[ A\cup( B\cap C) = (A\cup B) \cap (A\cup C)\,. \]
\end{prop}
\begin{proof}
  Let $A$, $B$, and $C$ be arbitrary.

  ($\subseteq$) Fix $x\in A\cup( B\cap C)$.
    By Definition \ref{defn:union},  $x\in A$ pr $x\in(B\cap C)$.

    If $x\in A$, then $x\in A\cup B$ and $x\in A\cup C$.
    Thus, $x\in (A\cup B) \cap (A\cup C)$.

    If $x\in B\cap C$, then $x\in B$ and $x\in C$.
    As $x\in B$ and $x\in C$, $x\in A\cup B$ and $x\in A\cup C$.
    So, $x\in (A\cup B) \cap (A\cup C)$.

    But, $x\in  A\cup( B\cap C)$ was arbitrary, so \[ A\cup( B\cap C) \subseteq (A\cup B) \cap (A\cup C)\,.\]

  ($\supseteq$) Fix $x\in (A\cup B)\cap (A\cup B)$.
  Then, $x\in A\cup B$ and $x\in A\cup C$.
  Either $x$ is in $A$ or it is not.

  If $x$ is in $A$, when $x\in A\cup(B\cap C)$, and we are done.

  Suppose $x$ is not in $A$.
  As $x\in A\cup B$ and $x$ is not in $A$, $x\in B$.
  Similarly, as $x\in A\cup B$ and $x\not\in A$, $x\in C$.
  Thus, $x\in B\cap C$.
  Whence, $x\in A\cup(B\cap C)$.
\end{proof}
