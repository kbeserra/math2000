\guard

\guard

\guard

\guard

\begin{defn}
\label{defn:statement}
  A \emph{statement} (or \emph{proposition}) is a sentence that is either true or false, but not both.
\end{defn}

\guard

\input{logic/defns/statement.tex}

\begin{defn}
\label{defn:statementTruthValue}
\index{truth value}
  The \emph{truth value} of a given statement is true if that sentence is itself true otherwise, the truth value of that statement is false.
\end{defn}



\begin{defn}
\label{defn:conjunctionOfStatement}
  Let $p$ and $q$ be statements.
  The \emph{conjunction} of $p$ and $q$, written $p \wedge q$, is the statement that is true precisely when both $p$ and $q$ are true and is otherwise false.
\end{defn}

\guard

\guard

\begin{defn}
\label{defn:statement}
  A \emph{statement} (or \emph{proposition}) is a sentence that is either true or false, but not both.
\end{defn}

\guard

\input{logic/defns/statement.tex}

% \input{logic/defns/negation.tex}
% \input{logic/defns/disjunction.tex}
% \input{logic/defns/conjunction.tex}

\begin{defn}
\label{defn:statementForm}
\index{statement form}
  A \emph{statement form} (or \emph{proposition form}) is an expression made up of statement variables and logical connections (such as $\neg$, $\vee$, or $\wedge$) which when substituting statements for statement variables becomes a statement.
\end{defn}

\guard

\input{logic/defns/statement.tex}

\begin{defn}
\label{defn:statementTruthValue}
\index{truth value}
  The \emph{truth value} of a given statement is true if that sentence is itself true otherwise, the truth value of that statement is false.
\end{defn}



\begin{defn}
\label{defn:conditionalStatement}
\index{conditional}
  Let $p$ and $q$ be statements forms.
  The \emph{conditional statement} ``$p$ implies $q$'', written $p \rightarrow q$, is the statement form that is false precisely when $p$ is true and $q$ is false ( that is, when the statement ``If $p$, then $q$'' is violated ).

  In the conditional $p\rightarrow q$, $p$ is refered to as the \emph{hypothesis} and $q$ is called the \emph{conclusion}.
\end{defn}

\guard

\begin{defn}
\label{defn:set}
\index{set}
  A \emph{set} is a collection of things.
\label{defn:membership}
\index{$\in$}
  The symbol $\in$ is used denote the \emph{membership} relation on sets.
  For $S$ a set and $x$ arbitrary, we say $x\in S$ precisely when $x$ is a member (or element) of $S$.
\end{defn}


\begin{defn}
\label{defn:subset}
\index{subset}
\index{$\subseteq$}
  The \emph{subset} relation, $\subseteq$, is defined as \[ x\subseteq y\iff\forall z(z\in x\rightarrow z\in y)\,.\]
\index{subset!proper}
\index{$\subset$}
\index{$\subsetneq$}
  We use $x\subset y$, or $x\subsetneq y$, to abbreviate $x\subseteq y\wedge x\not=y$.
  If $x\subsetneq y$, we say that $x$ is a \emph{proper subset} of $y$.
\index{super set}
\index{$\supseteq$}
  We use $A\supseteq B$ to mean $B\subseteq A$, and is read $A$ is a super set of $B$.
  The $\supset$ and $\supsetneq$ variations are similar to the subset variations.
\end{defn}

\guard

\guard

\begin{defn}
\label{defn:set}
\index{set}
  A \emph{set} is a collection of things.
\label{defn:membership}
\index{$\in$}
  The symbol $\in$ is used denote the \emph{membership} relation on sets.
  For $S$ a set and $x$ arbitrary, we say $x\in S$ precisely when $x$ is a member (or element) of $S$.
\end{defn}


\begin{defn}
\label{defn:union}
\index{union}
  Let $A$ and $B$ sets.
  The \emph{union} of $A$ and $B$, $A\cup B$ is the set which contains everything which is in $A$ or in $B$.
  \[ A\cup B := \set{ x \mid x\in A \vee x\in B}\,.\]
\end{defn}

\guard

\guard

\begin{defn}
\label{defn:set}
\index{set}
  A \emph{set} is a collection of things.
\label{defn:membership}
\index{$\in$}
  The symbol $\in$ is used denote the \emph{membership} relation on sets.
  For $S$ a set and $x$ arbitrary, we say $x\in S$ precisely when $x$ is a member (or element) of $S$.
\end{defn}


\begin{defn}
\label{defn:intersection}
\index{intersection}
  Let $A$ and $B$ sets.
  The \emph{intersection} of $A$ and $B$, $A\cap B$ is the set which contains everything which is in $A$ and in $B$.
  \[ A\cap B := \set{ x \mid x\in A \wedge x\in B}\,.\]
\end{defn}


\begin{prop}
\label{prop:unionDistributionOverIntersection}
  For any $A$, $B$, and $C$, \[ A\cup( B\cap C) = (A\cup B) \cap (A\cup C)\,. \]
\end{prop}
\begin{proof}
  Let $A$, $B$, and $C$ be arbitrary.

  ($\subseteq$) Fix $x\in A\cup( B\cap C)$.
    By Definition \ref{defn:union},  $x\in A$ pr $x\in(B\cap C)$.

    If $x\in A$, then $x\in A\cup B$ and $x\in A\cup C$.
    Thus, $x\in (A\cup B) \cap (A\cup C)$.

    If $x\in B\cap C$, then $x\in B$ and $x\in C$.
    As $x\in B$ and $x\in C$, $x\in A\cup B$ and $x\in A\cup C$.
    So, $x\in (A\cup B) \cap (A\cup C)$.

    But, $x\in  A\cup( B\cap C)$ was arbitrary, so \[ A\cup( B\cap C) \subseteq (A\cup B) \cap (A\cup C)\,.\]

  ($\supseteq$) Fix $x\in (A\cup B)\cap (A\cup B)$.
  Then, $x\in A\cup B$ and $x\in A\cup C$.
  Either $x$ is in $A$ or it is not.

  If $x$ is in $A$, when $x\in A\cup(B\cap C)$, and we are done.

  Suppose $x$ is not in $A$.
  As $x\in A\cup B$ and $x$ is not in $A$, $x\in B$.
  Similarly, as $x\in A\cup B$ and $x\not\in A$, $x\in C$.
  Thus, $x\in B\cap C$.
  Whence, $x\in A\cup(B\cap C)$.
\end{proof}
