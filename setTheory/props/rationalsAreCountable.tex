\guard

\guard

\guard

\guard

\input{setTheory/defns/injective.tex}
\input{setTheory/defns/surjective.tex}

\begin{defn}
\label{defn:bijection}
\index{bijection}
  A function $f:X\to Y$ is said to be \emph{bijection} provided $f$ is injective and surjective.
\end{defn}


\begin{defn}
\label{defn:sameCardinality}
\index{cardinality!equal}
  For sets $A$ and $B$, $A$ is said to have the \emph{same cardinality} as $B$, written $\vert A\vert = \vert B\vert$, provided there exists a bijection from $A$ to $B$.
\end{defn}

\guard

\guard

\input{setTheory/defns/infinite.tex}

\begin{defn}
\label{defn:countablyInfinite}
\index{finite}
  A sets $A$ is said to be \emph{countably infinite} provided $A$ has the same cardinality as $\NN$.
\end{defn}


\begin{defn}
\label{defn:countablyInfinite}
\index{finite}
  A sets $A$ is said to be \emph{countable} provided $A$ has the same cardinality as $\NN$ or is finite.
\end{defn}


\begin{prop}
\label{prop:integersAreCountable}
  The set of integers, $\ZZ$, is countable.
\end{prop}
\begin{proof}
  To show that $\ZZ$ is countable, we must show that either $\ZZ$ is finite or $\vert \ZZ\vert=\vert \NN\vert$.
  To show that $\vert\ZZ\vert=\vert\NN\vert$ we will produce, explicitly, a bijection from $\ZZ$ to $\NN$.
  To do this, we will keep $0$ where it is, map the negative integers to the even natural numbers, and the positive integers to the odd natural numbers.

  One such mapping is $f:\ZZ\to\NN$ defined be \[ f(z) = \begin{cases}
      0 & z= 0 \\
      -2z & z<0 \\
      2z-1 & z>0
   \end{cases}\,. \]
   It remains to show that this mapping has the desired properties.
   This is left as an exercise to the reader.
\end{proof}

\guard

\guard

\guard

\input{setTheory/defns/cartesianProduct.tex}

\begin{defn}
\label{defn:function}
\index{function}
\index{domain}
\index{codomain}
  Let $X$ and $Y$ be sets.
  We say that $f\subseteq X\times Y$ is a \emph{function} from a set $X$ to a set $Y$, denoted $f:X\to Y$, is a relation from $X$ to $Y$ such that every element of $x$ is related to exactly one element in $Y$.
  For $x\in X$ and $y\in Y$, $y$ is the unique value related to $x$ if, and only if, $(x,y)\in f$.
  In this case, we say that $f(x)=y$ or $f:x\mapsto y$.
\end{defn}


\begin{defn}
\label{defn:injective}
\index{injective}
  A function $f:X\to Y$ is said to be \emph{injective} (or \emph{one-to-one}) provided $\forall x,y\in X$ if $x\not=y$, then $f(x)\not= f(y)$.
\end{defn}

\guard

\guard

\input{setTheory/defns/function.tex}

\begin{defn}
\label{defn:injective}
\index{injective}
  A function $f:X\to Y$ is said to be \emph{injective} (or \emph{one-to-one}) provided $\forall x,y\in X$ if $x\not=y$, then $f(x)\not= f(y)$.
\end{defn}

\guard

\input{setTheory/defns/function.tex}

\begin{defn}
\label{defn:surjective}
\index{surjective}
  A function $f:X\to Y$ is said to be \emph{surjective} (or \emph{onto}) provided $\forall y\in Y\exists x\in X$ $f(x)=y$.
\end{defn}


\begin{defn}
\label{defn:bijection}
\index{bijection}
  A function $f:X\to Y$ is said to be \emph{bijection} provided $f$ is injective and surjective.
\end{defn}


\begin{thm}
\label{thm:schroderBernstein}
  If $f:X\to Y$ and $g:Y\to X$ are injective, then there exists a bijection $h:X\to Y$.
\end{thm}


\begin{prop}
\label{prop:rationalsAreCountable}
  The set of rationals, $\QQ$, is countable.
\end{prop}
\begin{proof}
  We will show this in steps.
  Assume that we have constructed a bijection $f:\ZZ^+\to \QQ^+$.
  Note this map, after multiplying by $-1$, sees that negative rationals are have the same cardinality as $\ZZ$ as well.
  Now, with this bijection, we can form a bijection $g:\ZZ\to\QQ$ defined as as \[ g(z) = \begin{cases}
      0 & z= 0 \\
      -f(z) & z<0 \\
      f(z) & z>0
   \end{cases} \]
   for each $z\in\ZZ$.
   With this bijection in hand, we use the fact that $\ZZ$ is countable, Proposition \ref{prop:integersAreCountable}, together with Proposition \ref{prop:cardinalityIsTransitive} to obtain that $\QQ$ is countable.

   So, it remains to show that there is a bijection $f:\ZZ^+\to\QQ^+$.
   (There is a classic zig-zag argument which informally describes such a bijection. This zig-zagging can be made formal, and a closed form of that description can be written. I will present that in lecture, but for the notes I will show a different way.)
   As $\ZZ^+\subset \QQ^+$, there exists, trivially, an injection $\ZZ^+\to\QQ^+$.
\end{proof}
