\guard

\guard

\guard

\begin{defn}
\label{defn:set}
\index{set}
  A \emph{set} is a collection of things.
\label{defn:membership}
\index{$\in$}
  The symbol $\in$ is used denote the \emph{membership} relation on sets.
  For $S$ a set and $x$ arbitrary, we say $x\in S$ precisely when $x$ is a member (or element) of $S$.
\end{defn}


\begin{defn}
\label{defn:emptyset}
\index{emptyset}
\index{$\emptyset$}
  The set which contains nothing is called the \emph{emptyset}.
  We use $\emptyset$ to denote the emptyset.
\end{defn}

\guard

\guard

\begin{defn}
\label{defn:set}
\index{set}
  A \emph{set} is a collection of things.
\label{defn:membership}
\index{$\in$}
  The symbol $\in$ is used denote the \emph{membership} relation on sets.
  For $S$ a set and $x$ arbitrary, we say $x\in S$ precisely when $x$ is a member (or element) of $S$.
\end{defn}

\guard

\guard

\begin{defn}
\label{defn:set}
\index{set}
  A \emph{set} is a collection of things.
\label{defn:membership}
\index{$\in$}
  The symbol $\in$ is used denote the \emph{membership} relation on sets.
  For $S$ a set and $x$ arbitrary, we say $x\in S$ precisely when $x$ is a member (or element) of $S$.
\end{defn}


\begin{defn}
\label{defn:subset}
\index{subset}
  The \emph{subset} relation, $\subseteq$, is defined as \[ x\subseteq y\iff\forall z(z\in x\rightarrow z\in y)\,.\]
  We use $x\subset y$, or $x\subsetneq y$, to abbreviate $x\subseteq y\wedge x\not=y$.
\index{subset!proper}
  If $x\subsetneq y$, we say that $x$ is a \emph{proper subset} of $y$.
\end{defn}


\begin{defn}
\label{defn:powerSet}
\index{power set}
  Let $A$ be a set.
  The set of all subsets of $A$, $\mathcal{P}$, is called the \emph{power set} of $A$.
  \[ \mathcal{P}(A) := \set{ B\mid B\subseteq A}\,.\]
\end{defn}



\begin{prop}
\label{prop:sizeOfFinitePowerset}
  For all $n\in\NN$, if $X$ is a set with $n$ elements, then $\mathcal{P}$ has $2^n$ elements.
\end{prop}
\begin{proof}
  We prove this by induction on $n$.

  In the case where $n=0$.
  Then, the only set with $0$ elements is the emptyset.
  We now see \[\mathcal{P}(\emptyset) = \set{\emptyset} \,\] has $2^0=1$ elements.

  Now, suppose $n\in\NN$ and the power set of any set with $n$ elements has  $2^n$ elements.
  Let $X$ be a set with $n+1$ many elements.
  As $X$ has $n+1$ many elements, there exists some $x\in X$.
  Fix any $x\in X$.

  Note, $X\setminus\set{x}$ has $n$ elements.
  So, by our inductive hypothesis $\mathcal{P}(X\setminus\set{x})$ has $2^n$ elements.

  $\mathcal{P}(X\setminus\set{x})$ contains half of the elements in $\mathcal{P}(X)$.
  In particular, it is missing all the subsets of $X$ which contain $x$.
  Note \[ \mathcal{P}(X) = \mathcal{P}(X\setminus\set{x}) \cup \set{ A\cup\set{x} \mid A\in \mathcal{P}(X\setminus\set{x}) }\,.\]
  As $\mathcal{P}(X\setminus\set{x})$ has $2^n$ elements, $\set{ A\cup\set{x} \mid A\in \mathcal{P}(X\setminus\set{x}) }$ has $2^n$ many elements.
  Further, as each element of $\mathcal{P}(X\setminus\set{x})$ does not contain $x$ and each element of $\set{ A\cup\set{x} \mid A\in \mathcal{P}(X\setminus\set{x}) }$ does, those two sets are disjoint.
  This, $\mathcal{P}(X)$ has $2\cdot 2^n=2^{n+1}$ elements.
\end{proof}
