\guard
\section{Functions}
\label{sec:functions}

\guard

\guard

\guard

\input{setTheory/defns/set.tex}

\begin{defn}
\label{defn:tuple}
\index{tuple}
  Let $n\in\ZZ^+$ and $x_1,x_2,\dots, x_n$ be given.
  Then, the \emph{$n$-tuple}, often just \emph{tuple}, $(x_1,x_2,\dots,x_n)$ is the collection containing $x_1,x_2,\dots,x_n$ together with their order.
\label{defn:orderedPair}
\index{orderedPair}
  A $2$-tuple is called an \emph{ordered pair}.
\label{defn:orderedTriple}
\index{ordered Triple}
  A $3$-tuple is called an \emph{ordered triple}.
\end{defn}


\begin{defn}
\label{defn:cartesianProduct}
\index{cartesian product}
  Let $A$ and $B$ be given.
  The \emph{caresian product} of $A$ and $B$, written $A\times B$, is the set of all ordered pairs $(a,b)$ where $a\in A$ and $b\in B$.
  \[ A\times B := \set{ (a,b)\mid a\in A,b\in B}\,.\]
\end{defn}


\begin{defn}
\label{defn:function}
\index{function}
\index{domain}
\index{codomain}
  Let $X$ and $Y$ be sets.
  We say that $f\subseteq X\times Y$ is a \emph{function} from a set $X$ to a set $Y$, denoted $f:X\to Y$, is a relation from $X$ to $Y$ such that every element of $x$ is related to exactly one element in $Y$.
  For $x\in X$ and $y\in Y$, $y$ is the unique value related to $x$ if, and only if, $(x,y)\in f$.
  In this case, we say that $f(x)=y$ or $f:x\mapsto y$.
\end{defn}


In some contexts, the symbol $f(x)$ refers to the function itself and not the value of the function.
This can lead to confusion.
We will use $f$ to refer to the function itself and $f(x)$ to refer to the value of the function at some value $x$.

Sometimes, an object presented as a function is not a function at all!
\guard

\guard

\guard

\input{setTheory/defns/tuple.tex}

\begin{defn}
\label{defn:cartesianProduct}
\index{cartesian product}
  Let $A$ and $B$ be given.
  The \emph{caresian product} of $A$ and $B$, written $A\times B$, is the set of all ordered pairs $(a,b)$ where $a\in A$ and $b\in B$.
  \[ A\times B := \set{ (a,b)\mid a\in A,b\in B}\,.\]
\end{defn}


\begin{defn}
\label{defn:function}
\index{function}
\index{domain}
\index{codomain}
  Let $X$ and $Y$ be sets.
  We say that $f\subseteq X\times Y$ is a \emph{function} from a set $X$ to a set $Y$, denoted $f:X\to Y$, is a relation from $X$ to $Y$ such that every element of $x$ is related to exactly one element in $Y$.
  For $x\in X$ and $y\in Y$, $y$ is the unique value related to $x$ if, and only if, $(x,y)\in f$.
  In this case, we say that $f(x)=y$ or $f:x\mapsto y$.
\end{defn}


\begin{exmp}
\label{exmp:nonWellDefinedFunction}
  Define $f:\RR\to\RR$ by for each $x\in X$, $f(x)=y$ where $x^2 + y^2 = 1$.
  Note, this described $f$ is not a function as for almost all $x\in\RR$ there either does not exist such a $y$ or there exists $2$ such $y$.
  In fact, the described $f$ is a function precisely on the set $\set{-1,1}$ as is not particularly interesting at those values.
\end{exmp}

\guard

\guard

%TODO define the integers
%TODO define integer multiplication.

\begin{defn}
\label{defn:rational}
\index{rational}
  A real number $r$ is said to be \emph{rational} if, and only if, $\exists a,b\in\ZZ$ with $b\not=0$ and $rb= a$.
\end{defn}

We denote the set of rational numbers $\QQ$.

\guard

\guard

\input{setTheory/defns/cartesianProduct.tex}

\begin{defn}
\label{defn:function}
\index{function}
\index{domain}
\index{codomain}
  Let $X$ and $Y$ be sets.
  We say that $f\subseteq X\times Y$ is a \emph{function} from a set $X$ to a set $Y$, denoted $f:X\to Y$, is a relation from $X$ to $Y$ such that every element of $x$ is related to exactly one element in $Y$.
  For $x\in X$ and $y\in Y$, $y$ is the unique value related to $x$ if, and only if, $(x,y)\in f$.
  In this case, we say that $f(x)=y$ or $f:x\mapsto y$.
\end{defn}


\begin{exmp}
\label{exmp:nonWellDefinedFunction}
  Define $f:\RR\to\RR$ by for each $x\in X$, $f(x)=y$ where $x^2 + y^2 = 1$.
  Note, this described $f$ is not a function as for almost all $x\in\RR$ there either does not exist such a $y$ or there exists $2$ such $y$.
  In fact, the described $f$ is a function precisely on the set $\set{-1,1}$ as is not particularly interesting at those values.
\end{exmp}


\begin{exercise}
\label{exercise:nonWellDefinedFunction}
  Define $f:\QQ\to\ZZ$ by $f(\frac{n}{m}) = n$.
  Is $f$ well-defined?
  That is, is the described object a function?
\end{exercise}


\guard

\guard

\guard

\input{numberTheory/defns/divides.tex}

\begin{thm}
\label{thm:quotientRemainderTheorem}
  \textbf{(Quotient-Remainder Theorem)}
  Given an integer $n$ and positive integer $d$, there exists unique integers $q$ and $r$ such that $0\leq r < d$ and $n=qd + r$.
\end{thm}
\begin{proof}
  Fix $n\in\ZZ$ and $d\in\ZZ^+$.

  First, we will show existence.
  Consider the set of non-negative integers of the form $n-dq$.
  This set is clearly non-empty.
  Thus, as $\NN$ is well-ordered, this set contains a least element, say $r=n-dq$.
  Certianly, $r\geq 0$.
  Now, $r<d$ as as otherwise $n-d(q+1)$ contradicts the minimality of $r$.

  Now to show uniqueness.
  Suppose that there exists $q_0,q_1,r_0,r_1\in\ZZ$ with $0\leq r_0,r_1 <d$ and \[n=dq_0+r_0=dq_1+r_1\,.\]
  Note, as $dq_0+r_0=dq_1+r_1$, it suffices to show $r_0=r_1$.
  Rewriting $dq_0+r_0=dq_1+r_1$, we see that $r_0-r_1=d(q_1-q_0)$ showing that $d|(r_0-r_1)$.
  Further, as $0\leq r_1 <d$, $-d<r_1\leq 0$.
  So, $-d < r_0-r_1 <d$.
  Though, as $d|(r_0-r_1)$ and $-d < r_0-r_1 <d$, $r_0-r_1=0$.
\end{proof}


\begin{defn}
\label{defn:mod}
\label{defn:div}
\index{mod}
\index{div}
  Let $n$ be an integer and $d$ a positive integer.
  Then the Quotient Remainder Theorem, Theorem \ref{thm:quotientRemainderTheorem}, applies, and there exists unique integers $q$ and $r$ such that $0\leq r<d$ and $n=qd+r$.
  We define \[n\bmod d = r\text~~{ and }~~n\bdiv d = q\] to denote the integer quotient and remainder after division by $d$ respectively.
\end{defn}

\guard

\guard

\input{setTheory/defns/cartesianProduct.tex}

\begin{defn}
\label{defn:function}
\index{function}
\index{domain}
\index{codomain}
  Let $X$ and $Y$ be sets.
  We say that $f\subseteq X\times Y$ is a \emph{function} from a set $X$ to a set $Y$, denoted $f:X\to Y$, is a relation from $X$ to $Y$ such that every element of $x$ is related to exactly one element in $Y$.
  For $x\in X$ and $y\in Y$, $y$ is the unique value related to $x$ if, and only if, $(x,y)\in f$.
  In this case, we say that $f(x)=y$ or $f:x\mapsto y$.
\end{defn}

\guard

\input{setTheory/defns/function.tex}

\begin{defn}
\label{defn:functionDomain}
\index{domain}
  For a given function $f:X\to Y$, we call $X$ the \emph{domain} of $f$, written $\dom(f)$.
\end{defn}

\guard

\input{setTheory/defns/function.tex}

\begin{defn}
\label{defn:functionRange}
\index{range}
\index{codomain}
  For a given function $f:X\to Y$, we call the set \[ \set{ y\in Y \mid \exists x\in X( f(x) = y)}\] the \emph{range} of $f$, and denote this set $\ran(f)$.
  We call the set $Y$ to \emph{codomain} of $f$.
\end{defn}


\begin{prop}
\label{prop:functionEquality}
  Given functions $f,g:X\to Y$, we say that $f=g$ if, and only if, $\forall x\in X(f(x)=g(x))$.
\end{prop}
\begin{proof}
  Let$f,g:X \to Y$ be arbitrary functions.
  As $f$ and $g$ share domain and co-domain, $f,g\subseteq X\times Y$.

  Suppose $\forall x\in X(f(x) = g(x))$.
  Then for any $(x,y)\in X\times Y$,
  \begin{align*}
    (x,y)\in f  &\iff f(x) = y \\
                &\iff g(x) = y \\
                &\iff (x,y)\in g\,.
  \end{align*}
  Thus, $f=g$ as sets.

  For the converse, suppose $f=g$ as sets.
  Fix any $x\in X$.
  Then, for any $y\in Y$
  \begin{align*}
    f(x)=y  &\iff (x,y)\in f \\
            &\iff (x,y)\in g \\
            &\iff g(x) = y\,.
  \end{align*}
  Thus $f(x)=y$ if, and only if, $g(x) = y$.

  Though, for any $x\in X$, there is a unique $y$ such that $f(x)=y$.
  Thus $\forall x\in X( f(x) = g(x) )$.
\end{proof}


\begin{exmp}
\label{exmp:functionEquality}
  Set $X = {0,1,2}$ and defin $f,g:X\to Y$ as \[ f(x) = (x^2+x+1)\mod 3\,,\] and \[ g(x) = (x+1)^2 \mod 3\,.\]
  Does $f=g$?

  As
  \begin{align*}
    f(0)  &= 1  &&= g(0) \\
    f(1)  &= 0  &&= g(1) \\
    f(2)  &= 1  &&= g(2)
  \end{align*}
  Proposition \ref{prop:functionEquality} gives that $f=g$.


  For any function $a,b:\RR\to\RR$, define $(a+b):\RR\to\RR$ as $\forall x\in RR$ $(a+b)(x) := a(x) + b(x)$.
  Let $f,g:\RR\to\RR$ be arbitrary.
  Does $f+g = g+f$.

  Again, yes.
  As $\forall x\in \RR$
  \begin{align*}
    (f+g)(x)  &= f(x) + g(x) \\
              &= g(x) + f(x) \\
              &= (g+f)(x)\,.
  \end{align*}
  So, using the same proposition as before, we have that $f=g$.
\end{exmp}


\guard

\guard

\guard

\input{setTheory/defns/function.tex}
\input{setTheory/defns/set.tex}
\input{setTheory/defns/subset.tex}

\begin{defn}
\label{defn:functionAppliedToSet}
\index{function!impage}
\index{function!pre-impage}
  Let $f:X\to Y$ be a function
  For any $A\subseteq X$, the \emph{image} of $A$ under $f$, written $f''A$ or sometimes $f(A)$, is \[ \set{ y\in Y \mid \exists x\in A( f(x) = y )}\,.\]
  For any $C\subseteq Y$, the \emph{pre-image} of $C$ under $f$, written $f''A$ or sometimes $f(A)$, is \[ \set{ y\in Y \mid \exists x\in A( f(x) = y )}\,.\]
\end{defn}

\guard

\input{logic/defns/conjunction.tex}
\input{logic/defns/conditional.tex}
\input{setTheory/defns/set.tex}

\begin{defn}
\label{defn:subset}
\index{subset}
\index{$\subseteq$}
  The \emph{subset} relation, $\subseteq$, is defined as \[ x\subseteq y\iff\forall z(z\in x\rightarrow z\in y)\,.\]
\index{subset!proper}
\index{$\subset$}
\index{$\subsetneq$}
  We use $x\subset y$, or $x\subsetneq y$, to abbreviate $x\subseteq y\wedge x\not=y$.
  If $x\subsetneq y$, we say that $x$ is a \emph{proper subset} of $y$.
\index{super set}
\index{$\supseteq$}
  We use $A\supseteq B$ to mean $B\subseteq A$, and is read $A$ is a super set of $B$.
  The $\supset$ and $\supsetneq$ variations are similar to the subset variations.
\end{defn}


\begin{prop}
\label{prop:functionAppliedToUnionIsUnionOfFunctionApplied}
  If $f:X\to Y$ is a function and $A,B\subseteq X$, then \[ f[A\cup B ]\subseteq f[A]\cup f[B]\,.\]
\end{prop}
\begin{proof}
  Left as an exercise to the reader.
\end{proof}


\begin{exercise}
\label{exercise:functionAppliedToUnionIsUnionOfFunctionApplied}
  Prove Proposition \ref{prop:functionAppliedToUnionIsUnionOfFunctionApplied}.
\end{exercise}


\guard

\guard

\guard

\input{setTheory/defns/function.tex}

\begin{defn}
\label{defn:injective}
\index{injective}
  A function $f:X\to Y$ is said to be \emph{injective} (or \emph{one-to-one}) provided $\forall x,y\in X$ if $x\not=y$, then $f(x)\not= f(y)$.
\end{defn}


\begin{exmp}
\label{exmp:easyInjectiveExamples}
  \begin{enumerate}
    \item Show that $f:\RR\to\RR$ defined as $f(x) = 4x-1$ is injective.\\
    Like many proofs of infectivity, we will show the contrapositive of Definition \ref{defn:injective}.
    Fix $x,y\in\RR$ and suppose that $f(x)=f(y)$.
    Then,
    \begin{alignat*}{2}
      && f(x)&=f(y) \\
      &\Rightarrow\quad& 4x-1 &= 4y-1 \\
      &\Rightarrow\quad& 4x &= 4y \\
      &\Rightarrow\quad& x &= y \,.\\
    \end{alignat*}
    Thus, $\forall x,y\in\RR(f(x)=f(y)\rightarrow x=y)$.

    \item Show that $g:\RR\to\RR$ defined by $g(x)=x^2$ is not injective.\\
    To do this, we need only show that there exists distinct $x,y\in\RR$ such that $g(x)=g(y)$.
    Note that $-1,1\in\RR$ and $-1\not= 1$, but $g(1)=g(-1)$.

    Note, we can change the domain of $g$ to make the function injective.
    How?
  \end{enumerate}

\end{exmp}

\guard

\guard

\input{setTheory/defns/function.tex}

\begin{defn}
\label{defn:surjective}
\index{surjective}
  A function $f:X\to Y$ is said to be \emph{surjective} (or \emph{onto}) provided $\forall y\in Y\exists x\in X$ $f(x)=y$.
\end{defn}


\begin{exmp}
\label{exmp:easySurjectiveExamples}
  \begin{enumerate}
    \item Show that $f:\RR\to\RR$ defined as $f(x) = 4x-1$ is surjective.\\
    Fix any $y\in \RR$.
    Our job is to find some $x\in\RR$ such that $f(x)=y$.
    Given our selection of $y$, set $x = \frac{y+1}{4}$.
    We claim that this $x$ works,
    \begin{align*}
      f(x)  &=  4x-1 \\
            &=  4\frac{y+1}{4} - 1 \\
            &=  y+1 - 1 \\
            &=  y\,.
    \end{align*}
    Thus, for each $y\in\RR$ we can find some $x\in\RR$ such that $f(x)=y$.
    Thus, $f$ is surjective.

    \item Show that $g:\RR\to\RR$ defined by $g(x)=x^2$ is not surjective.\\
    To do this, we need only produce some $y\in\RR$ such that there does not exist a $x\in\RR$ such that $g(x)=y$.
    Consider $y=-1$.
    Note, $\forall x\in\RR$ $x^2\geq 0$, thus there does not exist an $x\in\RR$ such that $g(x)=-1$.
    Thus, $g$ is not surjective.

    Note, we can change the codomain of $g$ to make the function surjective.
    How? 
  \end{enumerate}

\end{exmp}


\guard

\input{setTheory/defns/injection.tex}
\input{setTheory/defns/surjection.tex}

\begin{defn}
\label{defn:bijection}
\index{bijection}
  A function $f:X\to Y$ is said to be \emph{bijection} provided $f$ is injective and surjective.
\end{defn}


\begin{exmp}
\label{exmp:easyBijectionExamples}
  \begin{enumerate}
    \item Show that $f(x) = 4x-1$ is a bijection.\\

    In Examples \ref{exmp:easyInjectiveExamples} and \ref{exmp:easySurjectiveExamples} respectively, we showed that $f$ is injective and surjective.
    So, as $f$ is both an injection and a surjection, $f$ is a bijection.

    \item Show that $f:\RR^2\to\RR^2$ defined by \[ g(x,y) = (x+y,x-y)\] is a bijection.\\

    First, we show that $g$ is injective.
    Fix $x,y,a,b\in\RR$ such that $g(x,y) = g(a,b)$.
    Then by definiton of $g$, we have that
    \begin{alignat*}{2}
      && g(x,y)&=g(a,b) \\
      &\Rightarrow\quad& (x+y,x-y) &= (a+b,a-b) \\
      &\Rightarrow\quad& (x+y = a+b)\wedge (x-y=a-b)\,.
    \end{alignat*}
    Adding these two equalities gives that $ 2x = 2a$.
    So, $x=a$.
    Now, as $x=a$ and $x+y = a+b$, we have that $y=b$.
    Thus, $(x,y)=(a,b)$.
    So $g$ is injective.

    Now, to show that $g$ is surjective.
    Fix $u,v\in\RR$.
    Let $x = \frac{u+v}{2}$ and $y=\frac{u-v}{2}$.
    Then, we can note that
    \begin{align*}
      g(x,y)  &= \left( \frac{u+v}{2}+\frac{u-v}{2}, \frac{u+v}{2}-\frac{u-v}{2} \right)\\
              &= \left( \frac{u+v+u-v}{2}, \frac{u+v-u+v}{2} \right)\\
              &= \left( \frac{2u}{2}, \frac{2v}{2} \right)\\
              &= \left( u, v \right)
    \end{align*}
    So, $g$ is injective.
  \end{enumerate}

\end{exmp}


\guard

\guard

\guard

\input{setTheory/defns/injective.tex}

\begin{exmp}
\label{exmp:easyInjectiveExamples}
  \begin{enumerate}
    \item Show that $f:\RR\to\RR$ defined as $f(x) = 4x-1$ is injective.\\
    Like many proofs of infectivity, we will show the contrapositive of Definition \ref{defn:injective}.
    Fix $x,y\in\RR$ and suppose that $f(x)=f(y)$.
    Then,
    \begin{alignat*}{2}
      && f(x)&=f(y) \\
      &\Rightarrow\quad& 4x-1 &= 4y-1 \\
      &\Rightarrow\quad& 4x &= 4y \\
      &\Rightarrow\quad& x &= y \,.\\
    \end{alignat*}
    Thus, $\forall x,y\in\RR(f(x)=f(y)\rightarrow x=y)$.

    \item Show that $g:\RR\to\RR$ defined by $g(x)=x^2$ is not injective.\\
    To do this, we need only show that there exists distinct $x,y\in\RR$ such that $g(x)=g(y)$.
    Note that $-1,1\in\RR$ and $-1\not= 1$, but $g(1)=g(-1)$.

    Note, we can change the domain of $g$ to make the function injective.
    How?
  \end{enumerate}

\end{exmp}

\guard

\input{setTheory/defns/surjective.tex}

\begin{exmp}
\label{exmp:easySurjectiveExamples}
  \begin{enumerate}
    \item Show that $f:\RR\to\RR$ defined as $f(x) = 4x-1$ is surjective.\\
    Fix any $y\in \RR$.
    Our job is to find some $x\in\RR$ such that $f(x)=y$.
    Given our selection of $y$, set $x = \frac{y+1}{4}$.
    We claim that this $x$ works,
    \begin{align*}
      f(x)  &=  4x-1 \\
            &=  4\frac{y+1}{4} - 1 \\
            &=  y+1 - 1 \\
            &=  y\,.
    \end{align*}
    Thus, for each $y\in\RR$ we can find some $x\in\RR$ such that $f(x)=y$.
    Thus, $f$ is surjective.

    \item Show that $g:\RR\to\RR$ defined by $g(x)=x^2$ is not surjective.\\
    To do this, we need only produce some $y\in\RR$ such that there does not exist a $x\in\RR$ such that $g(x)=y$.
    Consider $y=-1$.
    Note, $\forall x\in\RR$ $x^2\geq 0$, thus there does not exist an $x\in\RR$ such that $g(x)=-1$.
    Thus, $g$ is not surjective.

    Note, we can change the codomain of $g$ to make the function surjective.
    How? 
  \end{enumerate}

\end{exmp}


\guard

\input{setTheory/defns/injection.tex}
\input{setTheory/defns/surjection.tex}

\begin{defn}
\label{defn:bijection}
\index{bijection}
  A function $f:X\to Y$ is said to be \emph{bijection} provided $f$ is injective and surjective.
\end{defn}


\begin{exmp}
\label{exmp:easyBijectionExamples}
  \begin{enumerate}
    \item Show that $f(x) = 4x-1$ is a bijection.\\

    In Examples \ref{exmp:easyInjectiveExamples} and \ref{exmp:easySurjectiveExamples} respectively, we showed that $f$ is injective and surjective.
    So, as $f$ is both an injection and a surjection, $f$ is a bijection.

    \item Show that $f:\RR^2\to\RR^2$ defined by \[ g(x,y) = (x+y,x-y)\] is a bijection.\\

    First, we show that $g$ is injective.
    Fix $x,y,a,b\in\RR$ such that $g(x,y) = g(a,b)$.
    Then by definiton of $g$, we have that
    \begin{alignat*}{2}
      && g(x,y)&=g(a,b) \\
      &\Rightarrow\quad& (x+y,x-y) &= (a+b,a-b) \\
      &\Rightarrow\quad& (x+y = a+b)\wedge (x-y=a-b)\,.
    \end{alignat*}
    Adding these two equalities gives that $ 2x = 2a$.
    So, $x=a$.
    Now, as $x=a$ and $x+y = a+b$, we have that $y=b$.
    Thus, $(x,y)=(a,b)$.
    So $g$ is injective.

    Now, to show that $g$ is surjective.
    Fix $u,v\in\RR$.
    Let $x = \frac{u+v}{2}$ and $y=\frac{u-v}{2}$.
    Then, we can note that
    \begin{align*}
      g(x,y)  &= \left( \frac{u+v}{2}+\frac{u-v}{2}, \frac{u+v}{2}-\frac{u-v}{2} \right)\\
              &= \left( \frac{u+v+u-v}{2}, \frac{u+v-u+v}{2} \right)\\
              &= \left( \frac{2u}{2}, \frac{2v}{2} \right)\\
              &= \left( u, v \right)
    \end{align*}
    So, $g$ is injective.
  \end{enumerate}

\end{exmp}

\guard

\guard

\input{setTheory/defns/bijection.tex}

\begin{prop}
\label{prop:existenceOfInverseFunction}
  Let $f:X\to Y$ be a bijection.
  Then, there exists a unique function $f^{-1}:Y\to X$ defined by $\forall y\in Y$ $f^{-1}(y)=x$ where $x$ is the unique element of $X$ such that $f(x)=y$.
  That is, $\forall x\in X\forall y\in Y$ \[ f^{-1}(y)=x\quad\iff\quad f(x)=y\,.\]
\end{prop}
\begin{proof}

\end{proof}


\begin{defn}
\label{defn:bijection}
\index{bijection}
  $f^{-1}$ in Proposition \ref{prop:existenceOfInverseFunction} is called the \emph{inverse} of $f$.
\end{defn}


\begin{exmp}
\label{exmp:easyInverseExamples}
  \begin{enumerate}
    \item Compute $f^{-1}$ given $f(x) = 4x-1$.\\

    First, we performa sanity check and ask if $f^{-1}$ makes sense given our function $f$.
    As we showed in Example \ref{exmp:easyBijectionExamples}, $f$ as above is a bijection.
    So, this by Proposition \ref{prop:existenceOfInverseFunction} $f^{-1}$ does exists.
    In fact, we have computed the inverse already, in Example \ref{exmp:easySurjectiveExamples}, it is \[ f^{-1}(y) = \frac{y+1}{4}\,.\]

    \item Compute the inverse of the function $g(x,y) = (x+y,x-y)$.\\

    Again, we have already shown that this $g$ is a bijection.
    Further, in Example \ref{exmp:easyBijectionExamples}, while showing that $g$ was a surjection, we nearly computed the inverse of $g$.
    Putting together the work we did there, it is not difficult to see that \[ g^{-1}(u,v) = \left( \frac{ u+v}{2}, \frac{u-v}{2} \right)\,.\]

  \end{enumerate}

\end{exmp}

\guard

\guard

\input{setTheory/defns/injection.tex}
\input{setTheory/defns/surjection.tex}

\begin{defn}
\label{defn:bijection}
\index{bijection}
  A function $f:X\to Y$ is said to be \emph{bijection} provided $f$ is injective and surjective.
\end{defn}

\guard

\guard

\input{setTheory/defns/bijection.tex}

\begin{prop}
\label{prop:existenceOfInverseFunction}
  Let $f:X\to Y$ be a bijection.
  Then, there exists a unique function $f^{-1}:Y\to X$ defined by $\forall y\in Y$ $f^{-1}(y)=x$ where $x$ is the unique element of $X$ such that $f(x)=y$.
  That is, $\forall x\in X\forall y\in Y$ \[ f^{-1}(y)=x\quad\iff\quad f(x)=y\,.\]
\end{prop}
\begin{proof}

\end{proof}


\begin{defn}
\label{defn:bijection}
\index{bijection}
  $f^{-1}$ in Proposition \ref{prop:existenceOfInverseFunction} is called the \emph{inverse} of $f$.
\end{defn}


\begin{prop}
\label{prop:inverseOfBijectionIsBijection}
  If $f:X\to Y$ is a bijections, then $f^{-1}$ is a bijection.
\end{prop}
\begin{proof}
  Fix $f:X\to Y$ a bijection.
  We must show that $f^{-1}$ is an injection and a surjection.

  To show that $f^{-1}:Y\to X$ is an injection, fix $y_1,y_2\in Y$.
  Suppose that $f^{-1}(y_1)=f^{-1}(y_2)$.
  Set $x = f^{-1}(y_1)\in X$.
  By definition of $f^{-1}$, $f(x)=y_1$
  Further, as $f^{-1}(y_1)=f^{-1}(y_2)$, $x = f^{-1}(y_2)$.
  Whence, $f(x)=y_2$.
  So, as $f(x)=y_1$ and $f(x)=y_2$, $y_1=y_2$.

  To show that $f^{-1}$ is a surjection, fix some $x\in X$.
  Now, set $y=f(x)$.
  Then, by definition of $f^{-1}$, $f^{-1}(y)=x$.
\end{proof}

\guard

\guard

\guard

\input{setTheory/defns/cartesianProduct.tex}

\begin{defn}
\label{defn:function}
\index{function}
\index{domain}
\index{codomain}
  Let $X$ and $Y$ be sets.
  We say that $f\subseteq X\times Y$ is a \emph{function} from a set $X$ to a set $Y$, denoted $f:X\to Y$, is a relation from $X$ to $Y$ such that every element of $x$ is related to exactly one element in $Y$.
  For $x\in X$ and $y\in Y$, $y$ is the unique value related to $x$ if, and only if, $(x,y)\in f$.
  In this case, we say that $f(x)=y$ or $f:x\mapsto y$.
\end{defn}

\guard

\input{setTheory/defns/function.tex}

\begin{defn}
\label{defn:functionDomain}
\index{domain}
  For a given function $f:X\to Y$, we call $X$ the \emph{domain} of $f$, written $\dom(f)$.
\end{defn}

\guard

\input{setTheory/defns/function.tex}

\begin{defn}
\label{defn:functionRange}
\index{range}
\index{codomain}
  For a given function $f:X\to Y$, we call the set \[ \set{ y\in Y \mid \exists x\in X( f(x) = y)}\] the \emph{range} of $f$, and denote this set $\ran(f)$.
  We call the set $Y$ to \emph{codomain} of $f$.
\end{defn}


\begin{prop}
\label{prop:functionEquality}
  Given functions $f,g:X\to Y$, we say that $f=g$ if, and only if, $\forall x\in X(f(x)=g(x))$.
\end{prop}
\begin{proof}
  Let$f,g:X \to Y$ be arbitrary functions.
  As $f$ and $g$ share domain and co-domain, $f,g\subseteq X\times Y$.

  Suppose $\forall x\in X(f(x) = g(x))$.
  Then for any $(x,y)\in X\times Y$,
  \begin{align*}
    (x,y)\in f  &\iff f(x) = y \\
                &\iff g(x) = y \\
                &\iff (x,y)\in g\,.
  \end{align*}
  Thus, $f=g$ as sets.

  For the converse, suppose $f=g$ as sets.
  Fix any $x\in X$.
  Then, for any $y\in Y$
  \begin{align*}
    f(x)=y  &\iff (x,y)\in f \\
            &\iff (x,y)\in g \\
            &\iff g(x) = y\,.
  \end{align*}
  Thus $f(x)=y$ if, and only if, $g(x) = y$.

  Though, for any $x\in X$, there is a unique $y$ such that $f(x)=y$.
  Thus $\forall x\in X( f(x) = g(x) )$.
\end{proof}

\guard

\guard

\input{setTheory/defns/cartesianProduct.tex}

\begin{defn}
\label{defn:function}
\index{function}
\index{domain}
\index{codomain}
  Let $X$ and $Y$ be sets.
  We say that $f\subseteq X\times Y$ is a \emph{function} from a set $X$ to a set $Y$, denoted $f:X\to Y$, is a relation from $X$ to $Y$ such that every element of $x$ is related to exactly one element in $Y$.
  For $x\in X$ and $y\in Y$, $y$ is the unique value related to $x$ if, and only if, $(x,y)\in f$.
  In this case, we say that $f(x)=y$ or $f:x\mapsto y$.
\end{defn}


\begin{defn}
\label{defn:functionComposition}
\index{function!composition}
  Let $f:X\to Y$ and $g:Y\to Z$.
  The \emph{composition} of $f$ and $g$, written $g\circ f:X\to Z$, is defined by $(g\circ f )(x) := g(f(x))$ for each $x\in X$.
\end{defn}

\guard

\guard

\begin{defn}
\label{defn:set}
\index{set}
  A \emph{set} is a collection of things.
\label{defn:membership}
\index{$\in$}
  The symbol $\in$ is used denote the \emph{membership} relation on sets.
  For $S$ a set and $x$ arbitrary, we say $x\in S$ precisely when $x$ is a member (or element) of $S$.
\end{defn}

\guard

\input{setTheory/defns/cartesianProduct.tex}

\begin{defn}
\label{defn:function}
\index{function}
\index{domain}
\index{codomain}
  Let $X$ and $Y$ be sets.
  We say that $f\subseteq X\times Y$ is a \emph{function} from a set $X$ to a set $Y$, denoted $f:X\to Y$, is a relation from $X$ to $Y$ such that every element of $x$ is related to exactly one element in $Y$.
  For $x\in X$ and $y\in Y$, $y$ is the unique value related to $x$ if, and only if, $(x,y)\in f$.
  In this case, we say that $f(x)=y$ or $f:x\mapsto y$.
\end{defn}


\begin{defn}
\label{defn:identityFunction}
\index{function!identity}
  Given a set $X$, the \emph{identity} function on $X$, $\id_X:X\to X$, is defined as $\id_X(x)=x$ for all $x\in X$.
\end{defn}


\begin{prop}
\label{prop:compositionIdentity}
  Let $f:X\to Y$ and $\id_X$, $\id_Y$ the identity functions on $X$ and $Y$ respectively.
  Then $f\circ\id_X=f$ and $\id_Y\circ f = f$.
\end{prop}
\begin{proof}
  Fix any $x\in X$.
  Then,
  \begin{align*}
    (f\circ\id_X)(x)  &= f(\id_X(x)) \\
                      &= f(x)
  \end{align*}
  So, $\forall x\in X(\,(f\circ\id_X)(x)=f(x)\,)$.
  Thus, by Proposition \ref{prop:functionEquality}, $f\circ\id_X = f$.

  $\id_Y\circ f = f$ is similar.
  Fix any $x\in X$.
  Then,
  \begin{align*}
    (\id_Y\circ f)(x) &= \id_Y( f(x)) \\
                      &= f(x) \\
  \end{align*}
  So, again by Proposition \ref{prop:functionEquality}, we have that $\id_Y\circ f = f$.
\end{proof}


\guard

\guard

\guard

\input{setTheory/defns/cartesianProduct.tex}

\begin{defn}
\label{defn:function}
\index{function}
\index{domain}
\index{codomain}
  Let $X$ and $Y$ be sets.
  We say that $f\subseteq X\times Y$ is a \emph{function} from a set $X$ to a set $Y$, denoted $f:X\to Y$, is a relation from $X$ to $Y$ such that every element of $x$ is related to exactly one element in $Y$.
  For $x\in X$ and $y\in Y$, $y$ is the unique value related to $x$ if, and only if, $(x,y)\in f$.
  In this case, we say that $f(x)=y$ or $f:x\mapsto y$.
\end{defn}

\guard

\input{setTheory/defns/function.tex}

\begin{defn}
\label{defn:functionDomain}
\index{domain}
  For a given function $f:X\to Y$, we call $X$ the \emph{domain} of $f$, written $\dom(f)$.
\end{defn}

\guard

\input{setTheory/defns/function.tex}

\begin{defn}
\label{defn:functionRange}
\index{range}
\index{codomain}
  For a given function $f:X\to Y$, we call the set \[ \set{ y\in Y \mid \exists x\in X( f(x) = y)}\] the \emph{range} of $f$, and denote this set $\ran(f)$.
  We call the set $Y$ to \emph{codomain} of $f$.
\end{defn}


\begin{prop}
\label{prop:functionEquality}
  Given functions $f,g:X\to Y$, we say that $f=g$ if, and only if, $\forall x\in X(f(x)=g(x))$.
\end{prop}
\begin{proof}
  Let$f,g:X \to Y$ be arbitrary functions.
  As $f$ and $g$ share domain and co-domain, $f,g\subseteq X\times Y$.

  Suppose $\forall x\in X(f(x) = g(x))$.
  Then for any $(x,y)\in X\times Y$,
  \begin{align*}
    (x,y)\in f  &\iff f(x) = y \\
                &\iff g(x) = y \\
                &\iff (x,y)\in g\,.
  \end{align*}
  Thus, $f=g$ as sets.

  For the converse, suppose $f=g$ as sets.
  Fix any $x\in X$.
  Then, for any $y\in Y$
  \begin{align*}
    f(x)=y  &\iff (x,y)\in f \\
            &\iff (x,y)\in g \\
            &\iff g(x) = y\,.
  \end{align*}
  Thus $f(x)=y$ if, and only if, $g(x) = y$.

  Though, for any $x\in X$, there is a unique $y$ such that $f(x)=y$.
  Thus $\forall x\in X( f(x) = g(x) )$.
\end{proof}

\guard

\guard

\input{setTheory/defns/cartesianProduct.tex}

\begin{defn}
\label{defn:function}
\index{function}
\index{domain}
\index{codomain}
  Let $X$ and $Y$ be sets.
  We say that $f\subseteq X\times Y$ is a \emph{function} from a set $X$ to a set $Y$, denoted $f:X\to Y$, is a relation from $X$ to $Y$ such that every element of $x$ is related to exactly one element in $Y$.
  For $x\in X$ and $y\in Y$, $y$ is the unique value related to $x$ if, and only if, $(x,y)\in f$.
  In this case, we say that $f(x)=y$ or $f:x\mapsto y$.
\end{defn}


\begin{defn}
\label{defn:functionComposition}
\index{function!composition}
  Let $f:X\to Y$ and $g:Y\to Z$.
  The \emph{composition} of $f$ and $g$, written $g\circ f:X\to Z$, is defined by $(g\circ f )(x) := g(f(x))$ for each $x\in X$.
\end{defn}

\guard

\guard

\begin{defn}
\label{defn:set}
\index{set}
  A \emph{set} is a collection of things.
\label{defn:membership}
\index{$\in$}
  The symbol $\in$ is used denote the \emph{membership} relation on sets.
  For $S$ a set and $x$ arbitrary, we say $x\in S$ precisely when $x$ is a member (or element) of $S$.
\end{defn}

\guard

\input{setTheory/defns/cartesianProduct.tex}

\begin{defn}
\label{defn:function}
\index{function}
\index{domain}
\index{codomain}
  Let $X$ and $Y$ be sets.
  We say that $f\subseteq X\times Y$ is a \emph{function} from a set $X$ to a set $Y$, denoted $f:X\to Y$, is a relation from $X$ to $Y$ such that every element of $x$ is related to exactly one element in $Y$.
  For $x\in X$ and $y\in Y$, $y$ is the unique value related to $x$ if, and only if, $(x,y)\in f$.
  In this case, we say that $f(x)=y$ or $f:x\mapsto y$.
\end{defn}


\begin{defn}
\label{defn:identityFunction}
\index{function!identity}
  Given a set $X$, the \emph{identity} function on $X$, $\id_X:X\to X$, is defined as $\id_X(x)=x$ for all $x\in X$.
\end{defn}


\begin{prop}
\label{prop:compositionIdentity}
  Let $f:X\to Y$ and $\id_X$, $\id_Y$ the identity functions on $X$ and $Y$ respectively.
  Then $f\circ\id_X=f$ and $\id_Y\circ f = f$.
\end{prop}
\begin{proof}
  Fix any $x\in X$.
  Then,
  \begin{align*}
    (f\circ\id_X)(x)  &= f(\id_X(x)) \\
                      &= f(x)
  \end{align*}
  So, $\forall x\in X(\,(f\circ\id_X)(x)=f(x)\,)$.
  Thus, by Proposition \ref{prop:functionEquality}, $f\circ\id_X = f$.

  $\id_Y\circ f = f$ is similar.
  Fix any $x\in X$.
  Then,
  \begin{align*}
    (\id_Y\circ f)(x) &= \id_Y( f(x)) \\
                      &= f(x) \\
  \end{align*}
  So, again by Proposition \ref{prop:functionEquality}, we have that $\id_Y\circ f = f$.
\end{proof}

\guard

\guard

\guard

\input{setTheory/defns/cartesianProduct.tex}

\begin{defn}
\label{defn:function}
\index{function}
\index{domain}
\index{codomain}
  Let $X$ and $Y$ be sets.
  We say that $f\subseteq X\times Y$ is a \emph{function} from a set $X$ to a set $Y$, denoted $f:X\to Y$, is a relation from $X$ to $Y$ such that every element of $x$ is related to exactly one element in $Y$.
  For $x\in X$ and $y\in Y$, $y$ is the unique value related to $x$ if, and only if, $(x,y)\in f$.
  In this case, we say that $f(x)=y$ or $f:x\mapsto y$.
\end{defn}

\guard

\input{setTheory/defns/function.tex}

\begin{defn}
\label{defn:functionDomain}
\index{domain}
  For a given function $f:X\to Y$, we call $X$ the \emph{domain} of $f$, written $\dom(f)$.
\end{defn}

\guard

\input{setTheory/defns/function.tex}

\begin{defn}
\label{defn:functionRange}
\index{range}
\index{codomain}
  For a given function $f:X\to Y$, we call the set \[ \set{ y\in Y \mid \exists x\in X( f(x) = y)}\] the \emph{range} of $f$, and denote this set $\ran(f)$.
  We call the set $Y$ to \emph{codomain} of $f$.
\end{defn}


\begin{prop}
\label{prop:functionEquality}
  Given functions $f,g:X\to Y$, we say that $f=g$ if, and only if, $\forall x\in X(f(x)=g(x))$.
\end{prop}
\begin{proof}
  Let$f,g:X \to Y$ be arbitrary functions.
  As $f$ and $g$ share domain and co-domain, $f,g\subseteq X\times Y$.

  Suppose $\forall x\in X(f(x) = g(x))$.
  Then for any $(x,y)\in X\times Y$,
  \begin{align*}
    (x,y)\in f  &\iff f(x) = y \\
                &\iff g(x) = y \\
                &\iff (x,y)\in g\,.
  \end{align*}
  Thus, $f=g$ as sets.

  For the converse, suppose $f=g$ as sets.
  Fix any $x\in X$.
  Then, for any $y\in Y$
  \begin{align*}
    f(x)=y  &\iff (x,y)\in f \\
            &\iff (x,y)\in g \\
            &\iff g(x) = y\,.
  \end{align*}
  Thus $f(x)=y$ if, and only if, $g(x) = y$.

  Though, for any $x\in X$, there is a unique $y$ such that $f(x)=y$.
  Thus $\forall x\in X( f(x) = g(x) )$.
\end{proof}

\guard

\guard

\input{logic/defns/statement.tex}
\input{logic/defns/statementForm.tex}
\input{logic/defns/truthValue.tex}


\begin{defn}
\label{defn:conditionalStatement}
\index{conditional}
  Let $p$ and $q$ be statements forms.
  The \emph{conditional statement} ``$p$ implies $q$'', written $p \rightarrow q$, is the statement form that is false precisely when $p$ is true and $q$ is false ( that is, when the statement ``If $p$, then $q$'' is violated ).

  In the conditional $p\rightarrow q$, $p$ is refered to as the \emph{hypothesis} and $q$ is called the \emph{conclusion}.
\end{defn}


\begin{defn}
\label{defn:inverse}
\index{inverse}
  The \emph{inverse} of a conditional statement form $p\rightarrow q$ is the statement form $\neg p\rightarrow \neg q$.
\end{defn}

\guard

\guard

\input{setTheory/defns/cartesianProduct.tex}

\begin{defn}
\label{defn:function}
\index{function}
\index{domain}
\index{codomain}
  Let $X$ and $Y$ be sets.
  We say that $f\subseteq X\times Y$ is a \emph{function} from a set $X$ to a set $Y$, denoted $f:X\to Y$, is a relation from $X$ to $Y$ such that every element of $x$ is related to exactly one element in $Y$.
  For $x\in X$ and $y\in Y$, $y$ is the unique value related to $x$ if, and only if, $(x,y)\in f$.
  In this case, we say that $f(x)=y$ or $f:x\mapsto y$.
\end{defn}


\begin{defn}
\label{defn:functionComposition}
\index{function!composition}
  Let $f:X\to Y$ and $g:Y\to Z$.
  The \emph{composition} of $f$ and $g$, written $g\circ f:X\to Z$, is defined by $(g\circ f )(x) := g(f(x))$ for each $x\in X$.
\end{defn}



\begin{prop}
\label{prop:inverseOfFunction}
  If $f:X\to Y$ is a bijection with inverse $f^{-1}:Y\to X$, then $f^{-1}\circ f=\id_X$.
\end{prop}
\begin{proof}
  Fix $f:X\to Y$ a bijection.
  Fix $x\in X$.
  Set $y=f(x)$.
  By definition of $f^{-1}$, $f^{-1}(y) = x$.
  So, we have that
  \begin{align*}
    (f^{-1}\circ f )(x) &= f^{-1}(f(x)) \\
                        &= f^{-1}(y) \\
                        &= x\,.
  \end{align*}
  Though, as $x\in X$ was arbitrary, we have that $\forall x\in X( f^{-1}\circ f=\id_X)$.
  Thus, by Proposition \ref{prop:functionEquality}, $f^{-1}\circ f=\id_X$.
\end{proof}

